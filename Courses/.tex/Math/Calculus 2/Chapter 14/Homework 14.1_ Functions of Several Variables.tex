\documentclass{article}
\usepackage{amsmath}
\usepackage{amssymb}
\usepackage[margin=1in]{geometry}
\usepackage{graphicx}
\usepackage{enumitem}

\title{Homework 14.1: Functions of Several Variables}
\author{Tashfeen Omran}
\date{\today}

\begin{document}

\maketitle

\section{Comprehensive Introduction and Prerequisites}

\subsection{Core Concepts}

This chapter introduces functions of several variables, extending the familiar concept of a single-variable function, $y = f(x)$, to functions that depend on two or more independent variables.

\subsubsection{Functions of Two Variables}
A function of two variables, denoted $f(x, y)$, assigns to each ordered pair of real numbers $(x, y)$ in a set $D$ a unique real number $z$.
\begin{itemize}
    \item \textbf{Notation:} We write $z = f(x, y)$.
    \item \textbf{Domain (D):} The domain is the set of all possible input pairs $(x, y)$ for which the function is defined. It is a subset of the $xy$-plane ($\mathbb{R}^2$).
    \item \textbf{Range:} The range is the set of all possible output values $z$ that the function can produce. It is a subset of the real numbers ($\mathbb{R}$).
    \item \textbf{Graph:} The graph of a function of two variables is the set of all points $(x, y, z)$ in three-dimensional space ($\mathbb{R}^3$) such that $z = f(x, y)$ and $(x, y)$ is in the domain $D$. The graph is typically a \textbf{surface}.
\end{itemize}

\subsubsection{Functions of Three Variables}
Similarly, a function of three variables, $f(x, y, z)$, assigns to each ordered triple $(x, y, z)$ in a domain $D \subseteq \mathbb{R}^3$ a unique real number $w$.
\begin{itemize}
    \item \textbf{Notation:} We write $w = f(x, y, z)$.
    \item \textbf{Visualization:} We cannot visualize the graph of a function of three variables because it would exist in four-dimensional space. However, we can analyze it mathematically.
\end{itemize}

\subsubsection{Level Curves and Level Surfaces}
Since visualizing graphs in 3D (or 4D) can be difficult, we use other tools:
\begin{itemize}
    \item \textbf{Level Curves (for $f(x, y)$):} A level curve of a function $f$ of two variables is the set of all points $(x, y)$ in the domain where the function has a constant value, i.e., $f(x, y) = k$ for some constant $k$. A collection of level curves is called a \textbf{contour map}. Think of a topographic map where each line represents a constant elevation.
    \item \textbf{Level Surfaces (for $f(x, y, z)$):} For a function of three variables, the set of points $(x, y, z)$ where $f(x, y, z) = k$ is called a \textbf{level surface}. For example, if $f(x, y, z)$ represents the temperature at a point in a room, the level surface for $k=70$ would be a surface connecting all points where the temperature is exactly $70^\circ$F.
\end{itemize}

\subsection{Intuition and Derivation}
The core idea is to generalize the input of a function from a single point on a line to a point in a plane or in space. The "why" behind the domain restrictions is a direct extension of single-variable calculus: we must avoid operations that are mathematically undefined.
\begin{itemize}
    \item \textbf{Division by Zero:} A function $f(x, y) = \frac{g(x, y)}{h(x, y)}$ is undefined wherever the denominator $h(x, y) = 0$.
    \item \textbf{Even Roots:} A function $f(x, y) = \sqrt{g(x, y)}$ is only defined for real numbers where the radicand is non-negative, so we require $g(x, y) \ge 0$.
    \item \textbf{Logarithms:} A function $f(x, y) = \ln(g(x, y))$ is only defined where its argument is positive, so we require $g(x, y) > 0$.
\end{itemize}
The domain is the region in the $xy$-plane (or xyz-space) where all such conditions are simultaneously met.

\subsection{Key Formulas}
This topic is more about concepts than formulas. The key "formulas" are the equations we use to describe graphs, domains, and level curves.
\begin{itemize}
    \item Graph of a function of two variables: $z = f(x, y)$
    \item Level curve: $f(x, y) = k$
    \item Level surface: $f(x, y, z) = k$
    \item Standard Surfaces: It is useful to recognize the equations of common surfaces.
    \begin{itemize}
        \item Plane: $ax + by + cz = d$
        \item Sphere: $(x-h)^2 + (y-k)^2 + (z-l)^2 = r^2$
        \item Elliptic Paraboloid: $z = \frac{x^2}{a^2} + \frac{y^2}{b^2}$ (opens up)
        \item Cone: $z^2 = \frac{x^2}{a^2} + \frac{y^2}{b^2}$
        \item Cylinder: If one variable is missing from the equation, the graph is a cylinder. For example, $x^2 + y^2 = 1$ is a cylinder in $\mathbb{R}^3$ that extends infinitely along the $z$-axis.
    \end{itemize}
\end{itemize}

\subsection{Prerequisites}
\begin{itemize}
    \item \textbf{Function Evaluation:} Comfort with substituting values into functions.
    \item \textbf{Algebra:} Solving equations and, crucially, inequalities. Manipulating expressions to identify standard forms.
    \item \textbf{2D Coordinate Geometry:} Graphing lines, parabolas, circles, ellipses, and hyperbolas. This is essential for sketching domains and level curves.
    \item \textbf{3D Coordinate System:} Understanding how to plot points and visualize basic surfaces in xyz-space.
    \item \textbf{Properties of Functions:} Knowing the domains of basic functions like $\sqrt{x}$, $\ln(x)$, $\arcsin(x)$, etc.
\end{itemize}

\section{Detailed Homework Solutions}

\subsection{Problem 1}
If $f(x, y) = \frac{x^2y}{2x - y^2}$, find:
\begin{enumerate}[label=\alph*.]
    \item \textbf{$f(1, 3)$} \\
    Substitute $x=1$ and $y=3$:
    \[ f(1, 3) = \frac{(1)^2(3)}{2(1) - (3)^2} = \frac{3}{2 - 9} = \frac{3}{-7} = -\frac{3}{7} \]
    \textbf{Answer:} $-\frac{3}{7}$

    \item \textbf{$f(-2, -1)$} \\
    Substitute $x=-2$ and $y=-1$:
    \[ f(-2, -1) = \frac{(-2)^2(-1)}{2(-2) - (-1)^2} = \frac{4(-1)}{-4 - 1} = \frac{-4}{-5} = \frac{4}{5} \]
    \textbf{Answer:} $\frac{4}{5}$

    \item \textbf{$f(x+h, y)$} \\
    Substitute $x+h$ for $x$:
    \[ f(x+h, y) = \frac{(x+h)^2y}{2(x+h) - y^2} = \frac{(x^2 + 2xh + h^2)y}{2x + 2h - y^2} \]
    \textbf{Answer:} $\frac{(x^2 + 2xh + h^2)y}{2x + 2h - y^2}$

    \item \textbf{$f(x, x)$} \\
    Substitute $x$ for $y$:
    \[ f(x, x) = \frac{x^2(x)}{2x - x^2} = \frac{x^3}{x(2 - x)} = \frac{x^2}{2 - x} \]
    This is defined for $x \neq 0$ and $x \neq 2$. \\
    \textbf{Answer:} $\frac{x^2}{2-x}$
\end{enumerate}

\subsection{Problem 2}
If $g(x, y) = x \sin y + y \sin x$, find:
\begin{enumerate}[label=\alph*.]
    \item \textbf{$g(\pi, 0)$} \\
    Substitute $x=\pi$ and $y=0$:
    \[ g(\pi, 0) = \pi \sin(0) + 0 \sin(\pi) = \pi(0) + 0(0) = 0 \]
    \textbf{Answer:} $0$

    \item \textbf{$g(\pi/2, \pi/4)$} \\
    Substitute $x=\pi/2$ and $y=\pi/4$:
    \[ g(\pi/2, \pi/4) = \frac{\pi}{2} \sin(\frac{\pi}{4}) + \frac{\pi}{4} \sin(\frac{\pi}{2}) = \frac{\pi}{2} \left(\frac{\sqrt{2}}{2}\right) + \frac{\pi}{4} (1) = \frac{\pi\sqrt{2}}{4} + \frac{\pi}{4} = \frac{\pi(\sqrt{2}+1)}{4} \]
    \textbf{Answer:} $\frac{\pi(\sqrt{2}+1)}{4}$

    \item \textbf{$g(0, y)$} \\
    Substitute $x=0$:
    \[ g(0, y) = 0 \sin(y) + y \sin(0) = 0 + y(0) = 0 \]
    \textbf{Answer:} $0$

    \item \textbf{$g(x, y+h)$} \\
    The problem states $f(x, y+h)$, but based on context, it should be $g(x, y+h)$. Substitute $y+h$ for $y$:
    \[ g(x, y+h) = x \sin(y+h) + (y+h) \sin(x) \]
    \textbf{Answer:} $x \sin(y+h) + (y+h) \sin(x)$
\end{enumerate}

\subsection{Problem 3}
Let $g(x, y) = x^2 \ln(x+y)$.
\begin{enumerate}[label=\alph*.]
    \item \textbf{Evaluate $g(3, 1)$.} \\
    \[ g(3, 1) = (3)^2 \ln(3+1) = 9 \ln(4) = 9 \ln(2^2) = 18 \ln(2) \]
    \textbf{Answer:} $9 \ln(4)$ or $18 \ln(2)$.

    \item \textbf{Find and sketch the domain of $g$.} \\
    The natural logarithm function $\ln(u)$ is defined only when $u > 0$. Therefore, we must have:
    \[ x + y > 0 \implies y > -x \]
    The domain is the set of all points $(x, y)$ such that $y > -x$. This is the open half-plane above the line $y = -x$. The line itself is not included. \\
    \textbf{Sketch:} Draw the line $y=-x$ with a dashed line. Shade the entire region above this line.

    \item \textbf{Find the range of $g$.} \\
    The term $x^2$ is always non-negative ($x^2 \ge 0$). The term $\ln(x+y)$ can take any real value. For example, if we let $x=1$, then we have $\ln(1+y)$, which can be any real number as $1+y$ ranges over all positive numbers. The product of a non-negative number and a number that can be anything (positive, negative, or zero) can be any real number. For example, let $x+y=c$. Then $g(x,y)=x^2\ln c$. As $c \to \infty$, $\ln c \to \infty$. As $c \to 0^+$, $\ln c \to -\infty$. Therefore, the range is all real numbers. \\
    \textbf{Answer:} $(-\infty, \infty)$ or $\mathbb{R}$.
\end{enumerate}

\subsection{Problem 4}
Let $h(x, y) = e^{\sqrt{y - x^2}}$.
\begin{enumerate}[label=\alph*.]
    \item \textbf{Evaluate $h(-2, 5)$.} \\
    \[ h(-2, 5) = e^{\sqrt{5 - (-2)^2}} = e^{\sqrt{5-4}} = e^{\sqrt{1}} = e^1 = e \]
    \textbf{Answer:} $e$

    \item \textbf{Find and sketch the domain of $h$.} \\
    The square root function $\sqrt{u}$ is defined only when $u \ge 0$. Therefore, we must have:
    \[ y - x^2 \ge 0 \implies y \ge x^2 \]
    The domain is the set of all points $(x, y)$ such that $y \ge x^2$. This is the region on and above the parabola $y = x^2$. \\
    \textbf{Sketch:} Draw the parabola $y=x^2$ with a solid line. Shade the region inside/above the parabola.

    \item \textbf{Find the range of $h$.} \\
    The expression $\sqrt{y - x^2}$ can take any value greater than or equal to 0. Let $u = \sqrt{y - x^2}$, so $u \ge 0$. The function is $h(x, y) = e^u$. Since $u \ge 0$, the range of $e^u$ is $[e^0, \infty)$, which is $[1, \infty)$. \\
    \textbf{Answer:} $[1, \infty)$
\end{enumerate}

\subsection{Problem 5}
Let $F(x, y, z) = \sqrt{y - \sqrt{x - 2z}}$.
\begin{enumerate}[label=\alph*.]
    \item \textbf{Evaluate $F(3, 4, 1)$.} \\
    The prompt says $f(3,4,1)$, likely a typo for $F$.
    \[ F(3, 4, 1) = \sqrt{4 - \sqrt{3 - 2(1)}} = \sqrt{4 - \sqrt{1}} = \sqrt{4 - 1} = \sqrt{3} \]
    \textbf{Answer:} $\sqrt{3}$

    \item \textbf{Find and describe the domain of $F$.} \\
    We have two square roots, so two conditions must be met:
    \begin{enumerate}
        \item The inner square root requires $x - 2z \ge 0 \implies x \ge 2z$.
        \item The outer square root requires $y - \sqrt{x - 2z} \ge 0 \implies y \ge \sqrt{x - 2z}$. Since the square root is non-negative, this also implies $y \ge 0$.
    \end{enumerate}
    The domain is the set of all points $(x, y, z)$ in $\mathbb{R}^3$ such that $x \ge 2z$ and $y \ge \sqrt{x-2z}$.
\end{enumerate}

\subsection{Problem 6}
Let $f(x, y, z) = \ln(z - \sqrt{x^2 + y^2})$.
\begin{enumerate}[label=\alph*.]
    \item \textbf{Evaluate $f(4, -3, 6)$.} \\
    \[ f(4, -3, 6) = \ln(6 - \sqrt{4^2 + (-3)^2}) = \ln(6 - \sqrt{16 + 9}) = \ln(6 - \sqrt{25}) = \ln(6 - 5) = \ln(1) = 0 \]
    \textbf{Answer:} $0$
    
    \item \textbf{Find and describe the domain of $f$.} \\
    The square root $\sqrt{x^2+y^2}$ is always defined since $x^2+y^2 \ge 0$. The logarithm requires its argument to be positive:
    \[ z - \sqrt{x^2 + y^2} > 0 \implies z > \sqrt{x^2 + y^2} \]
    This inequality describes the set of points $(x, y, z)$ that lie \textit{above} the cone $z = \sqrt{x^2 + y^2}$. This cone has its vertex at the origin and opens upward. The domain does not include the cone itself.
\end{enumerate}

\subsection{Problems 7-16: Find and sketch the domain of the function.}

\subsubsection{Problem 7: $f(x, y) = \sqrt{x-2} + \sqrt{y-1}$}
\textbf{Domain:} We need $x-2 \ge 0 \implies x \ge 2$ and $y-1 \ge 0 \implies y \ge 1$. The domain is the set of points $(x, y)$ such that $x \ge 2$ and $y \ge 1$. \\
\textbf{Sketch:} This is the first quadrant region to the right of the vertical line $x=2$ and above the horizontal line $y=1$, including the lines themselves.

\subsubsection{Problem 8: $f(x, y) = \sqrt[4]{x-3y}$}
\textbf{Domain:} The fourth root is an even root, so the radicand must be non-negative.
\[ x - 3y \ge 0 \implies x \ge 3y \implies y \le \frac{1}{3}x \]
The domain is the set of points $(x, y)$ on or below the line $y = \frac{1}{3}x$. \\
\textbf{Sketch:} Draw the line $y=\frac{1}{3}x$ as a solid line. Shade the region below it.

\subsubsection{Problem 9: $g(x, y) = \sqrt{x} + \sqrt{4 - 4x^2 - y^2}$}
\textbf{Domain:} We need $x \ge 0$ and $4 - 4x^2 - y^2 \ge 0$.
The second inequality is $4 \ge 4x^2 + y^2$, which can be written as:
\[ \frac{4x^2}{4} + \frac{y^2}{4} \le 1 \implies x^2 + \frac{y^2}{4} \le 1 \]
This is the region on and inside an ellipse centered at the origin with semi-axes of length $a=1$ (in the x-direction) and $b=2$ (in the y-direction).
We must satisfy both conditions: $x \ge 0$ and being inside the ellipse. This is the right half of the interior of the ellipse. \\
\textbf{Sketch:} Draw the ellipse $x^2 + y^2/4 = 1$. The domain is the region inside this ellipse that lies in the first and fourth quadrants (where $x \ge 0$), including the boundary.

\subsubsection{Problem 10: $g(x, y) = \ln(x^2 + y^2 - 9)$}
\textbf{Domain:} The argument of the logarithm must be positive.
\[ x^2 + y^2 - 9 > 0 \implies x^2 + y^2 > 9 \]
This is the set of all points $(x, y)$ outside the circle of radius 3 centered at the origin. The circle itself is not included. \\
\textbf{Sketch:} Draw a circle $x^2+y^2=9$ with a dashed line. Shade the region outside the circle.

\subsubsection{Problem 11: $g(x, y) = \frac{x-y}{x+y}$}
\textbf{Domain:} The denominator cannot be zero.
\[ x + y \neq 0 \implies y \neq -x \]
The domain is all points in the $xy$-plane except for those on the line $y = -x$. \\
\textbf{Sketch:} Draw the line $y = -x$ with a dashed line. The domain is the entire plane except for this line.

\subsubsection{Problem 12: $g(x, y) = \frac{\ln(2-x)}{1 - x^2 - y^2}$}
\textbf{Domain:} Two conditions must be met:
\begin{enumerate}
    \item Argument of ln must be positive: $2-x > 0 \implies x < 2$.
    \item Denominator cannot be zero: $1 - x^2 - y^2 \neq 0 \implies x^2 + y^2 \neq 1$.
\end{enumerate}
The domain is the set of all points $(x, y)$ such that $x < 2$ and $(x, y)$ is not on the unit circle. \\
\textbf{Sketch:} Draw the vertical line $x=2$ with a dashed line and shade the region to its left. Then, draw the unit circle $x^2+y^2=1$ with a dashed line. The domain is the shaded region to the left of $x=2$, with the points on the unit circle removed.

\subsubsection{Problem 13: $p(x, y) = \frac{\sqrt{xy}}{x+1}$}
\textbf{Domain:} Two conditions must be met:
\begin{enumerate}
    \item Radicand must be non-negative: $xy \ge 0$. This occurs when $x$ and $y$ have the same sign (or one is zero), which corresponds to the first and third quadrants, including the axes.
    \item Denominator cannot be zero: $x+1 \neq 0 \implies x \neq -1$.
\end{enumerate}
The domain is the set of all points in the first and third quadrants (including axes) except for the point $(-1, 0)$ which is on the x-axis in the restricted zone. The line $x=-1$ must be excluded. \\
\textbf{Sketch:} Shade the first and third quadrants. Draw the vertical line $x=-1$ as a dashed line to indicate its exclusion.

\subsubsection{Problem 14: $f(x, y) = \arcsin(x+y)$}
\textbf{Domain:} The domain of $\arcsin(u)$ is $-1 \le u \le 1$.
\[ -1 \le x+y \le 1 \]
This can be split into two inequalities: $x+y \le 1 \implies y \le -x+1$ and $x+y \ge -1 \implies y \ge -x-1$.
The domain is the region between the lines $y = -x+1$ and $y = -x-1$, including the lines themselves. \\
\textbf{Sketch:} Draw the two parallel lines $y=-x+1$ and $y=-x-1$ as solid lines. Shade the strip between them.

\subsubsection{Problem 15: $f(x, y, z) = \sqrt{4-x^2} + \sqrt{9-y^2} + \sqrt{1-z^2}$}
\textbf{Domain:} This is a function of three variables. All three radicands must be non-negative.
\begin{itemize}
    \item $4-x^2 \ge 0 \implies x^2 \le 4 \implies -2 \le x \le 2$
    \item $9-y^2 \ge 0 \implies y^2 \le 9 \implies -3 \le y \le 3$
    \item $1-z^2 \ge 0 \implies z^2 \le 1 \implies -1 \le z \le 1$
\end{itemize}
The domain is a closed rectangular box in $\mathbb{R}^3$ defined by these three inequalities.

\subsubsection{Problem 16: $f(x, y, z) = \ln(16 - 4x^2 - 4y^2 - z^2)$}
\textbf{Domain:} Argument of ln must be positive.
\[ 16 - 4x^2 - 4y^2 - z^2 > 0 \implies 4x^2 + 4y^2 + z^2 < 16 \]
\[ \frac{4x^2}{16} + \frac{4y^2}{16} + \frac{z^2}{16} < 1 \implies \frac{x^2}{4} + \frac{y^2}{4} + \frac{z^2}{16} < 1 \]
This is the set of all points $(x, y, z)$ strictly inside the ellipsoid centered at the origin with semi-axes of length $a=2$, $b=2$, and $c=4$.

\subsection{Problem 17}
A model for the surface area of a human body is $S = f(w, h) = 0.1091 w^{0.425} h^{0.725}$, where $w$ is weight (lbs) and $h$ is height (in).
\begin{enumerate}[label=\alph*.]
    \item \textbf{Find $f(160, 70)$ and interpret it.} \\
    \[ f(160, 70) = 0.1091 (160)^{0.425} (70)^{0.725} \]
    Using a calculator:
    $160^{0.425} \approx 8.344$
    $70^{0.725} \approx 21.46$
    \[ f(160, 70) \approx 0.1091 \times 8.344 \times 21.46 \approx 19.56 \]
    \textbf{Interpretation:} A person who weighs 160 pounds and is 70 inches tall has a body surface area of approximately 19.56 square feet.
    \textbf{Answer:} $\approx 19.56$ ft$^2$.
    \item \textbf{What is your own surface area?} \\
    This requires personal data. For example, a person with $w=150$ lbs and $h=68$ in:
    \[ S = 0.1091 (150)^{0.425} (68)^{0.725} \approx 0.1091(8.13)(20.95) \approx 18.59 \text{ ft}^2 \]
\end{enumerate}

\subsection{Problem 18}
A manufacturer's production is modeled by the Cobb-Douglas function $P(L, K) = 1.47 L^{0.65} K^{0.35}$, where $L$ is labor hours (thousands) and $K$ is capital (millions of dollars). Find $P(120, 20)$ and interpret it.
\[ P(120, 20) = 1.47 (120)^{0.65} (20)^{0.35} \]
Using a calculator:
$120^{0.65} \approx 20.15$
$20^{0.35} \approx 3.31$
\[ P(120, 20) \approx 1.47 \times 20.15 \times 3.31 \approx 98.1 \]
\textbf{Interpretation:} When 120,000 hours of labor are used and \$20 million of capital is invested, the monetary value of the production is approximately \$98.1 million.
\textbf{Answer:} $\approx 98.1$.

\subsection{Problems 19-21: Interpreting tables}
These problems refer to tables that are not fully provided in the prompt text, but the questions can be answered from the included excerpts and general principles.

\subsection{Problem 20 (Based on Table 3)}
$I = f(T, h)$ is the humidex, where $T$ is actual temperature and $h$ is relative humidity.
\begin{enumerate}[label=\alph*.]
    \item \textbf{What is the value of $f(95, 70)$? What is its meaning?} \\
    From the table, find the row for $T=95$ and the column for $h=70$. The value is 124.
    \textbf{Meaning:} When the actual temperature is 95$^\circ$F and the relative humidity is 70\%, the perceived air temperature (humidex) is 124$^\circ$F.
    \textbf{Answer:} $f(95, 70) = 124$.
    \item \textbf{For what value of $h$ is $f(90, h) = 100$?} \\
    Look at the row for $T=90$. Find the column where the value is 100. This occurs at $h=60$.
    \textbf{Answer:} $h = 60$.
    \item \textbf{For what value of $T$ is $f(T, 50) = 88$?} \\
    Look at the column for $h=50$. Find the row where the value is 88. This occurs at $T=85$.
    \textbf{Answer:} $T = 85$.
    \item \textbf{What are the meanings of the functions $I = f(80, h)$ and $I = f(100, h)$? Compare their behavior.} \\
    \textbf{Meaning:} $I = f(80, h)$ represents the humidex as a function of humidity when the actual temperature is held constant at 80$^\circ$F. $I = f(100, h)$ is the humidex as a function of humidity at a constant temperature of 100$^\circ$F.
    \textbf{Behavior:} Looking at the rows for T=80 and T=100, we see that in both cases, the humidex increases as humidity $h$ increases. However, the increase is much more dramatic at the higher temperature. For $T=80$, the humidex goes from 77 to 83 (a 6-point increase). For $T=100$, it goes from 99 to 144 (a 45-point increase).
\end{enumerate}

\subsection{Problem 21 (Based on Table 4)}
$h = f(v, t)$ is the wave height, where $v$ is wind speed (knots) and $t$ is duration (hours).
\begin{enumerate}[label=\alph*.]
    \item \textbf{What is the value of $f(40, 15)$? What is its meaning?} \\
    Find the row for $v=40$ and the column for $t=15$. The value is 25.
    \textbf{Meaning:} When the wind blows at 40 knots for 15 hours, the wave heights are 25 feet.
    \textbf{Answer:} $f(40, 15) = 25$.
    \item \textbf{What is the meaning of the function $h = f(30, t)$?} \\
    This function represents the wave height as a function of the duration $t$ that the wind has been blowing, assuming the wind speed is held constant at 30 knots.
    \textbf{Behavior:} Looking at the row for $v=30$, the values are 9, 13, 16, 17, 18, 19, 19. The wave height increases as the duration increases, but the rate of increase slows down over time.
    \item \textbf{What is the meaning of the function $h = f(v, 30)$?} \\
    This function represents the wave height as a function of wind speed $v$, assuming the wind has been blowing for a constant duration of 30 hours.
    \textbf{Behavior:} Looking at the column for $t=30$, the values are 2, 5, 9, 18, 31, 45, 62. The wave height increases as the wind speed increases, and the rate of increase appears to be accelerating.
\end{enumerate}

\subsection{Problem 22}
A company makes small ($x$), medium ($y$), and large ($z$) boxes. Costs: small \$2.50, medium \$4.00, large \$4.50. Fixed costs: \$8000.
\begin{enumerate}[label=\alph*.]
    \item \textbf{Express the cost $C$ as a function of three variables.} \\
    The total cost is the sum of the variable costs for producing all boxes plus the fixed costs.
    \[ C(x, y, z) = 2.50x + 4.00y + 4.50z + 8000 \]
    \textbf{Answer:} $C(x, y, z) = 2.5x + 4y + 4.5z + 8000$.
    \item \textbf{Find $f(3000, 5000, 4000)$ and interpret it.} \\
    \[ C(3000, 5000, 4000) = 2.5(3000) + 4(5000) + 4.5(4000) + 8000 \]
    \[ = 7500 + 20000 + 18000 + 8000 = 53500 \]
    \textbf{Interpretation:} The total cost to produce 3000 small boxes, 5000 medium boxes, and 4000 large boxes is \$53,500.
    \textbf{Answer:} \$53,500.
    \item \textbf{What is the domain of $f$?} \\
    The variables $x, y, z$ represent the number of boxes, so they cannot be negative. They must also be integers.
    The domain is the set of all triples $(x, y, z)$ such that $x, y, z$ are non-negative integers.
\end{enumerate}

\subsection{Problems 23-31: Sketch the graph of the function}

\subsubsection{Problem 23: $f(x, y) = y$}
The equation is $z=y$. This graph is a plane. It contains the x-axis (since if $y=0$, $z=0$ for any $x$) and the line $z=y$ in the yz-plane. It is a tilted plane that is horizontal in the x-direction.

\subsubsection{Problem 24: $f(x, y) = x^2$}
The equation is $z = x^2$. Since the variable $y$ is missing, this is a cylinder. The base curve is the parabola $z=x^2$ in the xz-plane. The graph is this parabola extended infinitely in the positive and negative y-directions, forming a "parabolic trough".

\subsubsection{Problem 25: $f(x, y) = 10 - 4x - 5y$}
The equation is $z = 10 - 4x - 5y$, or $4x + 5y + z = 10$. This is the equation of a plane. To sketch it, we can find the intercepts:
\begin{itemize}
    \item x-intercept (y=0, z=0): $4x = 10 \implies x = 2.5$
    \item y-intercept (x=0, z=0): $5y = 10 \implies y = 2$
    \item z-intercept (x=0, y=0): $z = 10$
\end{itemize}
The graph is the plane passing through the points $(2.5, 0, 0)$, $(0, 2, 0)$, and $(0, 0, 10)$.

\subsubsection{Problem 26: $f(x, y) = \cos y$}
The equation is $z = \cos y$. Since $x$ is missing, this is a cylinder. The base curve is the cosine wave $z = \cos y$ in the yz-plane. The graph is this wave extended infinitely along the x-axis, creating a corrugated sheet.

\subsubsection{Problem 27: $f(x, y) = \sin x$}
The equation is $z = \sin x$. Similar to the previous problem, this is a sinusoidal cylinder parallel to the y-axis.

\subsubsection{Problem 28: $f(x, y) = 2 - x^2 - y^2$}
The equation is $z = 2 - (x^2 + y^2)$. This is a circular paraboloid that opens downward. Its vertex is at $(0, 0, 2)$.

\subsubsection{Problem 29: $f(x, y) = x^2 + 4y^2 + 1$}
The equation is $z = x^2 + 4y^2 + 1$. This is an elliptic paraboloid that opens upward. Its vertex is at $(0, 0, 1)$. The traces in horizontal planes ($z=k$) are ellipses.

\subsubsection{Problem 30: $f(x, y) = \sqrt{4x^2 + y^2}$}
The equation is $z = \sqrt{4x^2 + y^2}$. Squaring both sides gives $z^2 = 4x^2 + y^2$ (with the condition $z \ge 0$). This is the top half of an elliptic cone. The vertex is at the origin.

\subsubsection{Problem 31: $f(x, y) = \sqrt{4 - 4x^2 - y^2}$}
The equation is $z = \sqrt{4 - 4x^2 - y^2}$. Squaring both sides gives $z^2 = 4 - 4x^2 - y^2$, so $4x^2 + y^2 + z^2 = 4$, or $x^2 + \frac{y^2}{4} + \frac{z^2}{4} = 1$. This is an ellipsoid. Since the original function has $z \ge 0$, the graph is the top half of this ellipsoid.

\subsection{Problem 32: Match the function with its graph}
\begin{enumerate}[label=\alph*.]
    \item $f(x, y) = \frac{1}{1 + x^2 + y^2}$: The function is always positive. At $(0,0)$, $f=1$. As $x$ or $y$ grow large, $f \to 0$. The function has radial symmetry. This corresponds to a single peak at the origin, tapering off in all directions. \textbf{Matches Graph III.}
    
    \item $f(x, y) = \frac{1}{1 + x^2y^2}$: At $(0,0)$, $f=1$. Along the axes ($x=0$ or $y=0$), $f(x,0)=1$ and $f(0,y)=1$. This means there are ridges of height 1 along the axes. As $|x|$ and $|y|$ grow, $f \to 0$. This doesn't match the remaining graphs perfectly but might be misidentified. Let's re-evaluate.
    Looking at the graphs, graph I has ridges. Let's check other functions.
    
    \item $f(x, y) = \ln(x^2 + y^2)$: Domain is $(x,y) \neq (0,0)$. As $(x,y) \to (0,0)$, $f \to -\infty$. As $|x|$ or $|y|$ grow, $f \to \infty$. This describes a shape that goes to $-\infty$ at the origin and rises outwards. \textbf{Matches Graph IV.}
    
    \item $f(x, y) = \cos(\sqrt{x^2+y^2})$: This function has radial symmetry. Along the positive x-axis ($y=0$), we have $f(x,0)=\cos(x)$. It should look like a cosine wave rotated around the z-axis, creating circular ripples. \textbf{Matches Graph V.}
    
    \item $f(x, y) = |xy|$: At the origin, $f=0$. Along the axes, $f=0$. In the quadrants, $f>0$. The function grows as we move away from the axes. This creates four valleys along the axes rising to four peaks in the quadrants. \textbf{Matches Graph VI.}
    
    \item $f(x, y) = \cos(xy)$: At the origin, $f=1$. Along the axes ($x=0$ or $y=0$), $f = \cos(0) = 1$. The function has ridges of height 1 along both axes. It oscillates between -1 and 1 as we move away from the origin. This pattern of ridges along axes and oscillations matches the description better than (b). \textbf{Matches Graph I.}
    
    Revisiting (b): $f(x, y) = \frac{1}{1 + x^2y^2}$. It has ridges of height 1 on the axes. Graph II has four peaks that are not along the axes, and dips towards the axes. This doesn't seem to match any of the functions perfectly. There might be a typo in the question or options. However, by elimination, there is no other function for Graph II. Let's check if $f(x,y)=e^{-(x^2+y^2)}$ or something similar would produce Graph II (a gaussian bump, but it has 4 symmetric bumps). Let's assume graph II is from a function not listed or there is a mismatch. The other pairings are strong.
\end{enumerate}

\subsection{Problems 45-55: Draw a contour map of the function}
A contour map is created by sketching several level curves $f(x, y) = k$.

\subsubsection{Problem 45: $f(x, y) = x^2 - y^2$}
Level curves: $x^2 - y^2 = k$.
\begin{itemize}
    \item If $k=0$, $x^2 = y^2 \implies y = \pm x$. Two lines through the origin.
    \item If $k > 0$, $x^2 - y^2 = k$ is a hyperbola opening along the x-axis.
    \item If $k < 0$, $x^2 - y^2 = k \implies y^2 - x^2 = -k$ is a hyperbola opening along the y-axis.
\end{itemize}
\textbf{Sketch:} Draw the lines $y=x$ and $y=-x$. Draw hyperbolas opening left/right for $k=1, 2$ and hyperbolas opening up/down for $k=-1, -2$.

\subsubsection{Problem 46: $f(x, y) = xy$}
Level curves: $xy=k$.
\begin{itemize}
    \item If $k=0$, $x=0$ or $y=0$. These are the coordinate axes.
    \item If $k \neq 0$, $y = k/x$. These are hyperbolas with the axes as asymptotes. For $k>0$, they are in quadrants I and III. For $k<0$, they are in quadrants II and IV.
\end{itemize}
\textbf{Sketch:} Draw hyperbolas in each quadrant, symmetric with respect to the origin.

\subsubsection{Problem 47: $f(x, y) = \sqrt{x} + y$}
Level curves: $\sqrt{x} + y = k \implies y = k - \sqrt{x}$.
Domain requires $x \ge 0$. These are square root curves, reflected vertically and shifted up or down by $k$.
\textbf{Sketch:} For $k=0, 1, 2, -1$, draw the curves $y=-\sqrt{x}$, $y=1-\sqrt{x}$, $y=2-\sqrt{x}$, $y=-1-\sqrt{x}$.

\subsubsection{Problem 53: $f(x, y) = x^2 + 9y^2$}
Level curves: $x^2 + 9y^2 = k$.
\begin{itemize}
    \item If $k<0$, no solution.
    \item If $k=0$, only the point $(0,0)$.
    \item If $k>0$, $\frac{x^2}{k} + \frac{9y^2}{k} = 1 \implies \frac{x^2}{k} + \frac{y^2}{k/9} = 1$. These are ellipses centered at the origin, with semi-axes $\sqrt{k}$ and $\sqrt{k}/3$. They are wider than they are tall.
\end{itemize}
\textbf{Sketch:} Draw a series of concentric ellipses that are stretched horizontally.

\subsubsection{Problem 54: $f(x, y) = \sqrt{36 - 9x^2 - 4y^2}$}
Level curves: $\sqrt{36 - 9x^2 - 4y^2} = k$.
Domain requires $36 - 9x^2 - 4y^2 \ge 0$. Range requires $k \ge 0$.
Square both sides: $36 - 9x^2 - 4y^2 = k^2 \implies 9x^2 + 4y^2 = 36 - k^2$.
Let $C = 36 - k^2$. The level curves are ellipses $9x^2 + 4y^2 = C$. Since $k \ge 0$, the maximum value for $f$ is at $(0,0)$ where $f(0,0)=\sqrt{36}=6$. So $0 \le k \le 6$.
\begin{itemize}
    \item $k=0 \implies 9x^2+4y^2=36 \implies \frac{x^2}{4} + \frac{y^2}{9}=1$. Outer boundary ellipse.
    \item $k=6 \implies 9x^2+4y^2=0 \implies (0,0)$. The center.
    \item For $0<k<6$, we get ellipses between these two extremes.
\end{itemize}
\textbf{Sketch:} Draw a series of concentric ellipses, taller than they are wide, filling the region inside $\frac{x^2}{4} + \frac{y^2}{9}=1$.

\subsubsection{Problem 55: $T(x, y) = \frac{100}{1 + x^2 + 2y^2}$}
Level curves (isothermals): $\frac{100}{1 + x^2 + 2y^2} = k$.
\[ 1 + x^2 + 2y^2 = \frac{100}{k} \implies x^2 + 2y^2 = \frac{100}{k} - 1 \]
Let $C = \frac{100}{k}-1$. The curves are ellipses. The maximum temperature is $T(0,0)=100$. As $(x,y) \to \infty$, $T \to 0$. So $0 < k \le 100$.
\textbf{Sketch:} For various $k$ values (e.g., 50, 25, 10), we get ellipses centered at the origin. $x^2+2y^2=C$ means the ellipses are wider than they are tall.

\subsection{Problems 61-66: Match the function with its graph and contour map}
This is a large matching exercise. Key strategy is to analyze symmetry, traces, and behavior at the origin.

\begin{itemize}
    \item \textbf{61. $z = \sin(xy)$}:
    \begin{itemize}
        \item Graph: Along axes ($x=0$ or $y=0$), $z=\sin(0)=0$. The graph is flat along the axes. As $xy$ increases, it oscillates. This saddle-like oscillation matches \textbf{Graph D}.
        \item Contour Map: Level curves are $xy=k$. These are hyperbolas. Matches \textbf{Contour Map II}.
    \end{itemize}
    
    \item \textbf{62. $z = e^x \cos y$}:
    \begin{itemize}
        \item Graph: Along the y-axis ($x=0$), $z=\cos y$ (a cosine wave). As $x$ increases, the amplitude of the cosine waves $e^x \cos y$ grows exponentially. As $x \to -\infty$, the amplitude shrinks to 0. This behavior matches \textbf{Graph B}.
        \item Contour Map: Level curves $e^x \cos y = k$. $e^x = k/\cos y$. These curves are periodic in $y$. Looks like repeating patterns that stretch out along the x-axis. Matches \textbf{Contour Map IV}.
    \end{itemize}
    
    \item \textbf{63. $z = \sin(x-y)$}:
    \begin{itemize}
        \item Graph: The value of $z$ is constant along any line where $x-y=c$. For example, along the line $y=x$, we have $x-y=0$, so $z=\sin(0)=0$. The function is a set of parallel waves/ridges oriented along the line $y=x$. This matches \textbf{Graph F}.
        \item Contour Map: Level curves are $x-y=k$, which are parallel lines with slope 1. This matches \textbf{Contour Map I}.
    \end{itemize}
    
    \item \textbf{64. $z = \sin x - \sin y$}:
    \begin{itemize}
        \item Graph: This function shows separate periodic behavior in $x$ and $y$. It creates a grid-like pattern of hills and valleys. This matches \textbf{Graph E}.
        \item Contour Map: The level curves are more complex, showing repeating islands. $\sin x - \sin y = 0$ gives $x=y+2n\pi$ or $x=\pi-y+2n\pi$. The zero level curves are diagonal lines. The map will have a repeating grid of closed curves. Matches \textbf{Contour Map III}.
    \end{itemize}
    
    \item \textbf{65. $z = (1-x^2)(1-y^2)$}:
    \begin{itemize}
        \item Graph: At $(0,0)$, $z=1$. Along the lines $x=\pm 1$ or $y=\pm 1$, $z=0$. This creates a central peak at the origin, surrounded by four "moats" and then the function rises again. This matches \textbf{Graph C}.
        \item Contour Map: The level curves will be somewhat square-shaped near the origin. The zero level curves are the lines $x=\pm 1$ and $y=\pm 1$. Matches \textbf{Contour Map VI}.
    \end{itemize}
    
    \item \textbf{66. $z = \frac{x-y}{1+x^2+y^2}$}:
    \begin{itemize}
        \item Graph: At $(0,0)$, $z=0$. The function is positive when $x>y$ and negative when $x<y$. It goes to 0 as $|x|,|y| \to \infty$. This creates one peak and one valley, separated by the line $y=x$. This matches \textbf{Graph A}.
        \item Contour Map: The level curves are complex, but will be centered around a maximum point in the region $x>y$ and a minimum point in the region $x<y$. The line $y=x$ is the zero contour. This matches \textbf{Contour Map V}.
    \end{itemize}
\end{itemize}

\subsection{Problems 67-70: Describe the level surfaces of the function}

\subsubsection{Problem 67: $f(x, y, z) = 2y - z + 1$}
Level surfaces: $2y - z + 1 = k$.
\[ 2y - z = k - 1 \]
This is a family of planes. All the planes are parallel to each other and parallel to the x-axis (since $x$ is not in the equation).

\subsubsection{Problem 68: $f(x, y, z) = x + y^2 - z^2$}
Level surfaces: $x + y^2 - z^2 = k$.
\[ x - k = z^2 - y^2 \]
This is a family of hyperbolic paraboloids (saddle surfaces). The axis of the saddle is the x-axis, and the shape is shifted along the x-axis by the value of $k$.

\subsubsection{Problem 69: $g(x, y, z) = x^2 + y^2 - z^2$}
Level surfaces: $x^2 + y^2 - z^2 = k$.
\begin{itemize}
    \item $k > 0$: $x^2+y^2-z^2=k$ is a hyperboloid of one sheet, with the z-axis as its axis of symmetry.
    \item $k = 0$: $x^2+y^2=z^2$ is a double cone with the z-axis as its axis.
    \item $k < 0$: $x^2+y^2-z^2=k \implies z^2-x^2-y^2=-k$ is a hyperboloid of two sheets, with the z-axis as its axis of symmetry.
\end{itemize}

\subsubsection{Problem 70: $f(x, y, z) = x^2 + 2y^2 + 3z^2$}
Level surfaces: $x^2 + 2y^2 + 3z^2 = k$.
\begin{itemize}
    \item $k < 0$: No surface.
    \item $k = 0$: The single point $(0,0,0)$.
    \item $k > 0$: The surfaces are ellipsoids centered at the origin.
\end{itemize}

\section{In-Depth Analysis of Problems and Techniques}

\subsection{Problem Types and General Approach}
\begin{itemize}
    \item \textbf{Type 1: Direct Evaluation (Probs 1, 2, 3a, 4a, 5a, 6a, 17, 18, 22b):}
    \begin{itemize}
        \item \textbf{Approach:} The most straightforward type. Substitute the given numerical or variable inputs into the function's definition and perform the required arithmetic or algebraic simplification.
    \end{itemize}
    
    \item \textbf{Type 2: Domain Determination (Probs 3b, 4b, 5b, 6b, 7-16, 22c):}
    \begin{itemize}
        \item \textbf{Approach:} Identify parts of the function that have mathematical restrictions (square roots, logarithms, denominators). Write down the corresponding inequality or non-equality for each restriction. Solve the system of inequalities. For 2D domains, sketch the boundary curves (solid for $\ge, \le$; dashed for $>, <, \neq$) and shade the valid region.
    \end{itemize}
    
    \item \textbf{Type 3: Sketching 3D Graphs (Probs 23-31):}
    \begin{itemize}
        \item \textbf{Approach:} First, try to recognize the function as a standard surface (plane, paraboloid, cylinder, etc.). If a variable is missing (e.g., $z=f(x)$), it's a cylinder extruded along the axis of the missing variable. If not immediately recognizable, use traces: set $x=0$, $y=0$, and $z=k$ to find the cross-sections of the surface, which can then be assembled into a sketch.
    \end{itemize}
    
    \item \textbf{Type 4: Contour Map Analysis (Probs 20, 21, 45-55, 61-66):}
    \begin{itemize}
        \item \textbf{Approach:} To create a contour map, set $f(x, y) = k$ for several different constants $k$. For each $k$, identify the resulting 2D curve ($y$ as a function of $x$, or a conic section) and sketch it on the $xy$-plane, labeling it with its $k$-value. To interpret a map, treat it like a topographic map: closely spaced lines mean a steep slope on the 3D surface, and widely spaced lines mean a gentle slope.
    \end{itemize}
    
    \item \textbf{Type 5: Matching and Identification (Probs 32, 61-66):}
    \begin{itemize}
        \item \textbf{Approach:} This is a process of elimination and feature detection.
        \begin{enumerate}
            \item \textbf{Symmetry:} Check for radial symmetry (if $f$ depends only on $x^2+y^2$) or symmetry across axes.
            \item \textbf{Traces:} Check the function's behavior along the axes (setting $x=0$ or $y=0$). Does it form a sine wave? A parabola? Is it constant? Match this to the graphs.
            \item \textbf{Key points/lines:} Where is the function zero? Where is it maximum? This is often revealing (e.g., in Prob 61, $z=0$ on the axes).
        \end{enumerate}
    \end{itemize}
     \item \textbf{Type 6: Level Surface Description (Probs 67-70):}
    \begin{itemize}
        \item \textbf{Approach:} Set $f(x, y, z) = k$. Rearrange the resulting equation into a standard form for a 3D surface (plane, sphere, ellipsoid, paraboloid, hyperboloid, cone). Describe how the parameters of the surface (e.g., radius, orientation, location) change as $k$ changes.
    \end{itemize}
\end{itemize}

\subsection{Key Algebraic and Calculus Manipulations}
\begin{itemize}
    \item \textbf{Recognizing Conic Sections:} This was the single most important algebraic skill for domain and level curve problems.
    \begin{itemize}
        \item In Problem 9, identifying $4x^2 + y^2 \le 4$ as the interior of an ellipse was the key to finding the domain.
        \item In Problem 10, recognizing $x^2 + y^2 > 9$ as the exterior of a circle.
        \item In Problem 45, knowing that $x^2 - y^2 = k$ defines a family of hyperbolas was essential for the contour map.
    \end{itemize}

    \item \textbf{Solving Inequalities:} The core of domain problems. This required isolating one variable or rearranging the inequality to match a known geometric shape.
    \begin{itemize}
        \item In Problem 14, the domain of arcsin led to $-1 \le x+y \le 1$, which had to be broken into two linear inequalities, defining a strip between two parallel lines.
    \end{itemize}

    \item \textbf{Analysis of Traces:} A fundamental technique for understanding and matching graphs. It's a "calculus" manipulation in the sense that we hold one variable constant and analyze the resulting single-variable function.
    \begin{itemize}
        \item In Problem 62, setting $x=0$ to get $z = \cos y$ immediately tells you the graph must look like a cosine wave along the y-axis, making it easy to identify Graph B.
    \end{itemize}
    
    \item \textbf{Identifying Cylinders:} Recognizing that a function of two variables like $f(x, y)$ that is missing one variable (e.g., $f(x, y) = x^2$) represents a cylinder was a crucial shortcut for sketching. (Problems 24, 26, 27).
\end{itemize}

\section{"Cheatsheet" and Tips for Success}

\subsection{Summary of Most Important Formulas/Concepts}
\begin{itemize}
    \item \textbf{Domain Rules:} The Big Three.
    \begin{itemize}
        \item Inside a square root must be $\ge 0$.
        \item Inside a logarithm must be $> 0$.
        \item A denominator must be $\neq 0$.
    \end{itemize}
    \item \textbf{Level Curve:} $f(x, y) = k$. A 2D curve in the plane. Spacing tells you steepness.
    \item \textbf{Level Surface:} $f(x, y, z) = k$. A 3D surface in space.
    \item \textbf{Graph Recognition:}
    \begin{itemize}
        \item Linear: $z = ax+by+c \implies$ Plane
        \item Quadratic Sum: $z = ax^2 + by^2 \implies$ Paraboloid
        \item Quadratic Difference: $z = ax^2 - by^2 \implies$ Hyperbolic Paraboloid (Saddle)
        \item Missing Variable: $z = f(x) \implies$ Cylinder along y-axis
        \item Root of Quadratic: $z = \sqrt{ax^2+by^2} \implies$ Cone (top half)
        \item Root of Constant - Quadratic: $z = \sqrt{k - ax^2 - by^2} \implies$ Ellipsoid (top half)
    \end{itemize}
\end{itemize}

\subsection{Cheats, Tricks, and Shortcuts}
\begin{itemize}
    \item \textbf{The Symmetry Trick:} If a function $f(x, y)$ depends only on the combination $\sqrt{x^2+y^2}$ or $x^2+y^2$, its graph is radially symmetric around the z-axis, and its level curves are circles. This is a huge time-saver for matching problems.
    \item \textbf{The Trace Test:} To quickly distinguish between graphs, check the trace along the x-axis ($y=0$). Is it a line? A parabola? A sine wave? This one check can often eliminate most of the options.
    \item \textbf{Domain Boundary Lines:} A strict inequality ($>, <$) in a domain means the boundary line is dashed. A non-strict inequality ($\ge, \le$) means it's a solid line.
    \item \textbf{Test a Point:} When sketching a domain defined by an inequality like $y > g(x)$, pick a simple test point not on the line (like $(0,0)$ if possible) and plug it into the inequality. If it's true, shade the region containing the point. If false, shade the other side.
\end{itemize}

\subsection{Common Pitfalls and Mistakes}
\begin{itemize}
    \item \textbf{Mixing up Domain and Range:} The domain is the set of valid inputs ($(x,y)$ pairs); the range is the set of possible outputs ($z$ values).
    \item \textbf{Forgetting Strict Inequalities:} The domain of $\ln(u)$ is $u > 0$, not $u \ge 0$. This means the boundary is not included.
    \item \textbf{Errors in 2D Graphing:} A weak foundation in graphing conic sections (ellipses, hyperbolas) will make sketching domains and level curves very difficult.
    \item \textbf{Interpreting Context:} In word problems, remember what the variables and the function output physically represent (e.g., cost, temperature, production value). An answer is incomplete without the interpretation and units.
\end{itemize}

\section{Common Variations and Untested Concepts}
This homework set provides a strong foundation. However, a typical calculus curriculum would also include these related topics that were not present.

\subsection{Limits and Continuity}
This is the immediate next step in the theory of multivariable functions. It explores the behavior of $f(x, y)$ as $(x, y)$ approaches a point $(a, b)$.
\begin{itemize}
    \item \textbf{Concept:} Unlike in single-variable calculus where you can only approach a point from the left or right, in 2D you can approach a point along infinitely many paths (lines, parabolas, etc.). For a limit to exist, the function must approach the same value along \textit{every} possible path.
    \item \textbf{Example Problem:} Show that the limit of $f(x, y) = \frac{2xy}{x^2+y^2}$ as $(x,y) \to (0,0)$ does not exist.
    \begin{itemize}
        \item \textbf{Solution:} We test two different paths to the origin.
        \begin{enumerate}
            \item Path 1: Along the x-axis (where $y=0$). The function becomes $f(x, 0) = \frac{0}{x^2} = 0$. The limit along this path is 0.
            \item Path 2: Along the line $y=x$. The function becomes $f(x, x) = \frac{2x(x)}{x^2+x^2} = \frac{2x^2}{2x^2} = 1$. The limit along this path is 1.
        \end{enumerate}
        Since we found two different limits along two different paths, the overall limit does not exist.
    \end{itemize}
\end{itemize}

\subsection{Piecewise-Defined Functions}
Functions can be defined by different formulas on different parts of their domain.
\begin{itemize}
    \item \textbf{Concept:} These are used to model objects or phenomena that have abrupt changes. Analyzing them often involves checking the behavior at the boundaries between the pieces.
    \item \textbf{Example Problem:} Sketch the graph of the function
    \[ f(x, y) = \begin{cases} \sqrt{1 - x^2 - y^2} & \text{if } x^2+y^2 \le 1 \\ 0 & \text{if } x^2+y^2 > 1 \end{cases} \]
    \begin{itemize}
        \item \textbf{Solution:} The condition $x^2+y^2 \le 1$ describes the unit disk centered at the origin. Inside this disk, the graph is $z = \sqrt{1-x^2-y^2}$, which is the top hemisphere of a sphere of radius 1. Outside this disk, the graph is $z=0$, which is the $xy$-plane. The final graph looks like a hemisphere sitting on an infinite flat plane.
    \end{itemize}
\end{itemize}

\subsection{More Complex Domains}
Domains can be defined by the intersection of several different types of curves.
\begin{itemize}
    \item \textbf{Concept:} The approach is the same—solve all inequalities—but the resulting sketch can be more intricate.
    \item \textbf{Example Problem:} Find and sketch the domain of $f(x, y) = \sqrt{y-x^2} + \ln(4-y)$.
    \begin{itemize}
        \item \textbf{Solution:} We need two conditions:
        \begin{enumerate}
            \item $y - x^2 \ge 0 \implies y \ge x^2$. This is the region on or above the parabola $y=x^2$.
            \item $4 - y > 0 \implies y < 4$. This is the region below the horizontal line $y=4$.
        \end{enumerate}
        The domain is the set of all points that are simultaneously above the parabola $y=x^2$ and below the line $y=4$. The parabola is included (solid line), but the horizontal line is not (dashed line).
    \end{itemize}
\end{itemize}

\end{document}