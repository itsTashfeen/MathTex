\documentclass{article}
\usepackage{graphicx} % Required for inserting images
\usepackage{amsmath}  % Required for advanced math environments like align*
\usepackage{amssymb}  % For more math symbols
\usepackage{amsfonts}

% --- GEOMETRY PACKAGE FOR PAGE LAYOUT ---
\usepackage[
    left=1in,
    textwidth=7in, 
    top=1in,
    bottom=1in
]{geometry}
% -----------------------------------------

\title{Integration Techniques Revision}
\author{Tashfeen Omran}
\date{September 2025}

\begin{document}

\maketitle

\section{Integration Problems and Solutions}

%---------------------------------------------------------------
\subsection{Problem 1}
Evaluate the integral: $ \int \ln(\cos(x)) \tan(x) \,dx $
\subsubsection*{Solution}
This integral is solved using a clever u-substitution.
Let $ u = \ln(\cos(x)) $.
Then $ du = \frac{1}{\cos(x)}(-\sin(x)) \,dx = -\tan(x) \,dx $.
Substituting these into the integral gives:
\[ \int u \,(-du) = -\int u \,du = -\frac{u^2}{2} + C \]
Substituting back for u:
\textbf{Answer:} $ -\frac{(\ln(\cos(x)))^2}{2} + C $


%---------------------------------------------------------------
\subsection{Problem 2}
Evaluate the integral: $ \int e^x \sin(x) \,dx $
\subsubsection*{Solution}
This requires "looping" integration by parts. Let $I = \int e^x \sin(x) \,dx$.
First IBP ($u=\sin(x), dv=e^x dx$): $ I = e^x\sin(x) - \int e^x\cos(x) \,dx $.
Second IBP ($u=\cos(x), dv=e^x dx$): $ \int e^x\cos(x) \,dx = e^x\cos(x) + \int e^x\sin(x) \,dx = e^x\cos(x) + I $.
Substitute back: $ I = e^x\sin(x) - (e^x\cos(x) + I) \implies 2I = e^x(\sin(x) - \cos(x)) $.
\textbf{Answer:} $ \frac{e^x(\sin(x) - \cos(x))}{2} + C $


%---------------------------------------------------------------
\subsection{Problem 3}
Evaluate the integral: $ \int \arctan(x) \,dx $
\subsubsection*{Solution}
This integral requires the "Stealth dx" IBP trick. Let $ u = \arctan(x) $ and $ dv = dx $.
Then $ du = \frac{1}{1+x^2} \,dx $ and $ v = x $.
The integral becomes: $ x\arctan(x) - \int \frac{x}{1+x^2} \,dx $.
The second integral is a u-substitution with $w = 1+x^2$, giving $\frac{1}{2}\ln(1+x^2)$.
\textbf{Answer:} $ x \arctan(x) - \frac{1}{2}\ln(1+x^2) + C $


%---------------------------------------------------------------
\subsection{Problem 4}
Evaluate the integral: $ \int \ln(x) \,dx $
\subsubsection*{Solution}
This uses the "Stealth dx" IBP trick. Let $ u = \ln(x) $ and $ dv = dx $.
Then $ du = \frac{1}{x} \,dx $ and $ v = x $.
The integral becomes: $ x\ln(x) - \int x \cdot \frac{1}{x} \,dx = x\ln(x) - \int 1 \,dx $.
\textbf{Answer:} $ x\ln(x) - x + C $


%---------------------------------------------------------------
\subsection{Problem 5}
Evaluate the integral: $ \int \ln(x^2 + 4) \,dx $
\subsubsection*{Solution}
Use IBP with $u=\ln(x^2+4), dv=dx$. This gives $x\ln(x^2+4) - \int \frac{2x^2}{x^2+4} \,dx$.
The second integral requires algebraic manipulation:
\[ \int \frac{2x^2}{x^2+4} \,dx = 2 \int \frac{x^2+4-4}{x^2+4} \,dx = 2 \int \left(1 - \frac{4}{x^2+4}\right) \,dx = 2(x - 2\arctan(\frac{x}{2})) \]
\textbf{Answer:} $ x\ln(x^2+4) - 2x + 4\arctan\left(\frac{x}{2}\right) + C $


%---------------------------------------------------------------
\subsection{Problem 6}
Evaluate the integral: $ \int \ln(x^2 + 5) \,dx $
\subsubsection*{Solution}
Use IBP with $u=\ln(x^2+5), dv=dx$. This gives $x\ln(x^2+5) - \int \frac{2x^2}{x^2+5} \,dx$.
The second integral requires algebraic manipulation:
\[ 2 \int \frac{x^2+5-5}{x^2+5} \,dx = 2 \int \left(1 - \frac{5}{x^2+5}\right) \,dx = 2\left(x - \frac{5}{\sqrt{5}}\arctan\left(\frac{x}{\sqrt{5}}\right)\right) \]
\textbf{Answer:} $ x\ln(x^2+5) - 2x + 2\sqrt{5}\arctan\left(\frac{x}{\sqrt{5}}\right) + C $


%---------------------------------------------------------------
\subsection{Problem 7}
Evaluate the integral: $ \int \frac{1}{x^2 + 6x + 13} \,dx $
\subsubsection*{Solution}
This integral requires completing the square in the denominator.
$ x^2 + 6x + 13 = (x^2 + 6x + 9) + 4 = (x+3)^2 + 2^2 $.
The integral becomes $ \int \frac{1}{(x+3)^2 + 2^2} \,dx $.
This is the standard arctangent form with $u=x+3$ and $a=2$.
\textbf{Answer:} $ \frac{1}{2}\arctan\left(\frac{x+3}{2}\right) + C $


%---------------------------------------------------------------
\subsection{Problem 8}
Evaluate the integral: $ \int x \arctan(x) \,dx $
\subsubsection*{Solution}
Use IBP with $u=\arctan(x), dv=x dx$. This gives $\frac{x^2}{2}\arctan(x) - \frac{1}{2} \int \frac{x^2}{1+x^2} \,dx$.
The second integral requires algebraic manipulation:
\[ \int \frac{x^2}{1+x^2} \,dx = \int \frac{1+x^2-1}{1+x^2} \,dx = \int \left(1 - \frac{1}{1+x^2}\right) \,dx = x - \arctan(x) \]
\textbf{Answer:} $ \frac{x^2}{2}\arctan(x) - \frac{1}{2}(x - \arctan(x)) + C $


%---------------------------------------------------------------
\subsection{Problem 9}
Evaluate the integral: $ \int_{0}^{\pi} e^{\cos(t)} \sin(2t) \,dt $
\subsubsection*{Solution}
Use identity $\sin(2t) = 2\sin(t)\cos(t)$. Let $u=\cos(t)$, so $du=-\sin(t)dt$.
Limits: $t=0 \to u=1$, $t=\pi \to u=-1$.
The integral becomes $ \int_{1}^{-1} e^u \cdot 2u \cdot (-du) = 2\int_{-1}^{1} u e^u \,du $.
Using IBP gives $ 2[ue^u - e^u]_{-1}^{1} = 2[(e-e) - (-e^{-1} - e^{-1})] = 2(2/e) $.
\textbf{Answer:} $ \frac{4}{e} $


%---------------------------------------------------------------
\subsection{Problem 10}
Evaluate the integral: $ \int_{\pi/6}^{\pi/2} \sin(2x) \ln(\sin(x)) \,dx $
\subsubsection*{Solution}
Use identity $\sin(2x) = 2\sin(x)\cos(x)$. Let $u=\sin(x)$, so $du=\cos(x)dx$.
Limits: $x=\pi/6 \to u=1/2$, $x=\pi/2 \to u=1$.
The integral becomes $ \int_{1/2}^{1} 2u \ln(u) \,du $.
Using IBP gives $ 2[\frac{u^2}{2}\ln(u) - \frac{u^2}{4}]_{1/2}^{1} = [\frac{1}{2}\ln(1) - \frac{1}{4}] - [\frac{(1/4)}{2}\ln(1/2) - \frac{1/4}{4}] $.
\textbf{Answer:} $ \frac{\ln(2)}{4} - \frac{3}{8} $


%---------------------------------------------------------------
\subsection{Problem 11}
Evaluate the integral: $ \int_{0}^{\pi/2} e^{\cos(x)} \sin(2x) \,dx $
\subsubsection*{Solution}
Use identity $\sin(2x) = 2\sin(x)\cos(x)$. Let $u=\cos(x)$, so $du=-\sin(x)dx$.
Limits: $x=0 \to u=1$, $x=\pi/2 \to u=0$.
The integral becomes $ \int_{1}^{0} e^u \cdot 2u \cdot (-du) = 2\int_{0}^{1} u e^u \,du $.
Using IBP gives $ 2[ue^u - e^u]_{0}^{1} = 2[(e-e) - (0-e^0)] = 2(1) $.
\textbf{Answer:} $ 2 $


%---------------------------------------------------------------
\subsection{Problem 12}
Evaluate the integral: $ \int_{\pi/3}^{\pi/2} \sin(2x) \cos(\cos(x)) \,dx $
\subsubsection*{Solution}
Use identity $\sin(2x) = 2\sin(x)\cos(x)$. Let $u=\cos(x)$, so $du=-\sin(x)dx$.
Limits: $x=\pi/3 \to u=1/2$, $x=\pi/2 \to u=0$.
The integral becomes $ \int_{1/2}^{0} 2u \cos(u) (-du) = 2\int_{0}^{1/2} u \cos(u) \,du $.
Using IBP gives $ 2[u\sin(u) + \cos(u)]_{0}^{1/2} = 2[(\frac{1}{2}\sin(\frac{1}{2}) + \cos(\frac{1}{2})) - (0 + \cos(0))] $.
\textbf{Answer:} $ \sin(\frac{1}{2}) + 2\cos(\frac{1}{2}) - 2 $


%---------------------------------------------------------------
\subsection{Problem 13}
Evaluate the integral: $ \int_{0}^{\pi/2} \sin(2x) e^{\sin(x)} \,dx $
\subsubsection*{Solution}
Use identity $\sin(2x) = 2\sin(x)\cos(x)$. Let $u=\sin(x)$, so $du=\cos(x)dx$.
Limits: $x=0 \to u=0$, $x=\pi/2 \to u=1$.
The integral becomes $ \int_{0}^{1} 2u e^u \,du $.
Using IBP gives $ 2[ue^u - e^u]_{0}^{1} = 2[(e-e) - (0-e^0)] = 2(1) $.
\textbf{Answer:} $ 2 $


%---------------------------------------------------------------
\subsection{Problem 14}
Evaluate the integral: $ \int_{0}^{\pi} \sin(2x) \cos(\cos(x)) \,dx $
\subsubsection*{Solution}
Use identity $\sin(2x) = 2\sin(x)\cos(x)$. Let $u=\cos(x)$, so $du=-\sin(x)dx$.
Limits: $x=0 \to u=1$, $x=\pi \to u=-1$.
The integral becomes $ \int_{1}^{-1} 2u \cos(u) (-du) = 2\int_{-1}^{1} u \cos(u) \,du $.
Using IBP gives $ 2[u\sin(u) + \cos(u)]_{-1}^{1} = 2[(\sin(1)+\cos(1)) - (-\sin(-1)+\cos(-1))] $.
Since sine is odd and cosine is even, this simplifies to $ 2[(\sin(1)+\cos(1)) - (\sin(1)+\cos(1))] $.
\textbf{Answer:} $ 0 $

%---------------------------------------------------------------
\subsection{Problem 15}
Evaluate the definite integral: $ \int_{0}^{\sqrt{\ln(2)}} x e^{x^2} \cos(e^{x^2}) \,dx $
\subsubsection*{Solution}
This integral can be solved with a u-substitution.
Let $u = e^{x^2}$. Then $du = 2x e^{x^2} \,dx$, which means $x e^{x^2} \,dx = \frac{1}{2}du$.
We must also change the bounds of integration.
When $x=0$, $u = e^{0^2} = 1$.
When $x=\sqrt{\ln(2)}$, $u = e^{(\sqrt{\ln(2)})^2} = e^{\ln(2)} = 2$.
The integral becomes $ \int_{1}^{2} \cos(u) \cdot \frac{1}{2}du = \frac{1}{2} \int_{1}^{2} \cos(u) \,du $.
Integrating gives $ \frac{1}{2} [\sin(u)]_{1}^{2} $.
\textbf{Answer:} $ \frac{1}{2}(\sin(2) - \sin(1)) $

%---------------------------------------------------------------
\subsection{Problem 16}
Evaluate the integral: $ \int \frac{x}{1 + \sqrt{x+4}} \,dx $
\subsubsection*{Solution}
This requires a "clever" u-substitution to rationalize the expression.
Let $u = \sqrt{x+4}$. Then $u^2 = x+4$, so $x = u^2 - 4$.
The differential is $dx = 2u \,du$.
Substitute everything into the integral: $ \int \frac{u^2 - 4}{1 + u} (2u) \,du = \int \frac{2u^3 - 8u}{u+1} \,du $.
Using polynomial long division, we get $ \int \left(2u^2 - 2u - 6 + \frac{6}{u+1}\right) \,du $.
Integrating term by term yields $ \frac{2}{3}u^3 - u^2 - 6u + 6\ln|u+1| + C $.
Finally, substitute back $u = \sqrt{x+4}$.
\textbf{Answer:} $ \frac{2}{3}(x+4)^{3/2} - (x+4) - 6\sqrt{x+4} + 6\ln|\sqrt{x+4}+1| + C $

%---------------------------------------------------------------
\subsection{Problem 17}
Evaluate the integral: $ \int \sin^2(x) \cos^4(x) \,dx $
\subsubsection*{Solution}
This integral requires power-reducing trigonometric identities.
Rewrite the integrand: $ \int (\sin(x)\cos(x))^2 \cos^2(x) \,dx = \int \left(\frac{1}{2}\sin(2x)\right)^2 \left(\frac{1+\cos(2x)}{2}\right) \,dx $.
This simplifies to $ \frac{1}{8} \int \sin^2(2x)(1+\cos(2x)) \,dx = \frac{1}{8} \int \sin^2(2x) \,dx + \frac{1}{8} \int \sin^2(2x)\cos(2x) \,dx $.
For the first integral, use the identity $ \sin^2(\theta) = \frac{1-\cos(2\theta)}{2} $: $ \frac{1}{8} \int \frac{1-\cos(4x)}{2} \,dx = \frac{1}{16}\left(x - \frac{1}{4}\sin(4x)\right) $.
For the second integral, let $u=\sin(2x)$, $du = 2\cos(2x)dx$: $ \frac{1}{16} \int u^2 \,du = \frac{1}{16} \frac{u^3}{3} = \frac{1}{48}\sin^3(2x) $.
Combining the parts gives the final answer.
\textbf{Answer:} $ \frac{1}{16}x - \frac{1}{64}\sin(4x) + \frac{1}{48}\sin^3(2x) + C $

%---------------------------------------------------------------
\subsection{Problem 18}
Evaluate the integral: $ \int \frac{2x+3}{x^2 - 4x + 20} \,dx $
\subsubsection*{Solution}
First, complete the square in the denominator: $ x^2 - 4x + 20 = (x^2 - 4x + 4) + 16 = (x-2)^2 + 16 $.
Next, split the numerator to create the derivative of the denominator's quadratic part, which is $2x-4$.
$ 2x+3 = (2x-4) + 7 $.
The integral becomes $ \int \frac{2x-4}{x^2 - 4x + 20} \,dx + \int \frac{7}{(x-2)^2 + 16} \,dx $.
The first integral is a logarithm: $ \ln(x^2 - 4x + 20) $.
The second integral is an arctangent form: $ 7 \int \frac{1}{(x-2)^2 + 4^2} \,dx = \frac{7}{4}\arctan\left(\frac{x-2}{4}\right) $.
\textbf{Answer:} $ \ln(x^2 - 4x + 20) + \frac{7}{4}\arctan\left(\frac{x-2}{4}\right) + C $

%---------------------------------------------------------------
\subsection{Problem 19}
Evaluate the integral: $ \int x \sqrt{x+1} \,dx $
\subsubsection*{Solution}
Use the substitution $u = x+1$. This implies $x = u-1$ and $dx=du$.
The integral transforms to $ \int (u-1)\sqrt{u} \,du = \int (u^{3/2} - u^{1/2}) \,du $.
Integrating term by term: $ \frac{2}{5}u^{5/2} - \frac{2}{3}u^{3/2} + C $.
Substitute back for $u=x+1$.
\textbf{Answer:} $ \frac{2}{5}(x+1)^{5/2} - \frac{2}{3}(x+1)^{3/2} + C $

%---------------------------------------------------------------
\subsection{Problem 20}
Evaluate the integral: $ \int \frac{1}{e^x + e^{-x}} \,dx $
\subsubsection*{Solution}
Multiply the numerator and denominator by $e^x$ to simplify the expression.
$ \int \frac{e^x}{(e^x)(e^x + e^{-x})} \,dx = \int \frac{e^x}{e^{2x} + 1} \,dx $.
Now, perform a u-substitution. Let $u = e^x$, so $du = e^x \,dx$.
The integral becomes $ \int \frac{1}{u^2+1} \,du $, which is the standard form for arctangent.
\textbf{Answer:} $ \arctan(e^x) + C $

%---------------------------------------------------------------
\subsection{Problem 21}
Evaluate the integral: $ \int \frac{1}{\sqrt{x} + \sqrt[3]{x}} \,dx $
\subsubsection*{Solution}
To handle the different roots, substitute using the least common multiple of the denominators of the fractional exponents (2 and 3), which is 6.
Let $u = x^{1/6}$. Then $x=u^6$ and $dx = 6u^5 \,du$. Also, $\sqrt{x} = u^3$ and $\sqrt[3]{x} = u^2$.
The integral becomes $ \int \frac{1}{u^3+u^2} (6u^5) \,du = \int \frac{6u^5}{u^2(u+1)} \,du = 6 \int \frac{u^3}{u+1} \,du $.
By polynomial long division, this is $ 6 \int \left(u^2 - u + 1 - \frac{1}{u+1}\right) \,du $.
Integrating yields $ 6\left(\frac{u^3}{3} - \frac{u^2}{2} + u - \ln|u+1|\right) + C $.
Substitute back $u=x^{1/6}$.
\textbf{Answer:} $ 2\sqrt{x} - 3\sqrt[3]{x} + 6\sqrt[6]{x} - 6\ln(x^{1/6}+1) + C $

%---------------------------------------------------------------
\subsection{Problem 22}
Evaluate the integral: $ \int \frac{x^3}{(x^2+1)^2} \,dx $
\subsubsection*{Solution}
Rewrite the integrand as $ \int \frac{x^2 \cdot x}{(x^2+1)^2} \,dx $.
Let $u = x^2+1$. Then $du = 2x\,dx$, so $x\,dx = \frac{1}{2}du$. Also, $x^2 = u-1$.
Substituting gives $ \int \frac{u-1}{u^2} \cdot \frac{1}{2}du = \frac{1}{2} \int \left(\frac{1}{u} - \frac{1}{u^2}\right) \,du $.
Integrating gives $ \frac{1}{2}\left(\ln|u| + \frac{1}{u}\right) + C $.
Substitute back $u = x^2+1$.
\textbf{Answer:} $ \frac{1}{2}\ln(x^2+1) + \frac{1}{2(x^2+1)} + C $

%---------------------------------------------------------------
\subsection{Problem 23}
Evaluate the definite integral: $ \int_{0}^{7} \frac{x}{\sqrt{x+1}} \,dx $
\subsubsection*{Solution}
Let $u = x+1$. Then $x=u-1$ and $dx=du$.
Change the bounds: when $x=0$, $u=1$; when $x=7$, $u=8$.
The integral becomes $ \int_{1}^{8} \frac{u-1}{\sqrt{u}} \,du = \int_{1}^{8} (u^{1/2} - u^{-1/2}) \,du $.
The antiderivative is $ \left[ \frac{2}{3}u^{3/2} - 2u^{1/2} \right]_{1}^{8} $.
Evaluate at the bounds: $ \left(\frac{2}{3}8^{3/2} - 2\cdot 8^{1/2}\right) - \left(\frac{2}{3} - 2\right) = \left(\frac{32\sqrt{2}}{3} - 4\sqrt{2}\right) - \left(-\frac{4}{3}\right) $.
\textbf{Answer:} $ \frac{20\sqrt{2} + 4}{3} $

%---------------------------------------------------------------
\subsection{Problem 24}
Evaluate the integral: $ \int \frac{e^x}{9 + e^{2x}} \,dx $
\subsubsection*{Solution}
This integral is in a form that leads to arctangent.
Rewrite as $ \int \frac{e^x}{3^2 + (e^x)^2} \,dx $.
Let $u = e^x$, so $du = e^x \,dx$.
The integral becomes $ \int \frac{1}{3^2+u^2} \,du $.
\textbf{Answer:} $ \frac{1}{3}\arctan\left(\frac{e^x}{3}\right) + C $

%---------------------------------------------------------------
\subsection{Problem 25}
Evaluate the integral: $ \int \frac{e^{2x}}{\sqrt{1 - e^{4x}}} \,dx $
\subsubsection*{Solution}
This integral is in a form that leads to arcsine.
Rewrite as $ \int \frac{e^{2x}}{\sqrt{1^2 - (e^{2x})^2}} \,dx $.
Let $u = e^{2x}$, so $du = 2e^{2x} \,dx$, which means $e^{2x}\,dx = \frac{1}{2}du$.
The integral becomes $ \frac{1}{2} \int \frac{1}{\sqrt{1-u^2}} \,du $.
\textbf{Answer:} $ \frac{1}{2}\arcsin(e^{2x}) + C $

%---------------------------------------------------------------
\subsection{Problem 26}
Evaluate the integral: $ \int \frac{e^x + 4}{e^{2x} + 2e^x + 5} \,dx $
\subsubsection*{Solution}
Split the integral: $ \int \frac{e^x}{e^{2x} + 2e^x + 5} \,dx + \int \frac{4}{e^{2x} + 2e^x + 5} \,dx $.
Complete the square on the denominator: $(e^x+1)^2+4$.
The first integral $I_1 = \int \frac{e^x}{(e^x+1)^2+4} \,dx$. Let $u=e^x+1, du=e^x dx$. This becomes $\int \frac{du}{u^2+4} = \frac{1}{2}\arctan(\frac{u}{2}) = \frac{1}{2}\arctan(\frac{e^x+1}{2})$.
The second integral $I_2 = 4 \int \frac{dx}{(e^x+1)^2+4}$. Let $u=e^x, dx=du/u$. This is $4 \int \frac{du}{u((u+1)^2+4)} = 4 \int \frac{du}{u(u^2+2u+5)}$.
Use partial fractions: $ \frac{4}{u(u^2+2u+5)} = \frac{4}{5u} - \frac{4u+8}{5(u^2+2u+5)} $.
This integrates to $ \frac{4}{5}\ln|u| - \frac{2}{5}\ln(u^2+2u+5) - \frac{2}{5}\arctan(\frac{u+1}{2}) $.
Substituting back $u=e^x$ and combining with $I_1$ gives the final answer.
\textbf{Answer:} $ \frac{1}{10}\arctan\left(\frac{e^x+1}{2}\right) - \frac{2}{5}\ln(e^{2x}+2e^x+5) + \frac{4}{5}x + C $

%---------------------------------------------------------------
\subsection{Problem 27}
Evaluate the integral: $ \int \frac{e^x}{e^x \sqrt{4e^{2x} - 1}} \,dx $
\subsubsection*{Solution}
First, cancel the $e^x$ terms: $ \int \frac{1}{\sqrt{4e^{2x} - 1}} \,dx $.
To transform this into a standard form, let $u=2e^x$. Then $du = 2e^x dx = u \,dx$, so $dx = \frac{du}{u}$.
The integral becomes $ \int \frac{1}{\sqrt{u^2-1}} \cdot \frac{du}{u} = \int \frac{1}{u\sqrt{u^2-1}} \,du $.
This is the standard form for the arcsecant function.
\textbf{Answer:} $ \text{arcsec}(2e^x) + C $

%---------------------------------------------------------------
\subsection{Problem 28}
Evaluate the definite integral: $ \int_{0}^{\ln(2)} \frac{1}{\sqrt{4 - e^{2x}}} \,dx $
\subsubsection*{Solution}
Let $u = e^x$. Then $dx = du/u$. The bounds become $u=1$ and $u=2$.
The integral is $ \int_{1}^{2} \frac{1}{\sqrt{4-u^2}} \frac{du}{u} = \int_{1}^{2} \frac{du}{u\sqrt{4-u^2}} $.
Use a trigonometric substitution. Let $u = 2\sin\theta$, so $du = 2\cos\theta d\theta$.
The integral becomes $ \int \frac{2\cos\theta d\theta}{2\sin\theta \sqrt{4-4\sin^2\theta}} = \int \frac{\cos\theta d\theta}{2\sin\theta\cos\theta} = \frac{1}{2}\int \csc\theta d\theta $.
The antiderivative is $ -\frac{1}{2}\ln|\csc\theta + \cot\theta| $.
Convert back to $u$: $\sin\theta = u/2$, $\csc\theta = 2/u$, $\cot\theta=\sqrt{4-u^2}/u$.
Antiderivative in terms of $u$: $ -\frac{1}{2}\ln\left|\frac{2+\sqrt{4-u^2}}{u}\right| $.
Evaluate from $u=1$ to $u=2$: $ \left[-\frac{1}{2}\ln\left|\frac{2+\sqrt{4-u^2}}{u}\right|\right]_1^2 = -\frac{1}{2}(\ln(1) - \ln(2+\sqrt{3})) $.
\textbf{Answer:} $ \frac{1}{2}\ln(2+\sqrt{3}) $

%---------------------------------------------------------------
\subsection{Problem 29}
Find the area of the region enclosed by the curves: $y = 5x - x^2$ and $y = x$.
\subsubsection*{Solution}
First, find the points of intersection by setting the equations equal to each other: $5x - x^2 = x \implies 4x - x^2 = 0 \implies x(4 - x) = 0$. The intersection points are at $x = 0$ and $x = 4$. These are the limits of integration.
In the interval $[0, 4]$, the curve $y = 5x - x^2$ is above the line $y = x$.
The area $A$ is given by the integral of the upper curve minus the lower curve:
$$ A = \int_{0}^{4} ((5x - x^2) - x) \,dx = \int_{0}^{4} (4x - x^2) \,dx $$
Now, evaluate the integral:
$$ A = \left[ 2x^2 - \frac{x^3}{3} \right]_{0}^{4} = \left(2(4)^2 - \frac{4^3}{3}\right) - (0) = 32 - \frac{64}{3} = \frac{96 - 64}{3} = \frac{32}{3} $$
\textbf{Answer:} $ \frac{32}{3} $

%---------------------------------------------------------------
\subsection{Problem 30}
Find the area of the region enclosed by the curves $y = e^x$, $y = x^6$, from $x = 0$ to $x = 1$.
\subsubsection*{Solution}
The limits of integration are given as $x=0$ and $x=1$. In this interval, $y=e^x$ is the upper curve and $y=x^6$ is the lower curve.
Set up the integral for the area $A$:
$$ A = \int_{0}^{1} (e^x - x^6) \,dx $$
Evaluate the integral:
$$ A = \left[ e^x - \frac{x^7}{7} \right]_{0}^{1} = \left(e^1 - \frac{1^7}{7}\right) - \left(e^0 - \frac{0^7}{7}\right) = \left(e - \frac{1}{7}\right) - (1 - 0) = e - \frac{8}{7} $$
\textbf{Answer:} $ e - \frac{8}{7} $

%---------------------------------------------------------------
\subsection{Problem 31}
Find the area of the region enclosed by the curves $x = y^2$, $x = y^2 - 5$, from $y = -1$ to $y = 1$.
\subsubsection*{Solution}
The region is bounded by functions of $y$, so we integrate with respect to $y$. The limits are given as $y=-1$ and $y=1$.
The right boundary is $x_{\text{right}} = y^2$ and the left boundary is $x_{\text{left}} = y^2 - 5$.
Set up the integral for the area $A$:
$$ A = \int_{-1}^{1} (x_{\text{right}} - x_{\text{left}}) \,dy = \int_{-1}^{1} (y^2 - (y^2 - 5)) \,dy = \int_{-1}^{1} 5 \,dy $$
Evaluate the integral:
$$ A = \left[ 5y \right]_{-1}^{1} = 5(1) - 5(-1) = 5 + 5 = 10 $$
\textbf{Answer:} $ 10 $

%---------------------------------------------------------------
\subsection{Problem 32}
Find the area of the region enclosed by the curves $x = 2y - y^2$ and $x = y^2 - 4y$.
\subsubsection*{Solution}
First, find the points of intersection: $2y - y^2 = y^2 - 4y \implies 2y^2 - 6y = 0 \implies 2y(y-3) = 0$. The intersection points are at $y=0$ and $y=3$.
In the interval $[0, 3]$, the curve $x = 2y - y^2$ is the right boundary.
Set up the integral with respect to $y$:
$$ A = \int_{0}^{3} ((2y - y^2) - (y^2 - 4y)) \,dy = \int_{0}^{3} (6y - 2y^2) \,dy $$
Evaluate the integral:
$$ A = \left[ 3y^2 - \frac{2y^3}{3} \right]_{0}^{3} = \left(3(3)^2 - \frac{2(3)^3}{3}\right) - (0) = 27 - 18 = 9 $$
\textbf{Answer:} $ 9 $

%---------------------------------------------------------------
\subsection{Problem 33}
Find the area of the region enclosed by the curves $y = x^3 - 15x$ and $y = x$.
\subsubsection*{Solution}
Find intersections: $x^3 - 15x = x \implies x^3 - 16x = 0 \implies x(x-4)(x+4) = 0$. Intersections are at $x=-4, 0, 4$. This defines two regions.
Region 1 (from -4 to 0): The curve $y = x^3 - 15x$ is above $y=x$.
$$ A_1 = \int_{-4}^{0} (x^3 - 16x) \,dx = \left[ \frac{x^4}{4} - 8x^2 \right]_{-4}^{0} = 0 - (64 - 128) = 64 $$
Region 2 (from 0 to 4): The line $y=x$ is above $y = x^3 - 15x$.
$$ A_2 = \int_{0}^{4} (16x - x^3) \,dx = \left[ 8x^2 - \frac{x^4}{4} \right]_{0}^{4} = (128 - 64) - 0 = 64 $$
Total Area: $A = A_1 + A_2 = 64 + 64 = 128$.
\textbf{Answer:} $ 128 $

%---------------------------------------------------------------
\subsection{Problem 34}
Find the area of the region enclosed by the curves $y = x^2$, $y = \frac{2}{3}x + \frac{16}{3}$, and $y = 8 - 2x$.
\subsubsection*{Solution}
The region must be split into two parts. The intersection points are at $x=-2, 1, 2$.
Region 1 (from -2 to 1): The upper boundary is $y = \frac{2}{3}x + \frac{16}{3}$ and the lower is $y=x^2$.
$$ A_1 = \int_{-2}^{1} \left(\frac{2}{3}x + \frac{16}{3} - x^2\right) \,dx = \left[ \frac{x^2}{3} + \frac{16x}{3} - \frac{x^3}{3} \right]_{-2}^{1} = \left(\frac{16}{3}\right) - \left(-\frac{20}{3}\right) = 12 $$
Region 2 (from 1 to 2): The upper boundary is $y = 8 - 2x$ and the lower is $y=x^2$.
$$ A_2 = \int_{1}^{2} (8 - 2x - x^2) \,dx = \left[ 8x - x^2 - \frac{x^3}{3} \right]_{1}^{2} = \left(12 - \frac{8}{3}\right) - \left(7 - \frac{1}{3}\right) = \frac{8}{3} $$
Total Area: $A = A_1 + A_2 = 12 + \frac{8}{3} = \frac{44}{3}$.
\textbf{Answer:} $ \frac{44}{3} $

%---------------------------------------------------------------
\subsection{Problem 35}
Set up an integral representing the area A of the region enclosed by the curves $x = y^4$ and $x = 2 - y^2$.
\subsubsection*{Solution}
Find intersections: $y^4 = 2 - y^2 \implies y^4 + y^2 - 2 = 0 \implies (y^2+2)(y^2-1) = 0$. The real solutions are $y = \pm 1$, which are the limits of integration.
In the interval $[-1, 1]$, the curve $x = 2 - y^2$ is the right boundary. The integral for the area $A$ is:
$$ A = \int_{-1}^{1} ((2 - y^2) - y^4) \,dy = \int_{-1}^{1} (2 - y^2 - y^4) \,dy $$
\textbf{Answer:} $ A = \int_{-1}^{1} (2 - y^2 - y^4) \,dy $

%---------------------------------------------------------------
\subsection{Problem 36}
Find the area of the region enclosed by the curves $y = 3 + x^3$, $y = 5 - x$, for $x = -1$ to $x = 0$.
\subsubsection*{Solution}
The limits are given as $x=-1$ and $x=0$. In this interval, the line $y=5-x$ is the upper boundary.
Set up the integral for the area $A$:
$$ A = \int_{-1}^{0} ((5 - x) - (3 + x^3)) \,dx = \int_{-1}^{0} (2 - x - x^3) \,dx $$
Evaluate the integral:
$$ A = \left[ 2x - \frac{x^2}{2} - \frac{x^4}{4} \right]_{-1}^{0} = (0) - \left(-2 - \frac{1}{2} - \frac{1}{4}\right) = \frac{11}{4} $$
\textbf{Answer:} $ \frac{11}{4} $

%---------------------------------------------------------------
\subsection{Problem 37}
Find the area of the region enclosed by the curves $y = 4\cos(x)$, $y = 4e^x$, and $x = \frac{\pi}{2}$.
\subsubsection*{Solution}
The curves intersect when $4\cos(x) = 4e^x$, which occurs at $x=0$. The limits of integration are $[0, \pi/2]$.
In this interval, $y=4e^x$ is the upper boundary.
$$ A = \int_{0}^{\pi/2} (4e^x - 4\cos(x)) \,dx = 4 \int_{0}^{\pi/2} (e^x - \cos(x)) \,dx $$
$$ A = 4 \left[ e^x - \sin(x) \right]_{0}^{\pi/2} = 4 \left( (e^{\pi/2} - \sin(\pi/2)) - (e^0 - \sin(0)) \right) = 4(e^{\pi/2} - 1 - 1) $$
\textbf{Answer:} $ 4e^{\pi/2} - 8 $

%---------------------------------------------------------------
\subsection{Problem 38}
Find the area of the region enclosed by the curves $y = x^2 - 4x$ and $y = 4x$.
\subsubsection*{Solution}
Find intersections: $x^2 - 4x = 4x \implies x^2 - 8x = 0 \implies x(x-8)=0$. The limits are $x=0$ and $x=8$.
In the interval $[0, 8]$, the line $y=4x$ is the upper boundary.
$$ A = \int_{0}^{8} (4x - (x^2 - 4x)) \,dx = \int_{0}^{8} (8x - x^2) \,dx $$
$$ A = \left[ 4x^2 - \frac{x^3}{3} \right]_{0}^{8} = \left(4(8)^2 - \frac{8^3}{3}\right) - 0 = 256 - \frac{512}{3} = \frac{768 - 512}{3} = \frac{256}{3} $$
\textbf{Answer:} $ \frac{256}{3} $

%---------------------------------------------------------------
\subsection{Problem 39}
Find the area of the region enclosed by the curves $x = 4 - y^2$ and $x = y^2 - 4$.
\subsubsection*{Solution}
Find intersections: $4 - y^2 = y^2 - 4 \implies 8 = 2y^2 \implies y^2 = 4$. The limits are $y=-2$ and $y=2$.
The right boundary is $x = 4 - y^2$.
$$ A = \int_{-2}^{2} ((4 - y^2) - (y^2 - 4)) \,dy = \int_{-2}^{2} (8 - 2y^2) \,dy $$
$$ A = \left[ 8y - \frac{2y^3}{3} \right]_{-2}^{2} = \left(16 - \frac{16}{3}\right) - \left(-16 + \frac{16}{3}\right) = 32 - \frac{32}{3} = \frac{64}{3} $$
\textbf{Answer:} $ \frac{64}{3} $

%---------------------------------------------------------------
\subsection{Problem 40}
Find the area of the region enclosed by the curves $2x + y^2 = 8$ and $x = y$.
\subsubsection*{Solution}
Solve for $x$: $x = 4 - \frac{1}{2}y^2$. Find intersections: $y = 4 - \frac{1}{2}y^2 \implies y^2 + 2y - 8 = 0 \implies (y+4)(y-2)=0$. The limits are $y=-4$ and $y=2$.
The right boundary is $x = 4 - \frac{1}{2}y^2$.
$$ A = \int_{-4}^{2} \left( \left(4 - \frac{1}{2}y^2\right) - y \right) \,dy $$
$$ A = \left[ 4y - \frac{y^2}{2} - \frac{y^3}{6} \right]_{-4}^{2} = \left(8 - 2 - \frac{8}{6}\right) - \left(-16 - 8 + \frac{64}{6}\right) = \left(6 - \frac{4}{3}\right) - \left(-24 + \frac{32}{3}\right) = \frac{14}{3} - \left(-\frac{40}{3}\right) = \frac{54}{3} $$
\textbf{Answer:} $ 18 $

%---------------------------------------------------------------
\subsection{Problem 41}
Find the area of the region enclosed by the curves $x = 8y^2$ and $x = 28 + y^2$.
\subsubsection*{Solution}
Find intersections: $8y^2 = 28 + y^2 \implies 7y^2 = 28 \implies y^2 = 4$. The limits are $y=-2$ and $y=2$.
The right boundary is $x = 28 + y^2$.
$$ A = \int_{-2}^{2} ((28 + y^2) - 8y^2) \,dy = \int_{-2}^{2} (28 - 7y^2) \,dy $$
$$ A = \left[ 28y - \frac{7y^3}{3} \right]_{-2}^{2} = \left(56 - \frac{56}{3}\right) - \left(-56 + \frac{56}{3}\right) = 112 - \frac{112}{3} = \frac{224}{3} $$
\textbf{Answer:} $ \frac{224}{3} $

%---------------------------------------------------------------
\subsection{Problem 42}
Find the area of the region enclosed by the curves $x = y^2 - 5$ and $x = e^y$, from $y = -1$ to $y = 1$.
\subsubsection*{Solution}
The limits are given as $y=-1$ and $y=1$. The right boundary is $x = e^y$.
$$ A = \int_{-1}^{1} (e^y - (y^2 - 5)) \,dy = \int_{-1}^{1} (e^y - y^2 + 5) \,dy $$
$$ A = \left[ e^y - \frac{y^3}{3} + 5y \right]_{-1}^{1} = \left(e - \frac{1}{3} + 5\right) - \left(e^{-1} + \frac{1}{3} - 5\right) = e - e^{-1} - \frac{2}{3} + 10 = e - \frac{1}{e} + \frac{28}{3} $$
\textbf{Answer:} $ e - \frac{1}{e} + \frac{28}{3} $

%---------------------------------------------------------------
\subsection{Problem 43}
Find the area of the region enclosed by the curves $y = \sqrt{x}$ and $y = \frac{1}{5}x$, for $0 \le x \le 36$.
\subsubsection*{Solution}
Find intersection: $\sqrt{x} = \frac{1}{5}x \implies x = \frac{1}{25}x^2 \implies x(x-25) = 0$. Intersection is at $x=25$. The region is split.
Region 1 ($0 \le x \le 25$): $y=\sqrt{x}$ is upper. $A_1 = \int_{0}^{25} (\sqrt{x} - \frac{1}{5}x) \,dx = [\frac{2}{3}x^{3/2} - \frac{x^2}{10}]_0^{25} = \frac{250}{3} - \frac{625}{10} = \frac{125}{6}$.
Region 2 ($25 \le x \le 36$): $y=\frac{1}{5}x$ is upper. $A_2 = \int_{25}^{36} (\frac{1}{5}x - \sqrt{x}) \,dx = [\frac{x^2}{10} - \frac{2}{3}x^{3/2}]_{25}^{36} = (\frac{1296}{10} - \frac{2}{3}(216)) - (\frac{625}{10} - \frac{250}{3}) = (129.6 - 144) - (62.5 - \frac{250}{3}) = -14.4 - (-\frac{125}{6}) = \frac{193}{30}$.
Total Area: $A = \frac{125}{6} + \frac{193}{30} = \frac{625+193}{30} = \frac{818}{30} = \frac{409}{15}$.
\textbf{Answer:} $ \frac{409}{15} $

%---------------------------------------------------------------
\subsection{Problem 44}
Find the area of the region enclosed by the curves $y = \cos(x)$ and $y = 2 - \cos(x)$, for $0 \le x \le 2\pi$.
\subsubsection*{Solution}
In the interval $[0, 2\pi]$, the curve $y = 2 - \cos(x)$ is always above $y = \cos(x)$. They touch at $x=0$ and $x=2\pi$.
$$ A = \int_{0}^{2\pi} ((2 - \cos(x)) - \cos(x)) \,dx = \int_{0}^{2\pi} (2 - 2\cos(x)) \,dx $$
$$ A = \left[ 2x - 2\sin(x) \right]_{0}^{2\pi} = (4\pi - 2\sin(2\pi)) - (0 - 2\sin(0)) = 4\pi $$
\textbf{Answer:} $ 4\pi $

%---------------------------------------------------------------
\subsection{Problem 45}
Find the area of the region enclosed by the curves $y = \cos(x)$ and $y = \sin(2x)$, for $0 \le x \le \frac{\pi}{2}$.
\subsubsection*{Solution}
Find intersection: $\cos(x) = \sin(2x) = 2\sin(x)\cos(x) \implies \cos(x)(1 - 2\sin(x)) = 0$. Intersections at $x = \pi/6$ and $x = \pi/2$.
Region 1 ($0 \le x \le \pi/6$): $y=\cos(x)$ is upper. $A_1 = \int_{0}^{\pi/6} (\cos(x) - \sin(2x)) \,dx = [\sin(x) + \frac{1}{2}\cos(2x)]_0^{\pi/6} = (\frac{1}{2} + \frac{1}{4}) - (\frac{1}{2}) = \frac{1}{4}$.
Region 2 ($\pi/6 \le x \le \pi/2$): $y=\sin(2x)$ is upper. $A_2 = \int_{\pi/6}^{\pi/2} (\sin(2x) - \cos(x)) \,dx = [-\frac{1}{2}\cos(2x) - \sin(x)]_{\pi/6}^{\pi/2} = (\frac{1}{2} - 1) - (-\frac{1}{4} - \frac{1}{2}) = -\frac{1}{2} - (-\frac{3}{4}) = \frac{1}{4}$.
Total Area: $A = A_1 + A_2 = \frac{1}{4} + \frac{1}{4} = \frac{1}{2}$.
\textbf{Answer:} $ \frac{1}{2} $
%---------------------------------------------------------------
%---------------------------------------------------------------
\subsection{Problem 46}
Evaluate the integral. (Remember the constant of integration.)
$$ \int 2 \sin^2(x) \cos^3(x) \,dx $$
\subsubsection*{Solution}
We have an odd power of cosine, so we save one cosine factor and convert the rest to sines using $\cos^2(x) = 1 - \sin^2(x)$.
$$ \int 2 \sin^2(x) \cos^2(x) \cos(x) \,dx = \int 2 \sin^2(x) (1 - \sin^2(x)) \cos(x) \,dx $$
Let $u = \sin(x)$, so $du = \cos(x) \,dx$. The integral becomes:
$$ \int 2 u^2 (1 - u^2) \,du = \int (2u^2 - 2u^4) \,du $$
$$ = \frac{2}{3}u^3 - \frac{2}{5}u^5 + C $$
Substituting back $u = \sin(x)$:
$$ = \frac{2}{3}\sin^3(x) - \frac{2}{5}\sin^5(x) + C $$
\textbf{Answer:} $ \frac{2}{3}\sin^3(x) - \frac{2}{5}\sin^5(x) + C $
%---------------------------------------------------------------
\subsection{Problem 47}
Evaluate the integral. (Remember the constant of integration.)
$$ \int \sin^3(y) \cos^4(y) \,dy $$
\subsubsection*{Solution}
We have an odd power of sine, so we save one sine factor and convert the rest to cosines using $\sin^2(y) = 1 - \cos^2(y)$.
$$ \int \sin^2(y) \cos^4(y) \sin(y) \,dy = \int (1 - \cos^2(y)) \cos^4(y) \sin(y) \,dy $$
Let $u = \cos(y)$, so $du = -\sin(y) \,dy$. The integral becomes:
$$ \int (1 - u^2) u^4 (-du) = -\int (u^4 - u^6) \,du $$
$$ = - \left( \frac{u^5}{5} - \frac{u^7}{7} \right) + C = \frac{1}{7}u^7 - \frac{1}{5}u^5 + C $$
Substituting back $u = \cos(y)$:
$$ = \frac{1}{7}\cos^7(y) - \frac{1}{5}\cos^5(y) + C $$
\textbf{Answer:} $ \frac{1}{7}\cos^7(y) - \frac{1}{5}\cos^5(y) + C $
%---------------------------------------------------------------
\subsection{Problem 48}
Evaluate the integral.
$$ \int_{0}^{\pi/2} \cos^{13}(x) \sin^5(x) \,dx $$
\subsubsection*{Solution}
The power of sine is odd. We save a $\sin(x)$ factor and convert the rest.
$$ \int_{0}^{\pi/2} \cos^{13}(x) \sin^4(x) \sin(x) \,dx = \int_{0}^{\pi/2} \cos^{13}(x) (1-\cos^2(x))^2 \sin(x) \,dx $$
Let $u = \cos(x)$, so $du = -\sin(x) \,dx$. The bounds change: $x=0 \implies u=1$, and $x=\pi/2 \implies u=0$.
$$ \int_{1}^{0} u^{13} (1-u^2)^2 (-du) = \int_{0}^{1} u^{13} (1 - 2u^2 + u^4) \,du $$
$$ = \int_{0}^{1} (u^{13} - 2u^{15} + u^{17}) \,du = \left[ \frac{u^{14}}{14} - \frac{2u^{16}}{16} + \frac{u^{18}}{18} \right]_{0}^{1} $$
$$ = \left( \frac{1}{14} - \frac{1}{8} + \frac{1}{18} \right) - (0) = \frac{36 - 63 + 28}{504} = \frac{1}{504} $$
\textbf{Answer:} $ \frac{1}{504} $
%---------------------------------------------------------------
\subsection{Problem 49}
Evaluate the integral.
$$ \int_{0}^{\pi/2} 9 \sin^2(x) \cos^2(x) \,dx $$
\subsubsection*{Solution}
We use the identity $\sin(x)\cos(x) = \frac{1}{2}\sin(2x)$, and then $\sin^2(\theta) = \frac{1 - \cos(2\theta)}{2}$.
$$ \int_{0}^{\pi/2} 9 (\sin(x) \cos(x))^2 \,dx = \int_{0}^{\pi/2} 9 \left(\frac{1}{2}\sin(2x)\right)^2 \,dx = \frac{9}{4} \int_{0}^{\pi/2} \sin^2(2x) \,dx $$
$$ = \frac{9}{4} \int_{0}^{\pi/2} \frac{1 - \cos(4x)}{2} \,dx = \frac{9}{8} \int_{0}^{\pi/2} (1 - \cos(4x)) \,dx $$
$$ = \frac{9}{8} \left[ x - \frac{1}{4}\sin(4x) \right]_{0}^{\pi/2} = \frac{9}{8} \left[ (\frac{\pi}{2} - \frac{1}{4}\sin(2\pi)) - (0 - \frac{1}{4}\sin(0)) \right] $$
$$ = \frac{9}{8} \left( \frac{\pi}{2} \right) = \frac{9\pi}{16} $$
\textbf{Answer:} $ \frac{9\pi}{16} $
%---------------------------------------------------------------
\subsection{Problem 50}
Evaluate the integral.
$$ \int_{0}^{\pi/2} 5 \cos^2(\theta) \,d\theta $$
\subsubsection*{Solution}
Using the half-angle identity $\cos^2(\theta) = \frac{1 + \cos(2\theta)}{2}$.
$$ \int_{0}^{\pi/2} 5 \left(\frac{1 + \cos(2\theta)}{2}\right) \,d\theta = \frac{5}{2} \int_{0}^{\pi/2} (1 + \cos(2\theta)) \,d\theta $$
$$ = \frac{5}{2} \left[ \theta + \frac{1}{2}\sin(2\theta) \right]_{0}^{\pi/2} = \frac{5}{2} \left[ (\frac{\pi}{2} + \frac{1}{2}\sin(\pi)) - (0 + \frac{1}{2}\sin(0)) \right] $$
$$ = \frac{5}{2} \left( \frac{\pi}{2} \right) = \frac{5\pi}{4} $$
\textbf{Answer:} $ \frac{5\pi}{4} $
%---------------------------------------------------------------
\subsection{Problem 51}
Evaluate the integral.
$$ \int \sqrt{\cos(\theta)} \sin^3(\theta) \,d\theta $$
\subsubsection*{Solution}
The power of sine is odd. We save a $\sin(\theta)$ factor.
$$ \int \sqrt{\cos(\theta)} \sin^2(\theta) \sin(\theta) \,d\theta = \int \sqrt{\cos(\theta)} (1 - \cos^2(\theta)) \sin(\theta) \,d\theta $$
Let $u = \cos(\theta)$, so $du = -\sin(\theta) \,d\theta$.
$$ \int \sqrt{u} (1 - u^2) (-du) = -\int (u^{1/2} - u^{5/2}) \,du $$
$$ = - \left( \frac{u^{3/2}}{3/2} - \frac{u^{7/2}}{7/2} \right) + C = -\frac{2}{3}u^{3/2} + \frac{2}{7}u^{7/2} + C $$
$$ = \frac{2}{7}\cos^{7/2}(\theta) - \frac{2}{3}\cos^{3/2}(\theta) + C $$
\textbf{Answer:} $ \frac{2}{7}\cos^{7/2}(\theta) - \frac{2}{3}\cos^{3/2}(\theta) + C $
%---------------------------------------------------------------
\subsection{Problem 52}
Evaluate the integral.
$$ \int \sin(3x) \sec^5(3x) \,dx $$
\subsubsection*{Solution}
Rewrite $\sec(3x)$ as $1/\cos(3x)$.
$$ \int \sin(3x) \frac{1}{\cos^5(3x)} \,dx = \int \frac{\sin(3x)}{\cos^5(3x)} \,dx $$
Let $u = \cos(3x)$, so $du = -3\sin(3x) \,dx$, which means $\sin(3x)dx = -du/3$.
$$ \int \frac{1}{u^5} \left(-\frac{du}{3}\right) = -\frac{1}{3} \int u^{-5} \,du $$
$$ = -\frac{1}{3} \frac{u^{-4}}{-4} + C = \frac{1}{12}u^{-4} + C = \frac{1}{12\cos^4(3x)} + C $$
$$ = \frac{1}{12}\sec^4(3x) + C $$
\textbf{Answer:} $ \frac{1}{12}\sec^4(3x) + C $
%---------------------------------------------------------------
\subsection{Problem 53}
Evaluate the integral.
$$ \int 4 \tan(x) \sec^3(x) \,dx $$
\subsubsection*{Solution}
We can rewrite the integrand to isolate a $\sec(x)\tan(x)$ factor.
$$ 4 \int \sec^2(x) (\sec(x)\tan(x)) \,dx $$
Let $u = \sec(x)$, so $du = \sec(x)\tan(x) \,dx$.
$$ 4 \int u^2 \,du = 4 \left( \frac{u^3}{3} \right) + C = \frac{4}{3}u^3 + C $$
Substituting back $u = \sec(x)$:
$$ = \frac{4}{3}\sec^3(x) + C $$
\textbf{Answer:} $ \frac{4}{3}\sec^3(x) + C $
%---------------------------------------------------------------
\subsection{Problem 54}
Evaluate the integral.
$$ \int 5 \tan^2(x) \,dx $$
\subsubsection*{Solution}
Using the identity $\tan^2(x) = \sec^2(x) - 1$.
$$ 5 \int (\sec^2(x) - 1) \,dx = 5 (\int \sec^2(x) \,dx - \int 1 \,dx) $$
$$ = 5(\tan(x) - x) + C = 5\tan(x) - 5x + C $$
\textbf{Answer:} $ 5\tan(x) - 5x + C $
%---------------------------------------------------------------
\subsection{Problem 55}
Evaluate the integral.
$$ \int 11 \tan^4(x) \sec^6(x) \,dx $$
\subsubsection*{Solution}
The power of secant is even. We save a $\sec^2(x)$ factor and convert the rest to tangents using $\sec^2(x) = 1 + \tan^2(x)$.
$$ 11 \int \tan^4(x) \sec^4(x) \sec^2(x) \,dx = 11 \int \tan^4(x) (1 + \tan^2(x))^2 \sec^2(x) \,dx $$
Let $u = \tan(x)$, so $du = \sec^2(x) \,dx$.
$$ 11 \int u^4 (1 + u^2)^2 \,du = 11 \int u^4 (1 + 2u^2 + u^4) \,du $$
$$ = 11 \int (u^4 + 2u^6 + u^8) \,du = 11 \left( \frac{u^5}{5} + \frac{2u^7}{7} + \frac{u^9}{9} \right) + C $$
$$ = \frac{11}{5}\tan^5(x) + \frac{22}{7}\tan^7(x) + \frac{11}{9}\tan^9(x) + C $$
\textbf{Answer:} $ \frac{11}{9}\tan^9(x) + \frac{22}{7}\tan^7(x) + \frac{11}{5}\tan^5(x) + C $
%---------------------------------------------------------------
\subsection{Problem 56}
Evaluate the integral.
$$ \int \tan^3(x) \sec(x) \,dx $$
\subsubsection*{Solution}
The power of tangent is odd. We save a $\sec(x)\tan(x)$ factor and convert the rest to secants.
$$ \int \tan^2(x) (\sec(x)\tan(x)) \,dx = \int (\sec^2(x) - 1) (\sec(x)\tan(x)) \,dx $$
Let $u = \sec(x)$, so $du = \sec(x)\tan(x) \,dx$.
$$ \int (u^2 - 1) \,du = \frac{u^3}{3} - u + C $$
$$ = \frac{1}{3}\sec^3(x) - \sec(x) + C $$
\textbf{Answer:} $ \frac{1}{3}\sec^3(x) - \sec(x) + C $
%---------------------------------------------------------------
\subsection{Problem 57}
Evaluate the integral.
$$ \int \tan^3(x) \sec^6(x) \,dx $$
\subsubsection*{Solution}
The power of tangent is odd, so we can let $u=\sec(x)$.
$$ \int \tan^2(x) \sec^5(x) (\sec(x)\tan(x)) \,dx = \int (\sec^2(x) - 1) \sec^5(x) (\sec(x)\tan(x)) \,dx $$
Let $u = \sec(x)$, so $du = \sec(x)\tan(x) \,dx$.
$$ \int (u^2 - 1) u^5 \,du = \int (u^7 - u^5) \,du $$
$$ = \frac{u^8}{8} - \frac{u^6}{6} + C = \frac{1}{8}\sec^8(x) - \frac{1}{6}\sec^6(x) + C $$
\textbf{Answer:} $ \frac{1}{8}\sec^8(x) - \frac{1}{6}\sec^6(x) + C $
%---------------------------------------------------------------
\subsection{Problem 58}
Evaluate the integral.
$$ \int_{0}^{\pi/6} \tan^4(t) \,dt $$
\subsubsection*{Solution}
We reduce the power of tangent using $\tan^2(t) = \sec^2(t) - 1$.
$$ \int_{0}^{\pi/6} \tan^2(t) (\sec^2(t) - 1) \,dt = \int_{0}^{\pi/6} \tan^2(t)\sec^2(t) \,dt - \int_{0}^{\pi/6} \tan^2(t) \,dt $$
The first integral is $\int u^2 du$ with $u=\tan(t)$. The second integral becomes $\int (\sec^2(t)-1)dt$.
$$ \left[ \frac{1}{3}\tan^3(t) \right]_{0}^{\pi/6} - \int_{0}^{\pi/6} (\sec^2(t) - 1) \,dt $$
$$ = \left[ \frac{1}{3}\tan^3(t) \right]_{0}^{\pi/6} - \left[ \tan(t) - t \right]_{0}^{\pi/6} = \left[ \frac{1}{3}\tan^3(t) - \tan(t) + t \right]_{0}^{\pi/6} $$
Since $\tan(\pi/6) = 1/\sqrt{3}$:
$$ = \left( \frac{1}{3}\left(\frac{1}{\sqrt{3}}\right)^3 - \frac{1}{\sqrt{3}} + \frac{\pi}{6} \right) - (0) = \frac{1}{9\sqrt{3}} - \frac{1}{\sqrt{3}} + \frac{\pi}{6} $$
$$ = \frac{1 - 9}{9\sqrt{3}} + \frac{\pi}{6} = -\frac{8}{9\sqrt{3}} + \frac{\pi}{6} = \frac{\pi}{6} - \frac{8\sqrt{3}}{27} $$
\textbf{Answer:} $ \frac{\pi}{6} - \frac{8\sqrt{3}}{27} $
%---------------------------------------------------------------
\subsection{Problem 59}
Evaluate the integral.
$$ \int \tan^5(x) \,dx $$
\subsubsection*{Solution}
We reduce the power of tangent.
$$ \int \tan^3(x) \tan^2(x) \,dx = \int \tan^3(x) (\sec^2(x) - 1) \,dx $$
$$ = \int \tan^3(x)\sec^2(x) \,dx - \int \tan^3(x) \,dx $$
The first integral is $\frac{1}{4}\tan^4(x)$. For the second integral:
$$ \int \tan^3(x) \,dx = \int \tan(x)(\sec^2(x)-1) \,dx = \int \tan(x)\sec^2(x) \,dx - \int \tan(x) \,dx $$
$$ = \frac{1}{2}\tan^2(x) - \ln|\sec(x)| $$
Combining the results:
$$ \frac{1}{4}\tan^4(x) - \left( \frac{1}{2}\tan^2(x) - \ln|\sec(x)| \right) + C $$
\textbf{Answer:} $ \frac{1}{4}\tan^4(x) - \frac{1}{2}\tan^2(x) + \ln|\sec(x)| + C $
%---------------------------------------------------------------
\subsection{Problem 60}
Evaluate the integral.
$$ \int \frac{\tan(x) \sec^2(x)}{\cos(x)} \,dx $$
\subsubsection*{Solution}
Since $1/\cos(x) = \sec(x)$, the integral becomes:
$$ \int \tan(x) \sec^3(x) \,dx = \int \sec^2(x) (\sec(x)\tan(x)) \,dx $$
Let $u = \sec(x)$, so $du = \sec(x)\tan(x) \,dx$.
$$ \int u^2 \,du = \frac{u^3}{3} + C $$
$$ = \frac{1}{3}\sec^3(x) + C $$
\textbf{Answer:} $ \frac{1}{3}\sec^3(x) + C $
%---------------------------------------------------------------
\subsection{Problem 61}
Evaluate the integral.
$$ \int_{\pi/6}^{\pi/2} 5 \cot^2(x) \,dx $$
\subsubsection*{Solution}
Using the identity $\cot^2(x) = \csc^2(x) - 1$.
$$ 5 \int_{\pi/6}^{\pi/2} (\csc^2(x) - 1) \,dx = 5 \left[ -\cot(x) - x \right]_{\pi/6}^{\pi/2} $$
$$ = 5 \left[ (-\cot(\frac{\pi}{2}) - \frac{\pi}{2}) - (-\cot(\frac{\pi}{6}) - \frac{\pi}{6}) \right] $$
Since $\cot(\pi/2) = 0$ and $\cot(\pi/6) = \sqrt{3}$:
$$ = 5 \left[ (0 - \frac{\pi}{2}) - (-\sqrt{3} - \frac{\pi}{6}) \right] = 5 \left[ -\frac{\pi}{2} + \sqrt{3} + \frac{\pi}{6} \right] $$
$$ = 5 \left[ \sqrt{3} - \frac{3\pi}{6} + \frac{\pi}{6} \right] = 5 \left( \sqrt{3} - \frac{2\pi}{6} \right) = 5\left(\sqrt{3} - \frac{\pi}{3}\right) $$
\textbf{Answer:} $ 5\left(\sqrt{3} - \frac{\pi}{3}\right) $
%---------------------------------------------------------------
\subsection{Problem 62}
Evaluate the integral.
$$ \int \sin(8x) \cos(5x) \,dx $$
\subsubsection*{Solution}
We use the product-to-sum identity $\sin(A)\cos(B) = \frac{1}{2}[\sin(A+B) + \sin(A-B)]$.
$$ \int \frac{1}{2}[\sin(8x+5x) + \sin(8x-5x)] \,dx = \frac{1}{2} \int (\sin(13x) + \sin(3x)) \,dx $$
$$ = \frac{1}{2} \left( -\frac{1}{13}\cos(13x) - \frac{1}{3}\cos(3x) \right) + C $$
$$ = -\frac{1}{26}\cos(13x) - \frac{1}{6}\cos(3x) + C $$
\textbf{Answer:} $ -\frac{1}{26}\cos(13x) - \frac{1}{6}\cos(3x) + C $
%---------------------------------------------------------------
\subsection{Problem 63}
Evaluate the integral.
$$ \int 5 \tan^2(x) \sec(x) \,dx $$
\subsubsection*{Solution}
Using the identity $\tan^2(x) = \sec^2(x) - 1$.
$$ 5 \int (\sec^2(x) - 1) \sec(x) \,dx = 5 \int (\sec^3(x) - \sec(x)) \,dx $$
We use the standard integrals for $\sec^3(x)$ and $\sec(x)$.
$$ \int \sec(x) \,dx = \ln|\sec(x) + \tan(x)| + C $$
$$ \int \sec^3(x) \,dx = \frac{1}{2}(\sec(x)\tan(x) + \ln|\sec(x) + \tan(x)|) + C $$
So, the integral is:
$$ 5 \left[ \frac{1}{2}(\sec(x)\tan(x) + \ln|\sec(x) + \tan(x)|) - \ln|\sec(x) + \tan(x)| \right] + C $$
$$ = 5 \left[ \frac{1}{2}\sec(x)\tan(x) - \frac{1}{2}\ln|\sec(x) + \tan(x)| \right] + C $$
\textbf{Answer:} $ \frac{5}{2}(\sec(x)\tan(x) - \ln|\sec(x) + \tan(x)|) + C $



\newpage
\section{Summary of Rules, Formulas, and Tricks}
This document covers a wide range of integration techniques, from basic substitutions to multi-stage problems requiring a sequence of tricks.

\subsection*{Category 1: U-Substitution}
\begin{itemize}
    \item \textbf{"Clever" First-Step Substitution:} Recognizing a non-obvious substitution that dramatically simplifies the integral from the start (e.g., Problem 1).
    \item \textbf{Standard Second-Step Substitution:} A more routine substitution that becomes necessary *after* an initial step like IBP has been applied.
\end{itemize}

\subsection*{Category 2: Integration by Parts (IBP)}
\begin{itemize}
    \item \textbf{The "Stealth dx" Trick:} Creating a product by setting $dv=dx$ to integrate single functions like $\ln(x)$ or $\arctan(x)$ (e.g., Problems 3, 4, 5, 6).
    \item \textbf{Looping / Circular Integration:} Applying IBP twice to problems like $ \int e^x\sin(x)dx $ to solve for the original integral algebraically (e.g., Problem 2).
    \item \textbf{Standard IBP:} The direct application of the formula, often used as a key step in a larger problem (e.g., Problem 8 and the definite integral sequence).
\end{itemize}

\subsection*{Category 3: Algebraic \& Trigonometric Manipulation}
\begin{itemize}
    \item \textbf{Trigonometric Identities:} Using identities like $ \sin(2x) = 2\sin(x)\cos(x) $ to unlock the problem (e.g., Problems 9-14).
    \item \textbf{The "Add and Subtract" Trick:} A shortcut for polynomial long division to simplify rational functions where the numerator's degree is equal to the denominator's (e.g., Problems 5, 6, 8).
    \item \textbf{Completing the Square:} Rewriting a quadratic denominator to fit the standard arctangent form (e.g., Problem 7).
\end{itemize}

\subsection*{Category 4: Multi-Stage "Grand Challenge" Problems}
\begin{itemize}
    \item \textbf{The "Trig Identity $\to$ U-Sub $\to$ IBP" Sequence:} A powerful pattern for solving complex definite integrals (e.g., Problems 9-14).
    \item \textbf{The "IBP $\to$ Algebraic Trick $\to$ Standard Form" Sequence:} A common pattern for integrating logarithmic and inverse trig functions (e.g., Problems 5, 6, 8).
\end{itemize}

\subsection*{Category 5: Definite Integral Skills}
\begin{itemize}
    \item \textbf{Changing Limits of Integration:} A mandatory step when performing u-substitution on a definite integral to avoid having to substitute back.
    \item \textbf{Flipping Limits of Integration:} Using the property $ \int_a^b f(x)dx = -\int_b^a f(x)dx $ to cancel negative signs and simplify calculations.
    \item \textbf{The F(b) - F(a) Evaluation:} The final, crucial arithmetic step of the Fundamental Theorem of Calculus.
    \item \textbf{Understanding Function Properties (Even/Odd):} Using symmetry, like $ \cos(-x)=\cos(x) $, to simplify the final evaluation (e.g., Problem 14).
\end{itemize}

\subsection*{Untouched Tricks \& Problem Types}
The following are integration strategies that were identified but not yet practiced.
\begin{itemize}
    \item \textbf{The DI Method (Tabular Method):} A shortcut for repeated IBP.
    \item \textbf{Reduction Formulas:} Using IBP to create a formula relating an integral to a simpler version of itself.
    \item \textbf{Partial Fraction Decomposition:} The main algebraic technique for integrating more complex rational functions.
    \item \textbf{The "King Property" Connection:} A specific trick for definite integrals.
    \item \textbf{Feynman's Trick (Differentiation Under the Integral Sign):} A powerful method where you introduce a parameter into the integral, differentiate with respect to that parameter, and then integrate back.
    \item \textbf{Weierstrass Substitution (Tangent Half-Angle Substitution):} A substitution that can convert any rational function of trigonometric functions into an algebraic rational function, which can then be solved using partial fractions.
    \item \textbf{The "King Property" Connection:} A specific trick for definite integrals.
    \item \textbf{Contour Integration and the Residue Theorem:}These are techniques from complex analysis that can be used to solve very difficult real-valued definite integrals.
    \item \textbf{Meijer G-Function:} A highly generalized function that can represent most elementary and many special functions. There are rules for finding the antiderivative of a Meijer G-function, making it a powerful, though complex, symbolic integration method.
 
    
\end{itemize}

\end{document}