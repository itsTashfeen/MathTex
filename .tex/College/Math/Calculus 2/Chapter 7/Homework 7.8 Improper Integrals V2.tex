\documentclass{article}
\usepackage{amsmath}
\usepackage{amssymb}
\usepackage[margin=1in]{geometry}

\title{Homework 7.8 Improper Integrals}
\author{Tashfeen Omran}
\date{\today}

\begin{document}

\maketitle

\part{Comprehensive Introduction, Context, and Prerequisites}

\section{Core Concepts}
An improper integral is a definite integral that has at least one of two potential issues:
\begin{enumerate}
    \item \textbf{Type 1: Infinite Limits of Integration.} The interval of integration is infinite, such as $[a, \infty)$, $(-\infty, b]$, or $(-\infty, \infty)$.
    \item \textbf{Type 2: Infinite Discontinuities.} The integrand $f(x)$ has an infinite discontinuity (a vertical asymptote) at some point $c$ within the interval of integration $[a, b]$.
\end{enumerate}
The fundamental idea is to handle the "improper" part by turning it into a limit problem. We integrate over a finite interval and then take the limit as that interval approaches the infinite or discontinuous point.

If the limit exists and is a finite number, we say the integral \textbf{converges}. If the limit does not exist or is infinite, we say the integral \textbf{diverges}.

\section{Intuition and Derivation}
The definite integral $\int_{a}^{b} f(x) \,dx$ represents the area under the curve $y=f(x)$ over a finite interval $[a, b]$. But what if the region is infinitely long or tall? Can an infinitely large region have a finite area? Calculus provides the answer through limits.

For a Type 1 integral like $\int_{1}^{\infty} \frac{1}{x^2} \,dx$, we ask: what happens to the area as we integrate from $1$ to some large number $t$, and then let $t$ grow without bound? We calculate the finite area $A(t) = \int_{1}^{t} \frac{1}{x^2} \,dx$ and then evaluate $\lim_{t \to \infty} A(t)$. If this limit is a finite number, that number is the area of the infinite region. This powerful idea allows us to "tame infinity" and assign a concrete value to a seemingly immeasurable quantity.

\section{Historical Context and Motivation}
The rigorous treatment of improper integrals evolved during the 18th and 19th centuries, well after the initial development of calculus by Newton and Leibniz. Early mathematicians grappled with paradoxes of the infinite. A classic motivating problem is \textbf{Gabriel's Horn}, formed by revolving the curve $y=1/x$ for $x \ge 1$ about the x-axis. Using improper integrals, one can show the astonishing result that the horn has a finite volume but an infinite surface area. This means you could fill the horn with a finite amount of paint, but you could never paint its infinite surface! Such counter-intuitive results demanded a formal and precise method for handling integrals over unbounded regions, which was a primary motivation for developing the theory of improper integrals.

\section{Key Formulas}
\subsection{Type 1: Infinite Intervals}
Assume $f(x)$ is continuous on the interval of integration.
\begin{itemize}
    \item $\displaystyle \int_{a}^{\infty} f(x) \,dx = \lim_{t \to \infty} \int_{a}^{t} f(x) \,dx$
    \item $\displaystyle \int_{-\infty}^{b} f(x) \,dx = \lim_{t \to -\infty} \int_{t}^{b} f(x) \,dx$
    \item $\displaystyle \int_{-\infty}^{\infty} f(x) \,dx = \int_{-\infty}^{c} f(x) \,dx + \int_{c}^{\infty} f(x) \,dx$ \\ (This requires both integrals on the right to converge independently. The choice of $c$ is arbitrary; 0 is often convenient.)
\end{itemize}

\subsection{Type 2: Discontinuous Integrands}
\begin{itemize}
    \item If $f$ is discontinuous at $b$: $\displaystyle \int_{a}^{b} f(x) \,dx = \lim_{t \to b^{-}} \int_{a}^{t} f(x) \,dx$
    \item If $f$ is discontinuous at $a$: $\displaystyle \int_{a}^{b} f(x) \,dx = \lim_{t \to a^{+}} \int_{t}^{b} f(x) \,dx$
    \item If $f$ is discontinuous at $c$ where $a < c < b$: \\ $\displaystyle \int_{a}^{b} f(x) \,dx = \int_{a}^{c} f(x) \,dx + \int_{c}^{b} f(x) \,dx$ \\ (Again, both integrals on the right must converge for the original to converge.)
\end{itemize}

\subsection{The p-Test for Improper Integrals}
A crucial benchmark for comparison:
\[ \int_{1}^{\infty} \frac{1}{x^p} \,dx \quad \text{is} \quad
\begin{cases}
    \text{convergent} & \text{if } p > 1 \\
    \text{divergent} & \text{if } p \le 1
\end{cases}
\]

\section{Prerequisites}
\begin{itemize}
    \item \textbf{Limit Evaluation:} Mastery of limits, especially limits at infinity, one-sided limits, and L'Hôpital's Rule.
    \item \textbf{Integration Techniques:} All foundational techniques are required: u-substitution, integration by parts, trigonometric substitution, and partial fraction decomposition.
    \item \textbf{The Fundamental Theorem of Calculus:} The mechanism for evaluating all definite integrals that arise in the limit steps.
    \item \textbf{Algebraic Manipulation:} Skill in factoring, simplifying complex fractions, and working with exponents and logarithms.
    \item \textbf{Function Behavior:} Ability to identify discontinuities (vertical asymptotes) in a function.
\end{itemize}

\part{Detailed Homework Solutions}

\section{Problem 1}
Determine whether the integral is convergent or divergent. If it is convergent, evaluate it.
\[ \int_{1}^{\infty} 3x^{-4} \,dx \]
\textbf{Solution:}
This is a Type 1 improper integral.
\begin{align*}
    \int_{1}^{\infty} 3x^{-4} \,dx &= \lim_{t \to \infty} \int_{1}^{t} 3x^{-4} \,dx \\
    &= \lim_{t \to \infty} \left[ 3 \frac{x^{-3}}{-3} \right]_{1}^{t} \\
    &= \lim_{t \to \infty} \left[ -x^{-3} \right]_{1}^{t} = \lim_{t \to \infty} \left[ -\frac{1}{x^3} \right]_{1}^{t} \\
    &= \lim_{t \to \infty} \left( -\frac{1}{t^3} - \left(-\frac{1}{1^3}\right) \right) \\
    &= \lim_{t \to \infty} \left( -\frac{1}{t^3} + 1 \right) \\
    &= 0 + 1 = 1
\end{align*}
\textbf{Answer:} The integral is convergent and its value is 1.

\section{Problem 2}
Determine whether the integral is convergent or divergent. If it is convergent, evaluate it.
\[ \int_{-\infty}^{-1} \frac{1}{\sqrt[5]{x}} \,dx = \int_{-\infty}^{-1} x^{-1/5} \,dx \]
\textbf{Solution:}
This is a Type 1 improper integral.
\begin{align*}
    \int_{-\infty}^{-1} x^{-1/5} \,dx &= \lim_{t \to -\infty} \int_{t}^{-1} x^{-1/5} \,dx \\
    &= \lim_{t \to -\infty} \left[ \frac{x^{4/5}}{4/5} \right]_{t}^{-1} = \lim_{t \to -\infty} \left[ \frac{5}{4}x^{4/5} \right]_{t}^{-1} \\
    &= \lim_{t \to -\infty} \left( \frac{5}{4}(-1)^{4/5} - \frac{5}{4}t^{4/5} \right) \\
    &= \lim_{t \to -\infty} \left( \frac{5}{4}(1) - \frac{5}{4}(\sqrt[5]{t})^4 \right) \\
    &= \frac{5}{4} - \infty
\end{align*}
As $t \to -\infty$, $t^{4/5} = (\sqrt[5]{t})^4 \to \infty$. The limit does not exist.
\textbf{Answer:} DIVERGES.

\section{Problem 3}
Determine whether the integral is convergent or divergent. If it is convergent, evaluate it.
\[ \int_{-6}^{\infty} \frac{1}{x+7} \,dx \]
\textbf{Solution:}
This is a Type 1 improper integral.
\begin{align*}
    \int_{-6}^{\infty} \frac{1}{x+7} \,dx &= \lim_{t \to \infty} \int_{-6}^{t} \frac{1}{x+7} \,dx \\
    &= \lim_{t \to \infty} \left[ \ln|x+7| \right]_{-6}^{t} \\
    &= \lim_{t \to \infty} (\ln|t+7| - \ln|-6+7|) \\
    &= \lim_{t \to \infty} (\ln(t+7) - \ln(1)) \\
    &= \infty - 0 = \infty
\end{align*}
The limit is infinite.
\textbf{Answer:} DIVERGES.

\section{Problem 4}
Determine whether the integral is convergent or divergent. If it is convergent, evaluate it.
\[ \int_{8}^{\infty} \frac{1}{(x-7)^{3/2}} \,dx \]
\textbf{Solution:}
This is a Type 1 improper integral. Let $u = x-7$, so $du = dx$. When $x=8$, $u=1$. When $x \to \infty$, $u \to \infty$.
\begin{align*}
    \int_{8}^{\infty} \frac{1}{(x-7)^{3/2}} \,dx &= \int_{1}^{\infty} \frac{1}{u^{3/2}} \,du \\
    &= \lim_{t \to \infty} \int_{1}^{t} u^{-3/2} \,du \\
    &= \lim_{t \to \infty} \left[ \frac{u^{-1/2}}{-1/2} \right]_{1}^{t} = \lim_{t \to \infty} \left[ -2u^{-1/2} \right]_{1}^{t} \\
    &= \lim_{t \to \infty} \left[ -\frac{2}{\sqrt{u}} \right]_{1}^{t} \\
    &= \lim_{t \to \infty} \left( -\frac{2}{\sqrt{t}} - \left(-\frac{2}{\sqrt{1}}\right) \right) \\
    &= 0 + 2 = 2
\end{align*}
\textbf{Answer:} The integral is convergent and its value is 2.

\section{Problem 5}
Determine whether the integral is convergent or divergent. If it is convergent, evaluate it.
\[ \int_{0}^{\infty} \frac{1}{\sqrt{1+x}} \,dx \]
\textbf{Solution:}
This is a Type 1 improper integral. Let $u = 1+x$, so $du = dx$. When $x=0$, $u=1$. When $x \to \infty$, $u \to \infty$.
\begin{align*}
    \int_{0}^{\infty} \frac{1}{\sqrt{1+x}} \,dx &= \int_{1}^{\infty} \frac{1}{\sqrt{u}} \,du = \int_{1}^{\infty} u^{-1/2} \,du \\
    &= \lim_{t \to \infty} \int_{1}^{t} u^{-1/2} \,du \\
    &= \lim_{t \to \infty} \left[ \frac{u^{1/2}}{1/2} \right]_{1}^{t} = \lim_{t \to \infty} \left[ 2\sqrt{u} \right]_{1}^{t} \\
    &= \lim_{t \to \infty} (2\sqrt{t} - 2\sqrt{1}) \\
    &= \infty - 2 = \infty
\end{align*}
The limit is infinite.
\textbf{Answer:} DIVERGES.

\section{Problem 6}
Determine whether the integral is convergent or divergent. If it is convergent, evaluate it.
\[ \int_{-\infty}^{0} \frac{x}{(x^2+4)^2} \,dx \]
\textbf{Solution:}
This is a Type 1 improper integral. Use u-substitution with $u = x^2+4$, so $du = 2x\,dx$, or $x\,dx = \frac{1}{2}du$.
When $x \to -\infty$, $u \to \infty$. When $x=0$, $u=4$.
\begin{align*}
    \int_{-\infty}^{0} \frac{x}{(x^2+4)^2} \,dx &= \int_{\infty}^{4} \frac{1}{u^2} \cdot \frac{1}{2}du \\
    &= \frac{1}{2} \int_{\infty}^{4} u^{-2} \,du = -\frac{1}{2} \int_{4}^{\infty} u^{-2} \,du \\
    &= -\frac{1}{2} \lim_{t \to \infty} \int_{4}^{t} u^{-2} \,du \\
    &= -\frac{1}{2} \lim_{t \to \infty} \left[ \frac{u^{-1}}{-1} \right]_{4}^{t} = -\frac{1}{2} \lim_{t \to \infty} \left[ -\frac{1}{u} \right]_{4}^{t} \\
    &= -\frac{1}{2} \lim_{t \to \infty} \left( -\frac{1}{t} - \left(-\frac{1}{4}\right) \right) \\
    &= -\frac{1}{2} \left( 0 + \frac{1}{4} \right) = -\frac{1}{8}
\end{align*}
\textbf{Answer:} The integral is convergent and its value is -1/8.

\section{Problem 7}
Determine whether the integral is convergent or divergent. If it is convergent, evaluate it.
\[ \int_{1}^{\infty} \frac{x^3 + x + 1}{x^5} \,dx \]
\textbf{Solution:}
This is a Type 1 improper integral. First, simplify the integrand by splitting the fraction.
\[ \frac{x^3 + x + 1}{x^5} = \frac{x^3}{x^5} + \frac{x}{x^5} + \frac{1}{x^5} = \frac{1}{x^2} + \frac{1}{x^4} + \frac{1}{x^5} \]
\begin{align*}
    \int_{1}^{\infty} \left( x^{-2} + x^{-4} + x^{-5} \right) \,dx &= \lim_{t \to \infty} \int_{1}^{t} \left( x^{-2} + x^{-4} + x^{-5} \right) \,dx \\
    &= \lim_{t \to \infty} \left[ \frac{x^{-1}}{-1} + \frac{x^{-3}}{-3} + \frac{x^{-4}}{-4} \right]_{1}^{t} \\
    &= \lim_{t \to \infty} \left[ -\frac{1}{x} - \frac{1}{3x^3} - \frac{1}{4x^4} \right]_{1}^{t} \\
    &= \lim_{t \to \infty} \left( \left(-\frac{1}{t} - \frac{1}{3t^3} - \frac{1}{4t^4}\right) - \left(-1 - \frac{1}{3} - \frac{1}{4}\right) \right) \\
    &= (0 - 0 - 0) - \left(-\frac{12}{12} - \frac{4}{12} - \frac{3}{12}\right) \\
    &= - \left(-\frac{19}{12}\right) = \frac{19}{12}
\end{align*}
\textbf{Answer:} The integral is convergent and its value is 19/12.

\section{Problem 8}
Determine whether the integral is convergent or divergent. If it is convergent, evaluate it.
\[ \int_{0}^{\infty} \frac{e^x}{(8+e^x)^2} \,dx \]
\textbf{Solution:}
This is a Type 1 improper integral. Use u-substitution with $u = 8+e^x$, so $du = e^x \,dx$.
When $x=0$, $u=8+e^0 = 9$. When $x \to \infty$, $u \to \infty$.
\begin{align*}
    \int_{0}^{\infty} \frac{e^x}{(8+e^x)^2} \,dx &= \int_{9}^{\infty} \frac{1}{u^2} \,du \\
    &= \lim_{t \to \infty} \int_{9}^{t} u^{-2} \,du \\
    &= \lim_{t \to \infty} \left[ -\frac{1}{u} \right]_{9}^{t} \\
    &= \lim_{t \to \infty} \left( -\frac{1}{t} - \left(-\frac{1}{9}\right) \right) \\
    &= 0 + \frac{1}{9} = \frac{1}{9}
\end{align*}
\textbf{Answer:} The integral is convergent and its value is 1/9.

\section{Problem 9}
Determine whether the integral is convergent or divergent. If it is convergent, evaluate it.
\[ \int_{-\infty}^{\infty} 9xe^{-x^2} \,dx \]
\textbf{Solution:}
This is a Type 1 integral over $(-\infty, \infty)$, so we must split it into two integrals, typically at $x=0$.
\[ \int_{-\infty}^{\infty} 9xe^{-x^2} \,dx = \int_{-\infty}^{0} 9xe^{-x^2} \,dx + \int_{0}^{\infty} 9xe^{-x^2} \,dx \]
Let's evaluate the second integral first. Use u-substitution with $u = -x^2$, so $du = -2x\,dx$, or $x\,dx = -\frac{1}{2}du$.
When $x=0$, $u=0$. When $x \to \infty$, $u \to -\infty$.
\begin{align*}
    \int_{0}^{\infty} 9xe^{-x^2} \,dx &= \int_{0}^{-\infty} 9 e^u \left(-\frac{1}{2}du\right) \\
    &= -\frac{9}{2} \int_{0}^{-\infty} e^u \,du = \frac{9}{2} \int_{-\infty}^{0} e^u \,du \\
    &= \frac{9}{2} \lim_{t \to -\infty} \int_{t}^{0} e^u \,du = \frac{9}{2} \lim_{t \to -\infty} [e^u]_t^0 \\
    &= \frac{9}{2} \lim_{t \to -\infty} (e^0 - e^t) = \frac{9}{2}(1-0) = \frac{9}{2}
\end{align*}
For the first integral, we note the integrand $f(x) = 9xe^{-x^2}$ is an odd function since $f(-x) = 9(-x)e^{-(-x)^2} = -9xe^{-x^2} = -f(x)$. Since the integral from $0$ to $\infty$ converges, the integral from $-\infty$ to $0$ will also converge to the negative of that value.
\[ \int_{-\infty}^{0} 9xe^{-x^2} \,dx = -\frac{9}{2} \]
Since both integrals converge, the original integral converges.
\[ \int_{-\infty}^{\infty} 9xe^{-x^2} \,dx = -\frac{9}{2} + \frac{9}{2} = 0 \]
\textbf{Answer:} The integral is convergent and its value is 0.

\section{Problem 10}
Determine whether the integral is convergent or divergent. If it is convergent, evaluate it.
\[ \int_{-\infty}^{\infty} \frac{x^5}{x^6+1} \,dx \]
\textbf{Solution:}
This is a Type 1 integral over $(-\infty, \infty)$. We split it at $x=0$.
\[ \int_{-\infty}^{\infty} \frac{x^5}{x^6+1} \,dx = \int_{-\infty}^{0} \frac{x^5}{x^6+1} \,dx + \int_{0}^{\infty} \frac{x^5}{x^6+1} \,dx \]
The integrand is an odd function. If the integral converges, its value will be 0. Let's test the convergence of the second part. Let $u = x^6+1$, so $du=6x^5\,dx$, or $x^5\,dx = \frac{1}{6}du$. When $x=0, u=1$. When $x \to \infty, u \to \infty$.
\begin{align*}
    \int_{0}^{\infty} \frac{x^5}{x^6+1} \,dx &= \int_{1}^{\infty} \frac{1}{u} \cdot \frac{1}{6}du = \frac{1}{6} \int_{1}^{\infty} \frac{1}{u} \,du \\
    &= \frac{1}{6} \lim_{t \to \infty} \int_{1}^{t} \frac{1}{u} \,du \\
    &= \frac{1}{6} \lim_{t \to \infty} [\ln|u|]_1^t \\
    &= \frac{1}{6} \lim_{t \to \infty} (\ln(t) - \ln(1)) = \infty
\end{align*}
Since one of the integrals diverges, the original integral diverges.
\textbf{Answer:} DIVERGES.

\section{Problem 11}
Determine whether the integral is convergent or divergent. If it is convergent, evaluate it.
\[ \int_{0}^{\infty} 4\sin^2(\alpha) \,d\alpha \]
\textbf{Solution:}
This is a Type 1 improper integral. Use the identity $\sin^2(\alpha) = \frac{1 - \cos(2\alpha)}{2}$.
\begin{align*}
    \int_{0}^{\infty} 4\sin^2(\alpha) \,d\alpha &= \lim_{t \to \infty} \int_{0}^{t} 4 \left( \frac{1 - \cos(2\alpha)}{2} \right) \,d\alpha \\
    &= \lim_{t \to \infty} \int_{0}^{t} 2(1 - \cos(2\alpha)) \,d\alpha \\
    &= \lim_{t \to \infty} 2 \left[ \alpha - \frac{1}{2}\sin(2\alpha) \right]_{0}^{t} \\
    &= \lim_{t \to \infty} 2 \left( \left(t - \frac{1}{2}\sin(2t)\right) - (0 - 0) \right) \\
    &= \lim_{t \to \infty} (2t - \sin(2t))
\end{align*}
As $t \to \infty$, $2t \to \infty$ and $\sin(2t)$ oscillates between -1 and 1. The overall limit goes to $\infty$.
\textbf{Answer:} DIVERGES.

\section{Problem 12}
Part 1: Evaluate the integral $\int_{0}^{t} 8\sin^2(\alpha) \,d\alpha$.
Part 2: Determine if $\int_{0}^{\infty} 8\sin^2(\alpha) \,d\alpha$ is convergent or divergent.
\textbf{Solution Part 1:}
\begin{align*}
    \int_{0}^{t} 8\sin^2(\alpha) \,d\alpha &= \int_{0}^{t} 8 \left( \frac{1 - \cos(2\alpha)}{2} \right) \,d\alpha \\
    &= \int_{0}^{t} 4(1 - \cos(2\alpha)) \,d\alpha \\
    &= 4 \left[ \alpha - \frac{1}{2}\sin(2\alpha) \right]_{0}^{t} \\
    &= 4 \left( \left(t - \frac{1}{2}\sin(2t)\right) - (0 - 0) \right) \\
    &= 4t - 2\sin(2t)
\end{align*}
\textbf{Answer Part 1:} $4t - 2\sin(2t)$.
\textbf{Solution Part 2:}
To determine if the improper integral converges, we take the limit of the result from Part 1 as $t \to \infty$.
\[ \lim_{t \to \infty} (4t - 2\sin(2t)) = \infty \]
Since the limit is infinite, the integral diverges.
\textbf{Answer Part 2:} DIVERGES.

\section{Problem 13}
Determine whether the integral is convergent or divergent. If it is convergent, evaluate it.
\[ \int_{0}^{\infty} \sin(\theta) e^{\cos(\theta)} \,d\theta \]
\textbf{Solution:}
This is a Type 1 improper integral. Let $u = \cos(\theta)$, so $du = -\sin(\theta)\,d\theta$.
\begin{align*}
    \int_{0}^{\infty} \sin(\theta) e^{\cos(\theta)} \,d\theta &= \lim_{t \to \infty} \int_{0}^{t} \sin(\theta) e^{\cos(\theta)} \,d\theta \\
    &= \lim_{t \to \infty} \int_{\cos(0)}^{\cos(t)} -e^u \,du \\
    &= \lim_{t \to \infty} [-e^u]_{\cos(0)}^{\cos(t)} = \lim_{t \to \infty} [-e^u]_{1}^{\cos(t)} \\
    &= \lim_{t \to \infty} (-e^{\cos(t)} - (-e^1)) = \lim_{t \to \infty} (e - e^{\cos(t)})
\end{align*}
As $t \to \infty$, $\cos(t)$ oscillates between -1 and 1. Therefore, $e^{\cos(t)}$ oscillates between $e^{-1}$ and $e^1$. The limit does not exist.
\textbf{Answer:} DIVERGES.

\section{Problem 14}
Determine whether the integral is convergent or divergent. If it is convergent, evaluate it.
\[ \int_{7}^{\infty} \frac{1}{x^2+x} \,dx = \int_{7}^{\infty} \frac{1}{x(x+1)} \,dx \]
\textbf{Solution:}
Use partial fraction decomposition. $\frac{1}{x(x+1)} = \frac{A}{x} + \frac{B}{x+1}$.
$1 = A(x+1) + Bx$. If $x=0$, $A=1$. If $x=-1$, $B=-1$. So $\frac{1}{x(x+1)} = \frac{1}{x} - \frac{1}{x+1}$.
\begin{align*}
    \int_{7}^{\infty} \left(\frac{1}{x} - \frac{1}{x+1}\right) \,dx &= \lim_{t \to \infty} \int_{7}^{t} \left(\frac{1}{x} - \frac{1}{x+1}\right) \,dx \\
    &= \lim_{t \to \infty} [\ln|x| - \ln|x+1|]_7^t = \lim_{t \to \infty} \left[\ln\left|\frac{x}{x+1}\right|\right]_7^t \\
    &= \lim_{t \to \infty} \left(\ln\left(\frac{t}{t+1}\right) - \ln\left(\frac{7}{8}\right)\right)
\end{align*}
The limit of the fraction is $\lim_{t \to \infty} \frac{t}{t+1} = 1$.
\[ = \ln(1) - \ln(7/8) = 0 - \ln(7/8) = \ln((7/8)^{-1}) = \ln(8/7) \]
\textbf{Answer:} The integral is convergent and its value is $\ln(8/7)$.

\section{Problem 15}
Determine whether the integral is convergent or divergent. If it is convergent, evaluate it.
\[ \int_{2}^{\infty} \frac{dv}{v^2+2v-3} \]
\textbf{Solution:}
Factor the denominator: $v^2+2v-3 = (v+3)(v-1)$. Use partial fractions: $\frac{1}{(v+3)(v-1)} = \frac{A}{v+3} + \frac{B}{v-1}$.
$1 = A(v-1) + B(v+3)$. If $v=1$, $1=4B \Rightarrow B=1/4$. If $v=-3$, $1=-4A \Rightarrow A=-1/4$.
\begin{align*}
    \int_{2}^{\infty} \frac{1/4}{v-1} - \frac{1/4}{v+3} \,dv &= \frac{1}{4} \lim_{t \to \infty} \int_{2}^{t} \left(\frac{1}{v-1} - \frac{1}{v+3}\right) \,dv \\
    &= \frac{1}{4} \lim_{t \to \infty} [\ln|v-1| - \ln|v+3|]_2^t \\
    &= \frac{1}{4} \lim_{t \to \infty} \left[\ln\left|\frac{v-1}{v+3}\right|\right]_2^t \\
    &= \frac{1}{4} \lim_{t \to \infty} \left(\ln\left(\frac{t-1}{t+3}\right) - \ln\left(\frac{1}{5}\right)\right)
\end{align*}
The limit of the fraction is $\lim_{t \to \infty} \frac{t-1}{t+3} = 1$.
\[ = \frac{1}{4}(\ln(1) - \ln(1/5)) = \frac{1}{4}(0 - (-\ln 5)) = \frac{\ln 5}{4} \]
\textbf{Answer:} The integral is convergent and its value is $\frac{\ln 5}{4}$.

\section{Problem 16}
Determine whether the integral is convergent or divergent. If it is convergent, evaluate it.
\[ \int_{-\infty}^{0} \frac{z}{z^4+81} \,dz \]
\textbf{Solution:}
Let $u=z^2$, so $du = 2z \,dz$, or $z\,dz = \frac{1}{2}du$. When $z \to -\infty$, $u \to \infty$. When $z=0$, $u=0$.
\begin{align*}
    \int_{-\infty}^{0} \frac{z}{z^4+81} \,dz &= \int_{\infty}^{0} \frac{1}{u^2+81} \cdot \frac{1}{2}du \\
    &= -\frac{1}{2} \int_{0}^{\infty} \frac{1}{u^2+9^2} \,du \\
    &= -\frac{1}{2} \lim_{t \to \infty} \int_{0}^{t} \frac{1}{u^2+9^2} \,du \\
    &= -\frac{1}{2} \lim_{t \to \infty} \left[ \frac{1}{9}\arctan\left(\frac{u}{9}\right) \right]_0^t \\
    &= -\frac{1}{18} \lim_{t \to \infty} \left( \arctan\left(\frac{t}{9}\right) - \arctan(0) \right) \\
    &= -\frac{1}{18} \left( \frac{\pi}{2} - 0 \right) = -\frac{\pi}{36}
\end{align*}
\textbf{Answer:} The integral is convergent and its value is $-\frac{\pi}{36}$.

\section{Problem 17}
Part 1: Evaluate $\int_{t}^{0} \frac{z}{z^4+36} \,dz$.
Part 2: Determine if $\int_{-\infty}^{0} \frac{z}{z^4+36} \,dz$ converges.
\textbf{Solution Part 1:}
Let $u=z^2$, $du=2z\,dz$. When $z=t, u=t^2$. When $z=0, u=0$.
\begin{align*}
    \int_{t}^{0} \frac{z}{(z^2)^2+36} \,dz &= \int_{t^2}^{0} \frac{1/2}{u^2+6^2} \,du \\
    &= \frac{1}{2} \left[ \frac{1}{6}\arctan\left(\frac{u}{6}\right) \right]_{t^2}^0 \\
    &= \frac{1}{12} \left( \arctan(0) - \arctan\left(\frac{t^2}{6}\right) \right) \\
    &= -\frac{1}{12}\arctan\left(\frac{t^2}{6}\right)
\end{align*}
\textbf{Answer Part 1:} $-\frac{1}{12}\arctan\left(\frac{t^2}{6}\right)$.
\textbf{Solution Part 2:}
We take the limit of the Part 1 result as $t \to -\infty$.
\begin{align*}
    \lim_{t \to -\infty} \left( -\frac{1}{12}\arctan\left(\frac{t^2}{6}\right) \right)
\end{align*}
As $t \to -\infty$, $t^2 \to \infty$, so $\frac{t^2}{6} \to \infty$.
\[ = -\frac{1}{12} \lim_{s \to \infty} \arctan(s) = -\frac{1}{12} \cdot \frac{\pi}{2} = -\frac{\pi}{24} \]
\textbf{Answer Part 2:} The integral converges to $-\frac{\pi}{24}$.

\section{Problem 18}
Determine whether the integral is convergent or divergent. If it is convergent, evaluate it.
\[ \int_{0}^{9} \frac{3}{\sqrt{x-1}} \,dx \]
\textbf{Solution:}
This is a Type 2 improper integral because the integrand has an infinite discontinuity at $x=1$, which is inside the interval $[0, 9]$. We must split the integral at the point of discontinuity.
\[ \int_{0}^{9} \frac{3}{\sqrt{x-1}} \,dx = \int_{0}^{1} \frac{3}{\sqrt{x-1}} \,dx + \int_{1}^{9} \frac{3}{\sqrt{x-1}} \,dx \]
Let's evaluate the first integral:
\begin{align*}
    \int_{0}^{1} 3(x-1)^{-1/2} \,dx &= \lim_{t \to 1^{-}} \int_{0}^{t} 3(x-1)^{-1/2} \,dx \\
    &= \lim_{t \to 1^{-}} \left[ 3 \frac{(x-1)^{1/2}}{1/2} \right]_{0}^{t} = \lim_{t \to 1^{-}} [6\sqrt{x-1}]_0^t \\
    &= \lim_{t \to 1^{-}} (6\sqrt{t-1} - 6\sqrt{-1})
\end{align*}
The term $\sqrt{-1}$ is not a real number. The function is not defined on the interval $[0, 1)$. This implies there may be a typo in the problem as stated, and perhaps the lower bound was meant to be greater than 1. Assuming the problem is as written, the integral is not well-defined over the real numbers.
However, if the question was intended to be, for example, $\int_{1}^{9} \frac{3}{\sqrt{x-1}} \,dx$:
\begin{align*}
    \int_{1}^{9} 3(x-1)^{-1/2} \,dx &= \lim_{t \to 1^{+}} \int_{t}^{9} 3(x-1)^{-1/2} \,dx \\
    &= \lim_{t \to 1^{+}} [6\sqrt{x-1}]_t^9 \\
    &= \lim_{t \to 1^{+}} (6\sqrt{9-1} - 6\sqrt{t-1}) \\
    &= 6\sqrt{8} - 6\sqrt{0} = 6(2\sqrt{2}) = 12\sqrt{2}
\end{align*}
If the intended problem was $\int_{2}^{9} \frac{3}{\sqrt{x-1}} \,dx$, it would be a proper integral. Given the context of improper integrals, it's most likely the domain was intended to be something like $[1, 9]$ or there's an error in the problem statement. If we assume the question meant to integrate where the function is real, starting from the discontinuity, the value is $12\sqrt{2}$. But as written, the integral diverges because the function is not real on $[0,1)$.
\textbf{Answer:} DIVERGES (as written).

\part{In-Depth Analysis of Problems and Techniques}
\subsection{Problem Types and General Approach}
\begin{itemize}
    \item \textbf{Direct p-Integrals (or reducible to them):} Problems 1, 2, 4, 5, 8. These are the simplest type. The general approach is to write the integrand as $1/x^p$ (or $1/u^p$ after a simple substitution), identify $p$, and apply the p-test rule ($p>1$ converges). The formal solution still requires writing out the limit definition.
    \item \textbf{Integrals Solvable with U-Substitution:} Problems 4, 5, 6, 8, 9, 13, 16, 17. The core strategy is to identify a composite function structure, where a part of the integrand is the derivative of an "inner" function. After substitution, these often transform into much simpler integrals, like p-integrals or standard forms ($\int e^u du$, $\int \frac{1}{u^2+a^2} du$).
    \item \textbf{Integrals Requiring Algebraic Simplification:} Problem 7. The strategy is to recognize that a complex rational function can be simplified by dividing each term in the numerator by the denominator. This turns one difficult integral into a sum of several easy p-integrals.
    \item \textbf{Integrals Requiring Partial Fractions:} Problems 14, 15. When faced with a rational function whose denominator can be factored, the go-to strategy is partial fraction decomposition. This breaks the complex fraction into a sum of simpler fractions, whose antiderivatives are typically logarithms.
    \item \textbf{Integrals with Oscillating Functions:} Problems 11, 12, 13. When the integrand contains a trigonometric function like $\sin(x)$ or $\cos(x)$ that does not approach zero as $x \to \infty$, divergence is highly likely. The area continues to accumulate (like in \#11) or the limit fails to exist because of oscillation (like in \#13).
    \item \textbf{Integrals over $(-\infty, \infty)$:} Problems 9, 10. The unbreakable rule is to split the integral into two pieces, $\int_{-\infty}^c$ and $\int_c^\infty$. A common mistake is to not do this. Checking for odd/even symmetry is a useful shortcut: for an odd function, the integral is 0 if it converges; for an even function, the total is twice the integral from 0 to $\infty$.
    \item \textbf{Type 2 Improper Integrals (Discontinuity):} Problem 18. The critical first step is to identify the location of the vertical asymptote. If it's within the interval, you must split the integral at that point and use one-sided limits. Forgetting to check for discontinuities is a frequent error.
\end{itemize}

\subsection{Key Algebraic and Calculus Manipulations}
\begin{itemize}
    \item \textbf{Limit Definition:} The fundamental technique used in every problem. It formally converts an improper integral into a standard definite integral and a limit.
    \item \textbf{U-Substitution:} In Problem 16, $\int \frac{z}{z^4+81} dz$, the substitution $u=z^2$ was crucial. It transformed a non-obvious integral into the standard arctangent form $\int \frac{1}{u^2+a^2} du$. This highlights how substitution can change the entire structure of a problem.
    \item \textbf{Partial Fraction Decomposition:} In Problem 15, integrating $\frac{1}{v^2+2v-3}$ is impossible directly. Decomposing it into $\frac{1/4}{v-1} - \frac{1/4}{v+3}$ was necessary to reveal the logarithmic antiderivatives.
    \item \textbf{Logarithm Properties:} In Problems 14 and 15, the property $\ln(A) - \ln(B) = \ln(A/B)$ was essential for evaluating the final limit. Without it, you would face the indeterminate form $\infty - \infty$.
    \item \textbf{Trigonometric Power-Reducing Identity:} In Problems 11 and 12, it's not obvious how to integrate $\sin^2(\alpha)$. The identity $\sin^2(\alpha) = \frac{1-\cos(2\alpha)}{2}$ was necessary to convert the integrand into a form whose antiderivative is known.
    \item \textbf{Splitting Integrals:} In Problem 9, the integral $\int_{-\infty}^{\infty}$ had to be split into two separate integrals, $\int_{-\infty}^0$ and $\int_0^\infty$. This is not optional; it is the definition of such integrals, and failure to do so is a conceptual error. The same applies to Problem 18 at the discontinuity $x=1$.
\end{itemize}

\part{"Cheatsheet" and Tips for Success}
\subsection{Summary of Formulas}
\begin{itemize}
    \item \textbf{Type 1:} $\int_{a}^{\infty} f(x) \,dx = \lim_{t \to \infty} \int_{a}^{t} f(x) \,dx$
    \item \textbf{Type 2:} $\int_{a}^{b} f(x) \,dx = \lim_{t \to c^{-}} \int_{a}^{t} f(x) \,dx + \lim_{s \to c^{+}} \int_{s}^{b} f(x) \,dx$ (for discontinuity at $c$)
    \item \textbf{p-Test:} $\int_{1}^{\infty} \frac{1}{x^p} \,dx$ converges if $p > 1$, diverges if $p \le 1$.
\end{itemize}

\subsection{Tricks and Shortcuts}
\begin{itemize}
    \item \textbf{Look for p-Integrals in Disguise:} An integral like $\int_2^\infty \frac{dx}{(x-1)^3}$ is just a p-integral with $p=3$ shifted by 1. It converges.
    \item \textbf{Check for Obvious Divergence:} If $\lim_{x \to \infty} f(x) \neq 0$, then $\int_a^\infty f(x) \,dx$ must diverge. The area of the "slices" isn't going to zero, so the total sum must be infinite. This applies to Problem 11, where $\sin^2(\alpha)$ does not go to zero.
    \item \textbf{Symmetry for $(-\infty, \infty)$:} If the integrand is \textbf{odd} ($f(-x) = -f(x)$), the integral is 0 IF it converges (see \#9 vs \#10). If the integrand is \textbf{even} ($f(-x)=f(x)$), the integral is $2\int_0^\infty f(x) dx$.
\end{itemize}

\subsection{Common Pitfalls and Mistakes}
\begin{itemize}
    \item \textbf{Forgetting the Limit:} Do not just plug in `$\infty$` as a number. Always write `$\lim_{t \to \infty}$`.
    \item \textbf{Ignoring Internal Discontinuities:} Always check if the integrand has a vertical asymptote *inside* the integration interval (e.g., $\int_{-1}^1 \frac{1}{x} dx$). This is a Type 2 integral that must be split.
    \item \textbf{Errors in Evaluating Limits:} Be careful with limits involving $\ln(x)$, $e^x$, and $\arctan(x)$. Remember that $\lim_{x\to\infty} \arctan(x) = \pi/2$.
    \item \textbf{Incorrectly Splitting Integrals:} For $\int_{-\infty}^\infty$, you must show that BOTH $\int_{-\infty}^c$ and $\int_c^\infty$ converge. If one diverges, the whole thing diverges. You cannot assume they cancel.
\end{itemize}

\part{Conceptual Synthesis and The "Big Picture"}

\subsection{Thematic Connections}
The core theme of this topic is **taming the infinite through the process of limits**. We are taking a concept designed for finite things—area—and extending its definition to apply to infinite regions. This same fundamental theme is the cornerstone of several other major calculus topics:
\begin{itemize}
    \item \textbf{Infinite Series:} An infinite series is a sum of infinitely many terms. To determine if this sum is finite (converges), we take the limit of its finite partial sums. This is the discrete analogue of the improper integral, where we take the limit of finite "partial integrals" (areas).
    \item \textbf{Limits at Infinity (Derivatives):} In differential calculus, we used limits at infinity to find horizontal asymptotes, which describe the behavior of a function at the "ends" of the number line. Improper integrals take this one step further by asking for the total *accumulated value* of the function over those infinite ends.
\end{itemize}

\subsection{Forward and Backward Links}
\begin{itemize}
    \item \textbf{Backward Links:} The entire machinery of improper integrals is built upon two pillars from earlier mathematics: the \textbf{definite integral} (the Fundamental Theorem of Calculus) and the concept of a \textbf{limit}. Without a robust understanding of how to find antiderivatives and how to evaluate limits at infinity or near a point of discontinuity, this topic is inaccessible. It is a true synthesis of the two major ideas of Calculus I.
    \item \textbf{Forward Links:} The skills developed here are not an endpoint; they are a crucial foundation for more advanced topics.
    \begin{itemize}
        \item \textbf{Infinite Series:} The \textbf{Integral Test} is a key method for determining the convergence of a series by comparing it to an improper integral.
        \item \textbf{Probability \& Statistics:} A continuous \textbf{probability density function} (PDF), like the famous bell curve for the normal distribution, is defined over $(-\infty, \infty)$. A core property of any PDF $f(x)$ is that $\int_{-\infty}^\infty f(x) dx = 1$, a direct application of an improper integral.
        \item \textbf{Differential Equations \& Transforms:} In fields like signal processing and control theory, integral transforms like the Laplace Transform and Fourier Transform are defined using improper integrals. For example, the Laplace Transform of a function $f(t)$ is $\mathcal{L}\{f(t)\} = \int_0^\infty f(t)e^{-st} dt$.
    \end{itemize}
\end{itemize}

\part{Real-World Application and Modeling}
\subsection{Concrete Scenarios in Finance and Economics}
\begin{enumerate}
    \item \textbf{Valuing Perpetual Bonds and Stocks (Finance):} Some financial instruments, like certain preferred stocks or government bonds (consols), promise to pay a dividend or coupon forever. This is called a perpetuity. To find the price of such a security, one must calculate the present value of this infinite stream of future payments. Using a continuous model, this price is given by the improper integral $PV = \int_0^\infty C(t)e^{-rt} dt$, where $C(t)$ is the payment at time $t$ and $r$ is the continuous discount rate. The integral must converge for the asset to have a finite price.

    \item \textbf{Option Pricing Models (Financial Engineering):} The famous Black-Scholes model for pricing stock options relies on the log-normal distribution of stock prices. The expected payoff of an option is calculated by integrating the payoff function against the probability density function of the stock price, often over an infinite or semi-infinite range $([0, \infty))$. This requires the evaluation of improper integrals to find a fair price for the derivative security.

    \item \textbf{Consumer Surplus with Unbounded Demand (Economics):} Economic models often use demand curves that are asymptotic, meaning the quantity demanded approaches zero as price approaches infinity, but never reaches it. To calculate the total consumer surplus—the total benefit consumers get from paying less than the maximum they would be willing to pay—an economist might need to evaluate an improper integral of the demand curve from the current quantity sold out to infinity.
\end{enumerate}

\subsection{Model Problem Setup: Present Value of a Perpetuity}
\begin{itemize}
    \item \textbf{Scenario:} A corporation establishes a foundation that is projected to generate a steady, continuous stream of profit of \$500,000 per year, which will be used for charitable donations. We want to find the total "capital value" of this foundation, which is the present value of all its future earnings, assuming a constant interest rate of 4\% compounded continuously.
    \item \textbf{Model Setup:}
    \begin{itemize}
        \item \textbf{Variables:}
        \begin{itemize}
            \item $C$: The continuous cash flow rate, $C = \$500,000$ per year.
            \item $r$: The continuous discount (interest) rate, $r = 0.04$.
            \item $t$: Time in years, starting from $t=0$.
        \end{itemize}
        \item \textbf{Function Formulation:} The value of a future payment is discounted to the present. A payment at a future time $t$ is worth $C e^{-rt}$ in today's money. The function representing the present value of the cash flow at any given time $t$ is $f(t) = 500000 e^{-0.04t}$.
        \item \textbf{Integral Statement:} To find the total present value (the capital value), we must sum (integrate) the present values of all future cash flows from now ($t=0$) to forever ($t=\infty$). This gives the improper integral:
        \[ \text{Capital Value} = \int_{0}^{\infty} 500000 e^{-0.04t} \,dt \]
        Solving this integral would give the single lump-sum amount of money that, if invested today at 4\%, would generate the same \$500,000 stream of income forever.
    \end{itemize}
\end{itemize}

\part{Common Variations and Untested Concepts}
The provided homework assignment focuses exclusively on direct computation of improper integrals. It omits a critical and very common technique: determining convergence without explicit evaluation.

\subsection{The Comparison Test}
\begin{itemize}
    \item \textbf{Explanation:} This test is used to determine convergence or divergence by comparing a difficult integral to a simpler one whose behavior is known (like a p-integral). If $f(x)$ and $g(x)$ are continuous functions such that $0 \le f(x) \le g(x)$ for all $x \ge a$:
    \begin{enumerate}
        \item If the \textbf{larger} integral $\int_a^\infty g(x) dx$ \textbf{converges}, then the \textbf{smaller} integral $\int_a^\infty f(x) dx$ must also \textbf{converge}.
        \item If the \textbf{smaller} integral $\int_a^\infty f(x) dx$ \textbf{diverges}, then the \textbf{larger} integral $\int_a^\infty g(x) dx$ must also \textbf{diverge}.
    \end{enumerate}
    \item \textbf{Worked Example (Not in Homework):} Determine if $\int_1^\infty \frac{\sin^2(x)}{x^3} dx$ converges.
    \begin{itemize}
        \item \textbf{Analysis:} Integrating this function directly is extremely difficult. However, we know that $0 \le \sin^2(x) \le 1$ for all $x$.
        \item \textbf{Comparison:} If we divide by $x^3$ (which is positive for $x \ge 1$), we get the inequality: $0 \le \frac{\sin^2(x)}{x^3} \le \frac{1}{x^3}$.
        \item \textbf{Conclusion:} We can now examine the simpler, larger integral: $\int_1^\infty \frac{1}{x^3} dx$. This is a p-integral with $p=3$. Since $p>1$, it converges. Because our original, smaller integral is squeezed between 0 and a convergent integral, it must also converge by the Comparison Test.
    \end{itemize}
\end{itemize}

\subsection{The Limit Comparison Test}
\begin{itemize}
    \item \textbf{Explanation:} This is often easier to apply than the standard Comparison Test. If $f(x)$ and $g(x)$ are positive functions for $x \ge a$, we evaluate the limit $L = \lim_{x \to \infty} \frac{f(x)}{g(x)}$.
    \begin{itemize}
        \item If $L$ is a finite, positive number ($0 < L < \infty$), then $\int f(x) dx$ and $\int g(x) dx$ do the same thing: they either both converge or both diverge.
    \end{itemize}
    \item \textbf{Worked Example (Not in Homework):} Determine if $\int_2^\infty \frac{x+5}{x^3 - 2x + 1} dx$ converges.
    \begin{itemize}
        \item \textbf{Analysis:} The integrand $f(x) = \frac{x+5}{x^3 - 2x + 1}$ is complicated. However, for very large $x$, the highest powers dominate. So, $f(x)$ should behave like $\frac{x}{x^3} = \frac{1}{x^2}$.
        \item \textbf{Comparison:} Let's compare $f(x)$ to the simpler function $g(x) = \frac{1}{x^2}$.
        \item \textbf{Limit Evaluation:}
        \[ L = \lim_{x \to \infty} \frac{\frac{x+5}{x^3 - 2x + 1}}{\frac{1}{x^2}} = \lim_{x \to \infty} \frac{x^2(x+5)}{x^3 - 2x + 1} = \lim_{x \to \infty} \frac{x^3+5x^2}{x^3 - 2x + 1} = 1 \]
        \item \textbf{Conclusion:} Since $L=1$ (a finite, positive number), our integral behaves just like $\int_2^\infty \frac{1}{x^2} dx$. This is a convergent p-integral ($p=2>1$), so our original integral must also converge by the Limit Comparison Test.
    \end{itemize}
\end{itemize}


\part{Advanced Diagnostic Testing: "Find the Flaw"}

\section{Problem 1}
Evaluate $\displaystyle \int_{-\infty}^{\infty} \frac{2x}{x^4 + 1} \,dx$.

\textbf{Flawed Solution:}
\begin{align*}
    \text{Let } f(x) &= \frac{2x}{x^4 + 1}. \\
    \text{Check for symmetry: } f(-x) &= \frac{2(-x)}{(-x)^4 + 1} = \frac{-2x}{x^4 + 1} = -f(x). \\
    \text{The function is odd.} \\
    \text{Therefore, } \int_{-\infty}^{\infty} \frac{2x}{x^4 + 1} \,dx &= 0.
\end{align*}
\textbf{Your Task:} Find the flaw in the reasoning, explain it in one sentence, and provide the correct solution.

\section{Problem 2}
Evaluate $\displaystyle \int_{0}^{5} \frac{1}{(x-3)^2} \,dx$.

\textbf{Flawed Solution:}
\begin{align*}
    \int_{0}^{5} (x-3)^{-2} \,dx &= \left[ \frac{(x-3)^{-1}}{-1} \right]_{0}^{5} \\
    &= \left[ -\frac{1}{x-3} \right]_{0}^{5} \\
    &= \left(-\frac{1}{5-3}\right) - \left(-\frac{1}{0-3}\right) \\
    &= -\frac{1}{2} - \frac{1}{3} = -\frac{5}{6}.
\end{align*}
\textbf{Your Task:} Find the flaw in the reasoning, explain it in one sentence, and provide the correct solution.

\section{Problem 3}
Evaluate $\displaystyle \int_{e}^{\infty} \frac{\ln(x)}{x} \,dx$.

\textbf{Flawed Solution:}
\begin{align*}
    \int_{e}^{\infty} \frac{\ln(x)}{x} \,dx &= \lim_{t \to \infty} \int_{e}^{t} \frac{\ln(x)}{x} \,dx. \\
    \text{Let } u = \ln(x), \text{ so } du &= \frac{1}{x} \,dx. \\
    &= \lim_{t \to \infty} \int_{\ln(e)}^{\ln(t)} u \,du = \lim_{t \to \infty} \int_{1}^{\ln(t)} u \,du \\
    &= \lim_{t \to \infty} \left[ \frac{u^2}{2} \right]_{1}^{\ln(t)} \\
    &= \lim_{t \to \infty} \left( \frac{(\ln(t))^2}{2} - \frac{1^2}{2} \right) \\
    &= \lim_{t \to \infty} \left( \frac{(\ln(\infty))^2}{2} - \frac{1}{2} \right) = \frac{0}{2} - \frac{1}{2} = -\frac{1}{2}.
\end{align*}
\textbf{Your Task:} Find the flaw in the reasoning, explain it in one sentence, and provide the correct solution.

\section{Problem 4}
Evaluate $\displaystyle \int_{0}^{\infty} \cos(x) \,dx$.

\textbf{Flawed Solution:}
\begin{align*}
    \int_{0}^{\infty} \cos(x) \,dx &= \lim_{t \to \infty} \int_{0}^{t} \cos(x) \,dx \\
    &= \lim_{t \to \infty} [\sin(x)]_{0}^{t} \\
    &= \lim_{t \to \infty} (\sin(t) - \sin(0)) \\
    &= \lim_{t \to \infty} \sin(t) \\
    \text{The sine function oscillates between -1 and 1, so the average value is 0.} \\
    \text{Therefore, the integral is 0.}
\end{align*}
\textbf{Your Task:} Find the flaw in the reasoning, explain it in one sentence, and provide the correct solution.

\section{Problem 5}
Evaluate $\displaystyle \int_{0}^{\infty} \frac{1}{x^2 + x} \,dx$.

\textbf{Flawed Solution:}
\begin{align*}
    \int_{0}^{\infty} \frac{1}{x(x+1)} \,dx &= \int_{0}^{\infty} \left( \frac{1}{x} - \frac{1}{x+1} \right) \,dx \quad \text{(by partial fractions)} \\
    &= \lim_{t \to \infty} \int_{0}^{t} \left( \frac{1}{x} - \frac{1}{x+1} \right) \,dx \\
    &= \lim_{t \to \infty} [\ln|x| - \ln|x+1|]_{0}^{t} \\
    &= \lim_{t \to \infty} \left( (\ln|t| - \ln|t+1|) - (\ln|0| - \ln|1|) \right) \\
    &= \lim_{t \to \infty} \left( \ln\left|\frac{t}{t+1}\right| - (-\infty - 0) \right) \\
    &= \ln(1) + \infty = \infty. \text{ So it diverges.}
\end{align*}
\textbf{Your Task:} Find the flaw in the reasoning, explain it in one sentence, and provide the correct solution.


\end{document}