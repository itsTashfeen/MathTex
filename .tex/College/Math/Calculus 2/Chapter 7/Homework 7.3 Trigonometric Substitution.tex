\documentclass{article}
\usepackage{graphicx} % Required for inserting images
\usepackage{amsmath}  % Required for advanced math environments like align*
\usepackage{amssymb}  % For more math symbols

% --- GEOMETRY PACKAGE FOR PAGE LAYOUT ---
% This is the new part that controls the margins.
% It sets a 1-inch top/bottom/left margin and a text width of 4.5 inches.
% This will create a large empty margin on the right side of the page.
\usepackage[
    left=1in,
    textwidth=7in, 
    top=1in,
    bottom=1in
]{geometry}
% -----------------------------------------

\title{Homework 7.3 Trigonometric Substitution}
\author{Tashfeen Omran}
\date{September 2025}

\begin{document}

\maketitle

\section{Integration Problems and Solutions}

%---------------------------------------------------------------
\subsection{Problem 1}
Evaluate the integral: $ \int \frac{x}{\sqrt{81 + x^2}} \,dx $
\subsubsection*{Solution}
This integral is solved using u-substitution.
Let $ u = 81 + x^2 $.
Then $ du = 2x \,dx $, which implies $ x \,dx = \frac{du}{2} $.
Substituting these into the integral gives:
\[ \int \frac{1}{\sqrt{u}} \frac{du}{2} = \frac{1}{2} \int u^{-1/2} \,du \]
Using the power rule for integration:
\[ \frac{1}{2} \left[ \frac{u^{1/2}}{1/2} \right] + C = u^{1/2} + C \]
Substituting back for u:
\textbf{Answer:} $ \sqrt{81 + x^2} + C $


%---------------------------------------------------------------
\subsection{Problem 2}
Evaluate the integral: $ \int \frac{x}{\sqrt{x^2 - 5}} \,dx $
\subsubsection*{Solution}
This is also solved with a u-substitution.
Let $ u = x^2 - 5 $.
Then $ du = 2x \,dx $, so $ x \,dx = \frac{du}{2} $.
The integral becomes:
\[ \int \frac{1}{\sqrt{u}} \frac{du}{2} = \frac{1}{2} \int u^{-1/2} \,du \]
Integrating gives:
\[ \frac{1}{2} \left[ \frac{u^{1/2}}{1/2} \right] + C = u^{1/2} + C \]
Substituting back for u:
\textbf{Answer:} $ \sqrt{x^2 - 5} + C $


%---------------------------------------------------------------
\subsection{Problem 3}
Evaluate the integral: $ \int_{0}^{3} \sqrt{x^2 + 9} \,dx $
\subsubsection*{Solution}
This integral requires trigonometric substitution. Let $ x = 3 \tan(\theta) $, so $ dx = 3 \sec^2(\theta) \,d\theta $.
The expression $ \sqrt{x^2 + 9} $ becomes $ \sqrt{9 \tan^2(\theta) + 9} = 3 \sec(\theta) $.
Change the limits of integration:
\begin{itemize}
    \item When $ x = 0, \tan(\theta) = 0 \implies \theta = 0 $.
    \item When $ x = 3, \tan(\theta) = 1 \implies \theta = \frac{\pi}{4} $.
\end{itemize}
The integral transforms to:
\[ \int_{0}^{\pi/4} (3 \sec(\theta)) (3 \sec^2(\theta) \,d\theta) = 9 \int_{0}^{\pi/4} \sec^3(\theta) \,d\theta \]
Using the standard integral of $ \sec^3(\theta) $:
\begin{align*}
& 9 \left[ \frac{1}{2}(\sec(\theta)\tan(\theta) + \ln|\sec(\theta) + \tan(\theta)|) \right]_{0}^{\pi/4} \\
=& \frac{9}{2} [(\sec(\tfrac{\pi}{4})\tan(\tfrac{\pi}{4}) + \ln|\sec(\tfrac{\pi}{4}) + \tan(\tfrac{\pi}{4})|) - (\sec(0)\tan(0) + \ln|\sec(0) + \tan(0)|)] \\
=& \frac{9}{2} [(\sqrt{2} \cdot 1 + \ln(\sqrt{2} + 1)) - (1 \cdot 0 + \ln(1))]
\end{align*}
\textbf{Answer:} $ \frac{9}{2}(\sqrt{2} + \ln(1 + \sqrt{2})) $


%---------------------------------------------------------------
\subsection{Problem 4}
Evaluate $ \int \frac{x^3}{\sqrt{16 + x^2}} \,dx $ using $ x = 4 \tan(\theta) $.
\subsubsection*{Solution}
Let $ x = 4 \tan(\theta) $, so $ dx = 4 \sec^2(\theta) \,d\theta $.
Then $ x^3 = 64 \tan^3(\theta) $ and $ \sqrt{16 + x^2} = 4 \sec(\theta) $.
Substitute into the integral:
\[ \int \frac{64 \tan^3(\theta)}{4 \sec(\theta)} (4 \sec^2(\theta) \,d\theta) = 64 \int \tan^3(\theta)\sec(\theta) \,d\theta \]
Rewrite as $ 64 \int (\sec^2\theta - 1) \sec\theta\tan\theta \,d\theta $. Let $ u = \sec(\theta) $, so $ du = \sec(\theta)\tan(\theta) \,d\theta $.
\[ 64 \int (u^2 - 1) \,du = 64\left(\frac{u^3}{3} - u\right) + C = \frac{64}{3}\sec^3(\theta) - 64\sec(\theta) + C \]
From $ x = 4 \tan(\theta) $, the triangle gives $ \sec(\theta) = \frac{\sqrt{x^2 + 16}}{4} $. Substituting back:
\[ \frac{64}{3}\left(\frac{\sqrt{x^2 + 16}}{4}\right)^3 - 64\left(\frac{\sqrt{x^2 + 16}}{4}\right) + C = \frac{1}{3}(x^2+16)^{3/2} - 16\sqrt{x^2+16} + C \]
\textbf{Answer:} $ \frac{1}{3}(x^2 - 32)\sqrt{x^2 + 16} + C $


%---------------------------------------------------------------
\subsection{Problem 5}
Evaluate $ \int \frac{\sqrt{4x^2 - 25}}{x} \,dx $ using $ x = \frac{5}{2} \sec(\theta) $.
\subsubsection*{Solution}
Let $ x = \frac{5}{2} \sec(\theta) $, so $ dx = \frac{5}{2} \sec(\theta)\tan(\theta) \,d\theta $.
Then $ \sqrt{4x^2 - 25} = \sqrt{25\sec^2(\theta) - 25} = 5\tan(\theta) $.
Substitute into the integral:
\[ \int \frac{5\tan(\theta)}{\frac{5}{2}\sec(\theta)} \left(\frac{5}{2}\sec(\theta)\tan(\theta) \,d\theta\right) = \int 5\tan^2(\theta) \,d\theta \]
Using the identity $ \tan^2(\theta) = \sec^2(\theta) - 1 $:
\[ 5 \int (\sec^2(\theta) - 1) \,d\theta = 5\tan(\theta) - 5\theta + C \]
From the substitution, $ \tan(\theta) = \frac{\sqrt{4x^2-25}}{5} $ and $ \theta = \operatorname{arcsec}(\frac{2x}{5}) $.
\textbf{Answer:} $ \sqrt{4x^2 - 25} - 5 \operatorname{arcsec}\left(\frac{2x}{5}\right) + C $


%---------------------------------------------------------------
\subsection{Problem 6}
Consider $ \int \frac{x^4}{\sqrt{1 + x^2}} \,dx $. Transform the integral using a trigonometric substitution.
\subsubsection*{Solution}
The form $ \sqrt{1 + x^2} $ suggests the substitution $ x = \tan(\theta) $, so $ dx = \sec^2(\theta) \,d\theta $.
\[ \int \frac{\tan^4(\theta)}{\sqrt{1 + \tan^2(\theta)}} \sec^2(\theta) \,d\theta = \int \frac{\tan^4(\theta)}{\sec(\theta)} \sec^2(\theta) \,d\theta \]
\textbf{Answer:} $ \int \tan^4(\theta)\sec(\theta) \,d\theta $


%---------------------------------------------------------------
\subsection{Problem 7}
Evaluate the integral: $ \int_{2}^{5} \frac{dx}{(x^2 - 1)^{3/2}} $
\subsubsection*{Solution}
Let $ x = \sec(\theta) $, so $ dx = \sec(\theta)\tan(\theta)d\theta $. The denominator is $ \tan^3(\theta) $.
Limits: $ x=2 \implies \theta = \pi/3 $ and $ x=5 \implies \theta = \operatorname{arcsec}(5) $.
\[ \int_{\pi/3}^{\operatorname{arcsec}(5)} \frac{\sec(\theta)\tan(\theta)}{\tan^3(\theta)} \,d\theta = \int_{\pi/3}^{\operatorname{arcsec}(5)} \frac{\sec(\theta)}{\tan^2(\theta)} \,d\theta = \int_{\pi/3}^{\operatorname{arcsec}(5)} \cot(\theta)\csc(\theta) \,d\theta \]
The integral is $ [-\csc(\theta)]_{\pi/3}^{\operatorname{arcsec}(5)} = -\csc(\operatorname{arcsec}(5)) - (-\csc(\frac{\pi}{3})) = \frac{2}{\sqrt{3}} - \frac{5}{\sqrt{24}} $.
\textbf{Answer:} $ \frac{2\sqrt{3}}{3} - \frac{5\sqrt{6}}{12} $


%---------------------------------------------------------------
\subsection{Problem 8}
Evaluate the integral: $ \int_{0}^{4} \frac{dt}{\sqrt{16 + t^2}} $
\subsubsection*{Solution}
The antiderivative of $ \frac{1}{\sqrt{a^2 + t^2}} $ is $ \ln|t + \sqrt{a^2 + t^2}| $.
\begin{align*}
[\ln|t + \sqrt{16 + t^2}|]_{0}^{4} &= (\ln|4 + \sqrt{16 + 16}|) - (\ln|0 + \sqrt{16}|) \\
&= \ln(4 + 4\sqrt{2}) - \ln(4) = \ln\left(\frac{4(1 + \sqrt{2})}{4}\right)
\end{align*}
\textbf{Answer:} $ \ln(1 + \sqrt{2}) $


%---------------------------------------------------------------
\subsection{Problem 9}
Evaluate $ \int_{0}^{7} \frac{7}{\sqrt{49 + t^2}} \,dt $.
\subsubsection*{Solution}
$ 7 \int_{0}^{7} \frac{dt}{\sqrt{49 + t^2}} = 7 [\ln|t + \sqrt{49 + t^2}|]_{0}^{7} $
\begin{align*}
&= 7 [(\ln|7 + \sqrt{49+49}|) - (\ln|0 + \sqrt{49}|)] \\
&= 7 [\ln(7 + 7\sqrt{2}) - \ln(7)] = 7 \ln\left(\frac{7(1+\sqrt{2})}{7}\right)
\end{align*}
\textbf{Answer:} $ 7 \ln(1 + \sqrt{2}) $


%---------------------------------------------------------------
\subsection{Problem 10}
Evaluate the integral: $ \int \frac{\sqrt{x^2 - 25}}{x^3} \,dx $
\subsubsection*{Solution}
Let $ x = 5 \sec(\theta) $, so $ dx = 5 \sec(\theta)\tan(\theta) \,d\theta $.
\[ \int \frac{5\tan(\theta)}{125\sec^3(\theta)} 5\sec(\theta)\tan(\theta) \,d\theta = \frac{1}{5} \int \frac{\tan^2(\theta)}{\sec^2(\theta)} \,d\theta = \frac{1}{5} \int \sin^2(\theta) \,d\theta \]
Using the half-angle identity: $ \frac{1}{10} \int (1 - \cos(2\theta)) \,d\theta = \frac{1}{10}(\theta - \sin(\theta)\cos(\theta)) + C $.
From $ x=5\sec(\theta) $, we have $ \theta=\operatorname{arcsec}(x/5) $, $ \sin(\theta)=\frac{\sqrt{x^2-25}}{x} $, and $ \cos(\theta)=\frac{5}{x} $.
\textbf{Answer:} $ \frac{1}{10}\left[\operatorname{arcsec}\left(\frac{x}{5}\right) - \frac{5\sqrt{x^2-25}}{x^2}\right] + C $


%---------------------------------------------------------------
\subsection{Problem 11}
Evaluate the integral: $ \int \frac{\sqrt{4 + x^2}}{x} \,dx $
\subsubsection*{Solution}
Let $ x = 2 \tan(\theta) $, so $ dx = 2 \sec^2(\theta) \,d\theta $.
\[ \int \frac{2\sec(\theta)}{2\tan(\theta)} 2\sec^2(\theta) \,d\theta = 2 \int \frac{\sec^3(\theta)}{\tan(\theta)} \,d\theta = 2 \int (\sec(\theta)\tan(\theta) + \csc(\theta)) \,d\theta \]
This integrates to $ 2[\sec(\theta) - \ln|\csc(\theta) + \cot(\theta)|] + C $.
From $ x=2\tan(\theta) $, we have $ \sec(\theta)=\frac{\sqrt{x^2+4}}{2} $, $ \csc(\theta)=\frac{\sqrt{x^2+4}}{x} $, and $ \cot(\theta)=\frac{2}{x} $.
\textbf{Answer:} $ \sqrt{4+x^2} - 2\ln\left|\frac{\sqrt{4+x^2}+2}{x}\right| + C $


%---------------------------------------------------------------
\subsection{Problem 12}
Evaluate the integral: $ \int 3x\sqrt{1 - x^4} \,dx $
\subsubsection*{Solution}
Let $ u = x^2 $, then $ du = 2x \,dx \implies x \,dx = \frac{du}{2} $.
\[ \int 3\sqrt{1 - u^2} \left(\frac{du}{2}\right) = \frac{3}{2} \int \sqrt{1 - u^2} \,du \]
The integral of $ \sqrt{1-u^2} $ is a standard form: $ \frac{1}{2}(u\sqrt{1-u^2} + \arcsin(u)) $.
\[ \frac{3}{2} \cdot \frac{1}{2}[u\sqrt{1 - u^2} + \arcsin(u)] + C \]
Substituting back $ u = x^2 $:
\textbf{Answer:} $ \frac{3}{4}[x^2\sqrt{1 - x^4} + \arcsin(x^2)] + C $


%---------------------------------------------------------------
\subsection{Problem 13}
Evaluate the integral: $ \int x^3\sqrt{64 + x^2} \,dx $
\subsubsection*{Solution}
Let $ u = 64 + x^2 $, so $ du = 2x \,dx $ and $ x^2 = u - 64 $.
Rewrite as $ \int x^2\sqrt{64 + x^2} \cdot (x \,dx) $. Substitute:
\[ \int (u - 64)\sqrt{u} \left(\frac{du}{2}\right) = \frac{1}{2} \int (u^{3/2} - 64u^{1/2}) \,du \]
Integrate: $ \frac{1}{2}\left[\frac{2}{5}u^{5/2} - 64 \cdot \frac{2}{3}u^{3/2}\right] + C = \frac{1}{5}u^{5/2} - \frac{64}{3}u^{3/2} + C $.
Factor to simplify: $ \frac{1}{15}u^{3/2}(3u - 320) + C $. Substitute back $ u = 64 + x^2 $:
\[ \frac{1}{15}(64+x^2)^{3/2}(3(64+x^2) - 320) + C \]
\textbf{Answer:} $ \frac{1}{15}(3x^2 - 128)(64 + x^2)^{3/2} + C $


%---------------------------------------------------------------
\subsection{Problem 14}
Evaluate the integral: $ \int \frac{x^2}{\sqrt{49 - x^2}} \,dx $
\subsubsection*{Solution}
Let $ x = 7 \sin(\theta) $, so $ dx = 7 \cos(\theta) \,d\theta $.
\[ \int \frac{49\sin^2(\theta)}{7\cos(\theta)} (7\cos(\theta) \,d\theta) = 49 \int \sin^2(\theta) \,d\theta \]
Use the half-angle identity:
\[ \frac{49}{2} \int (1 - \cos(2\theta)) \,d\theta = \frac{49}{2}\left(\theta - \frac{1}{2}\sin(2\theta)\right) + C = \frac{49}{2}(\theta - \sin(\theta)\cos(\theta)) + C \]
From $ x=7\sin(\theta) $, we have $ \theta=\arcsin(x/7) $ and $ \cos(\theta)=\frac{\sqrt{49-x^2}}{7} $.
\textbf{Answer:} $ \frac{49}{2}\arcsin\left(\frac{x}{7}\right) - \frac{x}{2}\sqrt{49 - x^2} + C $


\section*{Summary of Rules, Formulas, and Tricks}
This set of problems primarily tests u-substitution and trigonometric substitution.

\subsection*{Key Integration Rules \& Formulas}
\begin{itemize}
    \item \textbf{Power Rule:}
    \[ \int u^n \,du = \frac{u^{n+1}}{n+1} + C, \quad (n \neq -1) \]
    
    \item \textbf{U-Substitution:} The main strategy is to find a function \texttt{u} in the integrand whose derivative \texttt{du} also appears. This simplifies the integral into a more basic form. Look for an "inside" function and its derivative on the "outside".
    
    \item \textbf{Trigonometric Identities:}
    \begin{itemize}
        \item \textbf{Pythagorean:} $ \sin^2\theta + \cos^2\theta = 1 $ \quad and \quad $ 1 + \tan^2\theta = \sec^2\theta $
        \item \textbf{Half-Angle:} $ \sin^2\theta = \frac{1}{2}(1 - \cos(2\theta)) $ \quad and \quad $ \cos^2\theta = \frac{1}{2}(1 + \cos(2\theta)) $
    \end{itemize}
\end{itemize}

\subsection*{Trigonometric Substitution Standard Forms}
The trick is to recognize which form the integral takes based on the expression under the square root.
\begin{enumerate}
    \item \textbf{Form $ \sqrt{a^2 - x^2} $}
    \begin{itemize}
        \item \textbf{Substitution:} $ x = a \sin(\theta) $
        \item \textbf{Identity Used:} $ a^2 - a^2\sin^2(\theta) = a^2\cos^2(\theta) $
    \end{itemize}

    \item \textbf{Form $ \sqrt{a^2 + x^2} $}
    \begin{itemize}
        \item \textbf{Substitution:} $ x = a \tan(\theta) $
        \item \textbf{Identity Used:} $ a^2 + a^2\tan^2(\theta) = a^2\sec^2(\theta) $
    \end{itemize}

    \item \textbf{Form $ \sqrt{x^2 - a^2} $}
    \begin{itemize}
        \item \textbf{Substitution:} $ x = a \sec(\theta) $
        \item \textbf{Identity Used:} $ a^2\sec^2(\theta) - a^2 = a^2\tan^2(\theta) $
    \end{itemize}
\end{enumerate}

\subsection*{Tricks and Important Concepts Shown}
\begin{itemize}
    \item \textbf{Look for U-Sub First:} Before attempting a complex trigonometric substitution, always check if a simple u-substitution will work. It is often much faster.
    
    \item \textbf{Change Limits of Integration:} For definite integrals, when you substitute variables (e.g., from $x$ to $\theta$), you \textbf{must} change the limits of integration to the new variable's values. This avoids the final step of converting back to $x$.
    
    \item \textbf{Draw the Triangle:} For indefinite integrals, after you integrate in terms of $\theta$, you must convert back to $x$. Drawing a right triangle based on your initial substitution (e.g., if $x = a \tan(\theta)$, then $\tan(\theta) = x/a$) is the most reliable way to find expressions for $\sin(\theta)$, $\sec(\theta)$, etc., in terms of $x$.
\end{itemize}


\end{document}