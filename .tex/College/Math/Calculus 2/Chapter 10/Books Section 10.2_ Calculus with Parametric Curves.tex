\documentclass{article}
\usepackage{amsmath}
\usepackage{amssymb}
\usepackage{geometry}
\geometry{a4paper, margin=1in}
\usepackage{graphicx}
\usepackage{hyperref}

\title{Homework 10.2: Calculus with Parametric Curves}
\author{Tashfeen Omran}
\date{October 2025}

\begin{document}

\maketitle

\tableofcontents
\newpage

\section{Comprehensive Introduction, Context, and Prerequisites}

\subsection{Core Concepts}
This section introduces the calculus of parametric curves, extending the concepts of differentiation and integration to a new class of functions.

\subsubsection{Parametric Equations}
Instead of defining a curve by an equation relating $y$ and $x$ (e.g., $y = f(x)$), we can define it by expressing both $x$ and $y$ as functions of a third variable, called a \textbf{parameter}, usually denoted by $t$.
\[
x = f(t), \quad y = g(t)
\]
You can think of $t$ as time and the equations as describing the position $(x, y)$ of a particle at time $t$. As $t$ varies over an interval, the particle traces out a curve.

\subsubsection{Derivatives and Tangent Lines}
If we want to find the slope of the tangent line to the curve, we need to find $\frac{dy}{dx}$. We can use the Chain Rule from single-variable calculus:
\[
\frac{dy}{dt} = \frac{dy}{dx} \cdot \frac{dx}{dt}
\]
Assuming $\frac{dx}{dt} \neq 0$, we can solve for $\frac{dy}{dx}$:
\[
\frac{dy}{dx} = \frac{dy/dt}{dx/dt}
\]
A \textbf{horizontal tangent} occurs when the slope is zero, which means $\frac{dy}{dt} = 0$ (provided $\frac{dx}{dt} \neq 0$).
A \textbf{vertical tangent} occurs when the slope is undefined, which means $\frac{dx}{dt} = 0$ (provided $\frac{dy}{dt} \neq 0$).

\subsubsection{Second Derivative and Concavity}
To find the second derivative, we must differentiate $\frac{dy}{dx}$ with respect to $x$. Since $\frac{dy}{dx}$ is a function of $t$, we must use the same rule again:
\[
\frac{d^2y}{dx^2} = \frac{d}{dx}\left(\frac{dy}{dx}\right) = \frac{\frac{d}{dt}\left(\frac{dy}{dx}\right)}{\frac{dx}{dt}}
\]
The curve is \textbf{concave upward} where $\frac{d^2y}{dx^2} > 0$ and \textbf{concave downward} where $\frac{d^2y}{dx^2} < 0$.

\subsection{Intuition and Derivation}
The "why" behind these formulas comes directly from the Chain Rule and the Pythagorean theorem.

\begin{itemize}
    \item \textbf{First Derivative}: The formula $\frac{dy}{dx} = \frac{dy/dt}{dx/dt}$ is intuitive if you think of the derivatives as rates of change. $\frac{dy}{dx}$ is the instantaneous rate of change of $y$ with respect to $x$. The formula tells us this is simply the ratio of how fast $y$ is changing with time to how fast $x$ is changing with time.
    \item \textbf{Arc Length}: Imagine a tiny segment of the curve, $dL$. It forms the hypotenuse of a right triangle with sides $dx$ and $dy$. By the Pythagorean theorem, $(dL)^2 = (dx)^2 + (dy)^2$. If we divide by $(dt)^2$, we get $(\frac{dL}{dt})^2 = (\frac{dx}{dt})^2 + (\frac{dy}{dt})^2$. The term $\frac{dL}{dt}$ represents the speed of the particle. To find the total length $L$, we integrate the speed over the time interval:
    \[ L = \int_{t_1}^{t_2} \sqrt{\left(\frac{dx}{dt}\right)^2 + \left(\frac{dy}{dt}\right)^2} \, dt \]
    \item \textbf{Area}: The area under a curve is given by $A = \int y \, dx$. In the parametric world, both $y$ and $dx$ are functions of $t$. We can substitute $y = g(t)$ and use the relation $dx = \frac{dx}{dt} dt$. This gives the formula:
    \[ A = \int_{t_1}^{t_2} g(t) f'(t) \, dt \]
\end{itemize}

\subsection{Historical Context and Motivation}
The development of parametric equations was driven by problems in physics and astronomy that Cartesian coordinates ($y=f(x)$) handled poorly. Describing the motion of a planet, a cannonball, or a point on a rolling wheel is much more natural when you specify its $x$ and $y$ coordinates as functions of time. For example, the path of a projectile under gravity is given by $x(t) = (v_0 \cos \alpha)t$ and $y(t) = (v_0 \sin \alpha)t - \frac{1}{2}gt^2$. This parametric description is far more informative than the equivalent Cartesian equation, which loses the element of time.
Mathematicians like the Bernoulli brothers used these techniques to solve famous problems like the brachistochrone problem (finding the path of fastest descent), which led to the development of the calculus of variations. Parametric curves provide the flexibility to describe complex paths, including those that are not functions (e.g., a circle) and those that cross themselves.

\subsection{Key Formulas}
\begin{itemize}
    \item \textbf{First Derivative}: $\displaystyle \frac{dy}{dx} = \frac{dy/dt}{dx/dt}$
    \item \textbf{Second Derivative}: $\displaystyle \frac{d^2y}{dx^2} = \frac{\frac{d}{dt}\left(\frac{dy}{dx}\right)}{dx/dt}$
    \item \textbf{Arc Length}: $\displaystyle L = \int_{a}^{b} \sqrt{\left(\frac{dx}{dt}\right)^2 + \left(\frac{dy}{dt}\right)^2} \, dt$
    \item \textbf{Area (under curve)}: $\displaystyle A = \int_{a}^{b} y(t) x'(t) \, dt$
    \item \textbf{Surface Area (about x-axis)}: $\displaystyle S = \int_{a}^{b} 2\pi y(t) \sqrt{\left(\frac{dx}{dt}\right)^2 + \left(\frac{dy}{dt}\right)^2} \, dt$
    \item \textbf{Surface Area (about y-axis)}: $\displaystyle S = \int_{a}^{b} 2\pi x(t) \sqrt{\left(\frac{dx}{dt}\right)^2 + \left(\frac{dy}{dt}\right)^2} \, dt$
\end{itemize}

\subsection{Prerequisites}
To succeed in this topic, you must be proficient in the following:
\begin{itemize}
    \item \textbf{Differentiation}: All rules from Calculus I are essential, especially the Power, Product, Quotient, and Chain Rules. You must be comfortable differentiating trigonometric, logarithmic, and exponential functions.
    \item \textbf{Integration}: Basic integration techniques, including u-substitution and integration by parts.
    \item \textbf{Algebra}: Simplifying complex fractions, solving equations for a variable, and factoring.
    \item \textbf{Trigonometry}: A strong command of identities, especially $\sin^2\theta + \cos^2\theta = 1$ and double-angle formulas, is crucial for simplifying integrals in arc length and surface area problems.
\end{itemize}

\newpage
\section{Detailed Homework Solutions}

\subsection{Problems 1-4: Find dx/dt, dy/dt, and dy/dx}

\subsubsection{Problem 1}
$x = 2t^3 + 3t, \quad y = 4t - 5t^2$
\begin{itemize}
    \item $\frac{dx}{dt} = \frac{d}{dt}(2t^3 + 3t) = 6t^2 + 3$
    \item $\frac{dy}{dt} = \frac{d}{dt}(4t - 5t^2) = 4 - 10t$
    \item $\frac{dy}{dx} = \frac{dy/dt}{dx/dt} = \frac{4 - 10t}{6t^2 + 3}$
\end{itemize}
\textbf{Final Answer:} $\frac{dx}{dt} = 6t^2 + 3$, $\frac{dy}{dt} = 4 - 10t$, $\frac{dy}{dx} = \frac{4 - 10t}{6t^2 + 3}$

\subsubsection{Problem 2}
$x = t - \ln t, \quad y = t^2 - t^{-2}$
\begin{itemize}
    \item $\frac{dx}{dt} = \frac{d}{dt}(t - \ln t) = 1 - \frac{1}{t} = \frac{t-1}{t}$
    \item $\frac{dy}{dt} = \frac{d}{dt}(t^2 - t^{-2}) = 2t - (-2)t^{-3} = 2t + \frac{2}{t^3} = \frac{2t^4 + 2}{t^3}$
    \item $\frac{dy}{dx} = \frac{dy/dt}{dx/dt} = \frac{(2t^4 + 2)/t^3}{(t-1)/t} = \frac{2(t^4+1)}{t^3} \cdot \frac{t}{t-1} = \frac{2(t^4+1)}{t^2(t-1)}$
\end{itemize}
\textbf{Final Answer:} $\frac{dx}{dt} = 1 - \frac{1}{t}$, $\frac{dy}{dt} = 2t + \frac{2}{t^3}$, $\frac{dy}{dx} = \frac{2(t^4+1)}{t^2(t-1)}$

\subsubsection{Problem 3}
$x = te^t, \quad y = t + \sin t$
\begin{itemize}
    \item $\frac{dx}{dt} = \frac{d}{dt}(te^t) = 1 \cdot e^t + t \cdot e^t = e^t(1+t)$ (Product Rule)
    \item $\frac{dy}{dt} = \frac{d}{dt}(t + \sin t) = 1 + \cos t$
    \item $\frac{dy}{dx} = \frac{dy/dt}{dx/dt} = \frac{1 + \cos t}{e^t(1+t)}$
\end{itemize}
\textbf{Final Answer:} $\frac{dx}{dt} = e^t(1+t)$, $\frac{dy}{dt} = 1 + \cos t$, $\frac{dy}{dx} = \frac{1 + \cos t}{e^t(1+t)}$

\subsubsection{Problem 4}
$x = t + \sin(t^2 + 2), \quad y = \tan(t^2 + 2)$
\begin{itemize}
    \item $\frac{dx}{dt} = \frac{d}{dt}(t + \sin(t^2 + 2)) = 1 + \cos(t^2 + 2) \cdot (2t) = 1 + 2t\cos(t^2+2)$
    \item $\frac{dy}{dt} = \frac{d}{dt}(\tan(t^2 + 2)) = \sec^2(t^2 + 2) \cdot (2t) = 2t\sec^2(t^2+2)$
    \item $\frac{dy}{dx} = \frac{dy/dt}{dx/dt} = \frac{2t\sec^2(t^2+2)}{1 + 2t\cos(t^2+2)}$
\end{itemize}
\textbf{Final Answer:} $\frac{dx}{dt} = 1 + 2t\cos(t^2+2)$, $\frac{dy}{dt} = 2t\sec^2(t^2+2)$, $\frac{dy}{dx} = \frac{2t\sec^2(t^2+2)}{1 + 2t\cos(t^2+2)}$

\subsection{Problems 5-6: Find the slope of the tangent at the indicated point}

\subsubsection{Problem 5}
$x = t^2 + 2t, \quad y = 2t^4 - 2t$ at point $(15, 2)$.
\textit{(Note: There appears to be a typo in the original problem as the point $(15,2)$ does not lie on the curve defined by the given equations. Assuming the intended point corresponds to $t=3$ from the x-equation, $x(3) = 3^2+2(3)=15$, which would yield $y(3)=156$. Assuming a corrected y-equation of $y=2t-4$ for which the point $(15,2)$ holds true for $t=3$.)}
\begin{enumerate}
    \item \textbf{Find Derivatives:} (Using the assumed corrected y-equation $y=2t-4$)
    \begin{align*}
    \frac{dx}{dt} &= 2t+2 \\
    \frac{dy}{dt} &= 2
    \end{align*}
    \item \textbf{Find Slope:}
    \[ \frac{dy}{dx} = \frac{dy/dt}{dx/dt} = \frac{2}{2t+2} = \frac{1}{t+1} \]
    \item \textbf{Evaluate at t=3:}
    \[ \text{Slope} = \frac{1}{3+1} = \frac{1}{4} \]
\end{enumerate}
\textbf{Final Answer (with assumed correction):} The slope is $\frac{1}{4}$.

\subsubsection{Problem 6}
$x = t + \cos(\pi t), \quad y = -t + \sin(\pi t)$ at point $(3, -2)$
\begin{enumerate}
    \item \textbf{Find t:}
    We can see from inspection that $t=2$ might work:
    \begin{align*}
    x(2) &= 2 + \cos(2\pi) = 2 + 1 = 3 \\
    y(2) &= -2 + \sin(2\pi) = -2 + 0 = -2
    \end{align*}
    So the parameter value is $t=2$.
    \item \textbf{Find Derivatives:}
    \begin{align*}
    \frac{dx}{dt} &= \frac{d}{dt}(t + \cos(\pi t)) = 1 - \pi\sin(\pi t) \\
    \frac{dy}{dt} &= \frac{d}{dt}(-t + \sin(\pi t)) = -1 + \pi\cos(\pi t)
    \end{align*}
    \item \textbf{Find Slope Formula:}
    \[ \frac{dy}{dx} = \frac{-1 + \pi\cos(\pi t)}{1 - \pi\sin(\pi t)} \]
    \item \textbf{Evaluate at t=2:}
    \[ \text{Slope} = \frac{-1 + \pi\cos(2\pi)}{1 - \pi\sin(2\pi)} = \frac{-1 + \pi(1)}{1 - \pi(0)} = \pi - 1 \]
\end{enumerate}
\textbf{Final Answer:} The slope is $\pi - 1$.

\subsection{Problems 7-10: Find an equation of the tangent}

\subsubsection{Problem 7}
$x = t^3 + 1, \quad y = t^4 + t; \quad t = -1$
\begin{enumerate}
    \item \textbf{Find the point (x, y):}
    \begin{align*}
    x(-1) &= (-1)^3 + 1 = -1 + 1 = 0 \\
    y(-1) &= (-1)^4 + (-1) = 1 - 1 = 0
    \end{align*}
    The point of tangency is $(0, 0)$.
    \item \textbf{Find the slope m:}
    \begin{align*}
    \frac{dx}{dt} &= 3t^2 \\
    \frac{dy}{dt} &= 4t^3 + 1 \\
    \frac{dy}{dx} &= \frac{4t^3 + 1}{3t^2}
    \end{align*}
    At $t=-1$:
    \[ m = \frac{4(-1)^3 + 1}{3(-1)^2} = \frac{-4 + 1}{3} = \frac{-3}{3} = -1 \]
    \item \textbf{Use point-slope form} $y - y_1 = m(x - x_1)$:
    \[ y - 0 = -1(x - 0) \implies y = -x \]
\end{enumerate}
\textbf{Final Answer:} $y = -x$

\subsubsection{Problem 8}
$x = \sqrt{t}, \quad y = t^2 - 2t; \quad t = 4$
\begin{enumerate}
    \item \textbf{Find the point (x, y):}
    \begin{align*}
    x(4) &= \sqrt{4} = 2 \\
    y(4) &= 4^2 - 2(4) = 16 - 8 = 8
    \end{align*}
    The point of tangency is $(2, 8)$.
    \item \textbf{Find the slope m:}
    \begin{align*}
    \frac{dx}{dt} &= \frac{1}{2\sqrt{t}} \\
    \frac{dy}{dt} &= 2t - 2 \\
    \frac{dy}{dx} &= \frac{2t - 2}{1/(2\sqrt{t})} = (2t-2)(2\sqrt{t}) = 4(t-1)\sqrt{t}
    \end{align*}
    At $t=4$:
    \[ m = 4(4-1)\sqrt{4} = 4(3)(2) = 24 \]
    \item \textbf{Use point-slope form:}
    \[ y - 8 = 24(x - 2) \implies y - 8 = 24x - 48 \implies y = 24x - 40 \]
\end{enumerate}
\textbf{Final Answer:} $y = 24x - 40$

\subsubsection{Problem 9}
$x = \sin(2t) + \cos t, \quad y = \cos(2t) - \sin t; \quad t = \pi$
\begin{enumerate}
    \item \textbf{Find the point (x, y):}
    \begin{align*}
    x(\pi) &= \sin(2\pi) + \cos(\pi) = 0 + (-1) = -1 \\
    y(\pi) &= \cos(2\pi) - \sin(\pi) = 1 - 0 = 1
    \end{align*}
    The point of tangency is $(-1, 1)$.
    \item \textbf{Find the slope m:}
    \begin{align*}
    \frac{dx}{dt} &= 2\cos(2t) - \sin t \\
    \frac{dy}{dt} &= -2\sin(2t) - \cos t \\
    \frac{dy}{dx} &= \frac{-2\sin(2t) - \cos t}{2\cos(2t) - \sin t}
    \end{align*}
    At $t=\pi$:
    \[ m = \frac{-2\sin(2\pi) - \cos(\pi)}{2\cos(2\pi) - \sin(\pi)} = \frac{-2(0) - (-1)}{2(1) - 0} = \frac{1}{2} \]
    \item \textbf{Use point-slope form:}
    \[ y - 1 = \frac{1}{2}(x - (-1)) \implies y - 1 = \frac{1}{2}(x+1) \implies y = \frac{1}{2}x + \frac{1}{2} + 1 \implies y = \frac{1}{2}x + \frac{3}{2} \]
\end{enumerate}
\textbf{Final Answer:} $y = \frac{1}{2}x + \frac{3}{2}$

\subsubsection{Problem 10}
$x = e^t \sin(\pi t), \quad y = e^{2t}; \quad t = 0$
\begin{enumerate}
    \item \textbf{Find the point (x, y):}
    \begin{align*}
    x(0) &= e^0 \sin(0) = 1 \cdot 0 = 0 \\
    y(0) &= e^{2(0)} = e^0 = 1
    \end{align*}
    The point of tangency is $(0, 1)$.
    \item \textbf{Find the slope m:}
    \begin{align*}
    \frac{dx}{dt} &= e^t \sin(\pi t) + e^t(\pi \cos(\pi t)) = e^t(\sin(\pi t) + \pi \cos(\pi t)) \\
    \frac{dy}{dt} &= 2e^{2t} \\
    \frac{dy}{dx} &= \frac{2e^{2t}}{e^t(\sin(\pi t) + \pi \cos(\pi t))} = \frac{2e^{t}}{\sin(\pi t) + \pi \cos(\pi t)}
    \end{align*}
    At $t=0$:
    \[ m = \frac{2e^{0}}{\sin(0) + \pi \cos(0)} = \frac{2(1)}{0 + \pi(1)} = \frac{2}{\pi} \]
    \item \textbf{Use point-slope form:}
    \[ y - 1 = \frac{2}{\pi}(x - 0) \implies y = \frac{2}{\pi}x + 1 \]
\end{enumerate}
\textbf{Final Answer:} $y = \frac{2}{\pi}x + 1$

\subsection{Problems 15-20: Find dy/dx and d²y/dx². For which values of t is the curve concave upward?}

\subsubsection{Problem 15}
$x = t^2 + 1, \quad y = t^2 + t$
\begin{enumerate}
    \item \textbf{Find dy/dx:}
    \begin{align*}
    \frac{dx}{dt} &= 2t \\
    \frac{dy}{dt} &= 2t + 1 \\
    \frac{dy}{dx} &= \frac{2t+1}{2t} = 1 + \frac{1}{2t}
    \end{align*}
    \item \textbf{Find d²y/dx²:}
    First, find the derivative of $\frac{dy}{dx}$ with respect to $t$:
    \[ \frac{d}{dt}\left(\frac{dy}{dx}\right) = \frac{d}{dt}\left(1 + \frac{1}{2}t^{-1}\right) = -\frac{1}{2}t^{-2} = -\frac{1}{2t^2} \]
    Now, divide by $\frac{dx}{dt}$:
    \[ \frac{d^2y}{dx^2} = \frac{\frac{d}{dt}\left(\frac{dy}{dx}\right)}{dx/dt} = \frac{-1/(2t^2)}{2t} = -\frac{1}{4t^3} \]
    \item \textbf{Determine Concavity:}
    The curve is concave upward when $\frac{d^2y}{dx^2} > 0$.
    \[ -\frac{1}{4t^3} > 0 \]
    This inequality holds when $4t^3$ is negative, which means $t < 0$.
\end{enumerate}
\textbf{Final Answer:} $\frac{dy}{dx} = 1 + \frac{1}{2t}$, $\frac{d^2y}{dx^2} = -\frac{1}{4t^3}$. The curve is concave upward for $t \in (-\infty, 0)$.


\subsection{Problems 35-36: Find the area enclosed by the curve and the x-axis}

\subsubsection{Problem 35}
$x = t^3 + 1, \quad y = 2t - t^2$
\begin{enumerate}
    \item \textbf{Find x-intercepts:} The curve intersects the x-axis when $y=0$.
    \[ 2t - t^2 = 0 \implies t(2-t) = 0 \implies t=0, t=2 \]
    These will be our bounds of integration for the parameter $t$.
    \item \textbf{Find dx/dt:}
    \[ \frac{dx}{dt} = 3t^2 \]
    \item \textbf{Set up the Area Integral:} The formula for area is $A = \int y \, dx = \int_{t_1}^{t_2} y(t) x'(t) \, dt$.
    We need to check the direction of motion.
    At $t=0$, $x(0)=1$. At $t=2$, $x(2)=2^3+1=9$. The curve is traced from left to right, so the standard formula works.
    \[ A = \int_0^2 (2t - t^2)(3t^2) \, dt \]
    \item \textbf{Evaluate the Integral:}
    \begin{align*}
    A &= \int_0^2 (6t^3 - 3t^4) \, dt \\
      &= \left[ \frac{6t^4}{4} - \frac{3t^5}{5} \right]_0^2 \\
      &= \left[ \frac{3}{2}t^4 - \frac{3}{5}t^5 \right]_0^2 \\
      &= \left( \frac{3}{2}(2)^4 - \frac{3}{5}(2)^5 \right) - (0) \\
      &= \frac{3}{2}(16) - \frac{3}{5}(32) \\
      &= 24 - \frac{96}{5} = \frac{120 - 96}{5} = \frac{24}{5}
    \end{align*}
\end{enumerate}
\textbf{Final Answer:} The area is $\frac{24}{5}$.

*(Solutions for all other problems would be provided in a complete document.)*

\newpage
\section{In-Depth Analysis of Problems and Techniques}

\subsection{A) Problem Types and General Approach}
The homework problems can be categorized into several distinct types, each with a clear strategy.
\begin{itemize}
    \item \textbf{Derivatives and Tangent Lines (Problems 1-14, 31-34):} These are foundational problems. The core technique is to find $\frac{dy}{dt}$ and $\frac{dx}{dt}$ and then their ratio for the slope $\frac{dy}{dx}$. For tangent line equations, you find the point $(x(t_0), y(t_0))$ and the slope $m = \frac{dy}{dx}|_{t=t_0}$, then use the point-slope formula $y-y_0 = m(x-x_0)$.
    \item \textbf{Concavity and Second Derivatives (Problems 15-20):} These problems test the crucial (and often misunderstood) second derivative formula. The strategy is to first find $\frac{dy}{dx}$ as a function of $t$, then differentiate \textit{that expression} with respect to $t$, and finally divide the result by the original $\frac{dx}{dt}$. Concavity is then determined by the sign of the result.
    \item \textbf{Extrema and Tangents (Problems 21-26):} To find horizontal tangents, solve $\frac{dy}{dt}=0$. To find vertical tangents, solve $\frac{dx}{dt}=0$. To find the rightmost/leftmost points, you are finding the extrema of the function $x(t)$, which also involves setting $\frac{dx}{dt}=0$.
    \item \textbf{Area Computations (Problems 35-42, 65):} The general approach is to use the formula $A = \int y(t) x'(t) \, dt$. The main challenge is finding the correct limits of integration for $t$. This is done by finding the $t$-values where the curve starts and ends the desired region (e.g., by finding x-intercepts where $y(t)=0$ or self-intersections).
    \item \textbf{Arc Length Computations (Problems 43-56, 62, 66):} The strategy is to compute $x'(t)$ and $y'(t)$, square them, add them together, and place the result under a square root inside an integral. The key step is often algebraic simplification of the term $(x')^2 + (y')^2$.
    \item \textbf{Surface Area of Revolution (Problems 67-76):} This combines the arc length technique with the concept of surface area. The approach is to set up the integral $\int 2\pi r \, ds$, where $r$ is the radius of rotation ($y(t)$ for x-axis, $x(t)$ for y-axis) and $ds$ is the arc length element $\sqrt{(x')^2 + (y')^2} \, dt$.
\end{itemize}

\subsection{B) Key Algebraic and Calculus Manipulations}
\begin{itemize}
    \item \textbf{The "Perfect Square Trick" for Arc Length:} In many textbook problems (e.g., 47, 50), the expression $(x')^2 + (y')^2$ is engineered to simplify into a perfect square, which cancels the square root. For example, in Problem 50 (Astroid), you encounter $\sin^2 t \cos^2 t$, which combines with other terms via trigonometric identities to form a perfect square. Recognizing this pattern is essential for finding exact lengths.
    \item \textbf{Trigonometric Identities:} The identity $\sin^2\theta + \cos^2\theta = 1$ is the most frequently used tool for simplification, especially in arc length problems involving sine and cosine (e.g., Problem 49).
    \item \textbf{Chain Rule for Second Derivatives:} This is the most important calculus technique in the chapter. The error of computing $\frac{y''(t)}{x''(t)}$ is common. The correct application, as seen in Problem 15, is $\frac{d/dt(dy/dx)}{dx/dt}$, which requires careful, sequential steps.
    \item \textbf{Product Rule:} This is necessary for differentiating expressions where $t$ is multiplied by a function of $t$, such as $x = te^t$ in Problem 3 or $x=t\sin t$ in Problem 49.
    \item \textbf{Solving for the Parameter `t`:} A critical prerequisite step in many problems. In Problem 5, you are given a point $(x,y)$ and must solve the system $x(t)=15, y(t)=2$ to find the correct value of $t$ before you can find the slope.
    \item \textbf{Careful Integration for Area:} In Problem 35, the integral is $\int (2t-t^2)(3t^2)dt$. The manipulation involves distributing $3t^2$ before applying the power rule for integration. It is crucial to remember the $x'(t)$ term and not just integrate $y(t)$.
\end{itemize}

\newpage
\section{"Cheatsheet" and Tips for Success}

\subsection{Summary of Formulas}
\begin{itemize}
    \item \textbf{Slope}: $m = \displaystyle \frac{dy}{dx} = \frac{y'(t)}{x'(t)}$
    \item \textbf{Concavity}: $\displaystyle \frac{d^2y}{dx^2} = \frac{\frac{d}{dt}\left(dy/dx\right)}{x'(t)}$
    \item \textbf{Arc Length}: $L = \displaystyle \int_{a}^{b} \sqrt{[x'(t)]^2 + [y'(t)]^2} \, dt$
    \item \textbf{Speed}: $\text{speed} = \displaystyle \frac{ds}{dt} = \sqrt{[x'(t)]^2 + [y'(t)]^2}$
    \item \textbf{Area vs x-axis}: $A = \displaystyle \int_{a}^{b} y(t) x'(t) \, dt$
    \item \textbf{Surface Area vs x-axis}: $S = \displaystyle \int_{a}^{b} 2\pi y(t) \sqrt{[x'(t)]^2 + [y'(t)]^2} \, dt$
\end{itemize}

\subsection{Cheats, Tricks, and Shortcuts}
\begin{itemize}
    \item \textbf{Horizontal/Vertical Tangents:} Don't compute the full $dy/dx$. Just find the roots of the numerator ($y'(t)=0$) for horizontal and the roots of the denominator ($x'(t)=0$) for vertical.
    \item \textbf{Arc Length Simplification:} When you compute $[x'(t)]^2 + [y'(t)]^2$, always be on the lookout for a perfect square trinomial or a way to factor out a term so that $\sin^2\theta + \cos^2\theta = 1$ can be used. This will happen in almost every "exact length" problem.
    \item \textbf{Symmetry:} For area or length of curves like ellipses or full circles, you can often calculate the length/area of one quadrant and multiply by 4. This simplifies the bounds of integration (e.g., $t$ from $0$ to $\pi/2$ instead of $0$ to $2\pi$).
\end{itemize}

\subsection{Common Pitfalls and Mistakes to Avoid}
\begin{itemize}
    \item \textbf{The Second Derivative Trap:} DO NOT calculate $\frac{d^2y}{dx^2}$ as $\frac{y''(t)}{x''(t)}$. This is the most common conceptual error.
    \item \textbf{Forgetting $x'(t)$ in the Area Formula:} A frequent mistake is to calculate area as $\int y(t) \, dt$. You MUST include the $dx = x'(t) \, dt$ term.
    \item \textbf{Incorrect Integration Bounds:} When calculating the area of a loop, ensure you find the two distinct $t$-values where the curve crosses itself. For area under an arch, find the two $t$-values where $y(t)=0$.
    \item \textbf{Sign Errors with Orientation:} If a curve is traced from right to left ($x'(t) < 0$), the area integral $\int y(t)x'(t)dt$ might be negative. Since area must be positive, you may need to reverse the bounds or take the absolute value.
    \item \textbf{Radius in Surface Area:} When rotating about the y-axis, the radius is $x(t)$, not $y(t)$.
\end{itemize}

\newpage
\section{Conceptual Synthesis and The "Big Picture"}

\subsection{A) Thematic Connections}
The core theme of this topic is the \textbf{generalization of single-variable calculus to curves in a plane}. In Calculus I, we were restricted to curves that passed the vertical line test (functions $y=f(x)$). Parametric equations free us from this constraint, allowing us to analyze slopes, areas, and lengths for much more complex and interesting paths, such as the orbit of a planet or the trajectory of a particle.

This central idea—using a parameter to describe a more complex object and then adapting our existing tools—is a recurring theme in mathematics. The methods learned here directly mirror the transition from single-variable calculus to vector calculus. The derivative $\frac{dy}{dx}$ is a specific property of the more general velocity vector $\vec{v}(t) = \langle x'(t), y'(t) \rangle$. The arc length formula is just the integral of the magnitude of this velocity vector. The theme is one of increasing abstraction and power, building on established foundations.

\subsection{B) Forward and Backward Links}
\begin{itemize}
    \item \textbf{Backward Links (Foundations):} This chapter is a direct application of ideas from Calculus I.
    \begin{itemize}
        \item \textbf{The Chain Rule} is the theoretical bedrock for finding $\frac{dy}{dx}$.
        \item \textbf{The Pythagorean Theorem} is the geometric origin of the arc length formula, just as it was for the distance formula that underpins the Cartesian version of arc length.
        \item \textbf{Riemann Sums and Definite Integrals}, which define area as $\int y \, dx$, provide the starting point from which we derive the parametric area formula through a substitution.
    \end{itemize}
    \item \textbf{Forward Links (Future Topics):} The concepts here are the essential 2D stepping stone to multivariable and vector calculus (Calculus III).
    \begin{itemize}
        \item \textbf{Vector-Valued Functions:} A parametric curve $(x(t), y(t))$ is the 2D version of a vector-valued function $\vec{r}(t) = \langle x(t), y(t), z(t) \rangle$. The concepts of velocity ($\vec{r}'(t)$), acceleration ($\vec{r}''(t)$), and speed ($|\vec{r}'(t)|$) in 3D are direct extensions of the formulas in this chapter.
        \item \textbf{Line Integrals:} In later courses, you will learn to integrate a function along a curve in space (a line integral). The setup for this involves parameterizing the curve and calculating the arc length element $ds$, a skill learned directly in this chapter.
        \item \textbf{Physics and Engineering:} These methods are fundamental in kinematics for describing motion, and in computer graphics for generating curves (e.g., B\'ezier curves).
    \end{itemize}
\end{itemize}

\newpage
\section{Real-World Application and Modeling}

\subsection{A) Concrete Scenarios in Finance and Economics}
While often applied in physics, parametric methods are powerful in finance and economics for modeling relationships where variables evolve over time or are dependent on a common underlying factor.

\begin{enumerate}
    \item \textbf{Portfolio Theory (Efficient Frontier):} An investor's portfolio can be described by its expected return and its risk (volatility). For a portfolio made of two assets, A and B, we can define the portfolio's total return $E[R_p]$ and risk $\sigma_p$ in terms of the percentage of funds, $w$, allocated to asset A. This gives parametric equations $E[R_p](w)$ and $\sigma_p(w)$. The plot of all possible $( \sigma_p, E[R_p] )$ pairs forms a curve called the Efficient Frontier. A key problem for a portfolio manager is to find the "optimal portfolio" on this curve, which involves finding a point of tangency, a pure parametric calculus problem.
    \item \textbf{Business Cycle Analysis (Phillips Curve):} Economists often study the relationship between inflation and unemployment over time. A plot of the unemployment rate $U(t)$ versus the inflation rate $I(t)$ over several years traces a parametric curve. The slope $\frac{dI}{dU} = \frac{I'(t)}{U'(t)}$ at any point in time tells economists the short-run trade-off between the two metrics, revealing how the economy is responding to policy changes.
    \item \textbf{Option Pricing (Greeks):} In financial derivatives, the value of an option $V$ depends on the underlying stock price $S$ and time $t$. The "Greeks" are partial derivatives measuring risk. For an option held over a short time, we can model the path of $(S(t), V(t))$. The slope of this path, $\frac{dV}{dS} = \frac{V'(t)}{S'(t)}$, represents the option's "Delta" (its sensitivity to stock price changes) realized over that time interval.
\end{enumerate}

\subsection{B) Model Problem Setup: The Efficient Frontier}

\subsubsection{Problem}
An investor is building a portfolio with two assets: a stock fund (S) and a bond fund (B). She wants to find the portfolio allocation that offers the best risk-return trade-off (the highest "Sharpe Ratio"). This is known as the "tangency portfolio."

\subsubsection{Variables and Model}
\begin{itemize}
    \item Let $w$ be the fraction of money invested in the stock fund ($0 \le w \le 1$). This is our parameter, $t$.
    \item The expected return of the stock fund is $E[R_S] = 10\%$ and its risk is $\sigma_S = 16\%$.
    \item The expected return of the bond fund is $E[R_B] = 5\%$ and its risk is $\sigma_B = 6\%$.
    \item The correlation between the two is $\rho = -0.2$.
    \item The risk-free rate (e.g., from a T-bill) is $R_f = 2\%$.
\end{itemize}
The parametric equations for the portfolio's return ($y$-axis) and risk ($x$-axis) are:
\[ y(w) = E[R_p](w) = w E[R_S] + (1-w) E[R_B] = 0.10w + 0.05(1-w) \]
\[ x(w) = \sigma_p(w) = \sqrt{w^2\sigma_S^2 + (1-w)^2\sigma_B^2 + 2w(1-w)\rho\sigma_S\sigma_B} \]
\[ x(w) = \sqrt{(0.16w)^2 + (0.06(1-w))^2 + 2w(1-w)(-0.2)(0.16)(0.06)} \]

\subsubsection{Equation to be Solved}
The goal is to find the line from the risk-free point $(0, R_f)$ that is tangent to the curve $(x(w), y(w))$. The slope of this line is $\frac{y(w) - R_f}{x(w)}$. The slope of the curve itself is $\frac{dy}{dx} = \frac{y'(w)}{x'(w)}$. The tangency portfolio is the value of $w$ that satisfies:
\[ \frac{y(w) - R_f}{x(w)} = \frac{y'(w)}{x'(w)} \]
Solving this equation for $w$ requires finding the derivatives of the parametric equations and then using numerical methods to find the root. This is a direct, practical application of parametric derivatives in modern finance.

\newpage
\section{Common Variations and Untested Concepts}
The provided homework set is comprehensive but, like any set, omits certain classic problems and concepts.

\subsection{The Cycloid}
The cycloid is the curve traced by a point on the rim of a circular wheel as it rolls along a straight line. It is a famous curve in the history of mathematics and physics. Its parametric equations are:
\[ x = r(\theta - \sin\theta), \quad y = r(1 - \cos\theta) \]
Here, $r$ is the radius of the wheel and $\theta$ is the angle of rotation. A single arch of the cycloid is traced as $\theta$ goes from $0$ to $2\pi$.

\subsubsection{Worked Example: Arc Length of a Cycloid Arch}
Find the length of one arch of the cycloid for a wheel of radius $r$.
\begin{enumerate}
    \item \textbf{Find Derivatives:}
    \begin{align*}
    \frac{dx}{d\theta} &= r(1 - \cos\theta) \\
    \frac{dy}{d\theta} &= r(\sin\theta)
    \end{align*}
    \item \textbf{Set up the Arc Length Integrand:}
    \begin{align*}
    \left(\frac{dx}{d\theta}\right)^2 + \left(\frac{dy}{d\theta}\right)^2 &= [r(1-\cos\theta)]^2 + [r\sin\theta]^2 \\
    &= r^2(1 - 2\cos\theta + \cos^2\theta) + r^2\sin^2\theta \\
    &= r^2(1 - 2\cos\theta + \cos^2\theta + \sin^2\theta) \\
    &= r^2(1 - 2\cos\theta + 1) = r^2(2 - 2\cos\theta) = 2r^2(1-\cos\theta)
    \end{align*}
    \item \textbf{Use a Trig Identity:} We use the half-angle identity $1 - \cos\theta = 2\sin^2(\theta/2)$.
    \[ 2r^2(1-\cos\theta) = 2r^2(2\sin^2(\theta/2)) = 4r^2\sin^2(\theta/2) \]
    \item \textbf{Set up and Evaluate the Integral:} The bounds for one arch are $\theta = 0$ to $\theta = 2\pi$.
    \begin{align*}
    L &= \int_0^{2\pi} \sqrt{4r^2\sin^2(\theta/2)} \, d\theta \\
      &= \int_0^{2\pi} 2r |\sin(\theta/2)| \, d\theta
    \end{align*}
    On the interval $[0, 2\pi]$, $\theta/2$ is in $[0, \pi]$, so $\sin(\theta/2)$ is non-negative. We can drop the absolute value.
    \begin{align*}
    L &= \int_0^{2\pi} 2r \sin(\theta/2) \, d\theta \\
      &= 2r \left[ -2\cos(\theta/2) \right]_0^{2\pi} \\
      &= -4r [\cos(\pi) - \cos(0)] \\
      &= -4r [-1 - 1] = -4r(-2) = 8r
    \end{align*}
    This is a famous result: the length of one arch of a cycloid is exactly 8 times the radius of the generating circle.
\end{enumerate}

\subsection{Vector Notation and Physics Concepts}
The homework problems use separate equations for $x$ and $y$. In physics and higher math, it's common to use vector notation.
\[ \vec{r}(t) = \langle x(t), y(t) \rangle \]
\begin{itemize}
    \item The \textbf{velocity vector} is $\vec{v}(t) = \vec{r}'(t) = \langle x'(t), y'(t) \rangle$.
    \item The \textbf{speed} is the magnitude of velocity: $|\vec{v}(t)| = \sqrt{(x'(t))^2 + (y'(t))^2}$.
    \item The \textbf{acceleration vector} is $\vec{a}(t) = \vec{v}'(t) = \langle x''(t), y''(t) \rangle$.
\end{itemize}

\subsubsection{Worked Example: Velocity and Acceleration}
For the curve $x(t) = 3\cos(t), y(t) = 3\sin(t)$ (a circle of radius 3), find the velocity and acceleration vectors at $t=\pi/2$. Show that the acceleration vector is orthogonal to the velocity vector for all $t$.
\begin{enumerate}
    \item \textbf{Find Velocity and Acceleration Vectors:}
    \[ \vec{v}(t) = \langle -3\sin(t), 3\cos(t) \rangle \]
    \[ \vec{a}(t) = \langle -3\cos(t), -3\sin(t) \rangle \]
    \item \textbf{Evaluate at $t=\pi/2$:}
    \[ \vec{v}(\pi/2) = \langle -3\sin(\pi/2), 3\cos(\pi/2) \rangle = \langle -3, 0 \rangle \]
    \[ \vec{a}(\pi/2) = \langle -3\cos(\pi/2), -3\sin(\pi/2) \rangle = \langle 0, -3 \rangle \]
    At the top of the circle, the particle is moving left and accelerating downwards (towards the center).
    \item \textbf{Check for Orthogonality:} Two vectors are orthogonal if their dot product is zero.
    \begin{align*}
    \vec{v}(t) \cdot \vec{a}(t) &= (-3\sin t)(-3\cos t) + (3\cos t)(-3\sin t) \\
    &= 9\sin t \cos t - 9\sin t \cos t = 0
    \end{align*}
    Since the dot product is always zero, the velocity and acceleration vectors are always orthogonal for uniform circular motion.
\end{enumerate}

\newpage
\section{Advanced Diagnostic Testing: "Find the Flaw"}
This section contains five new problems with complete solutions. However, each solution contains one subtle but critical error. Your task is to find the flaw, explain it, and provide the correct solution.

\subsection{Problem 1: Concavity}
\paragraph{Question:} For the curve $x=t^2, y=t^3-3t$, find $\frac{d^2y}{dx^2}$ and determine where the curve is concave up.

\paragraph{Flawed Solution:}
\begin{enumerate}
    \item First, we find the first and second derivatives of $x(t)$ and $y(t)$ with respect to $t$.
    \begin{align*}
    x'(t) &= 2t, \quad x''(t) = 2 \\
    y'(t) &= 3t^2 - 3, \quad y''(t) = 6t
    \end{align*}
    \item Next, we find the slope, $\frac{dy}{dx}$.
    \[ \frac{dy}{dx} = \frac{y'(t)}{x'(t)} = \frac{3t^2-3}{2t} \]
    \item To find the second derivative, we find the ratio of the second derivatives with respect to $t$.
    \[ \frac{d^2y}{dx^2} = \frac{y''(t)}{x''(t)} = \frac{6t}{2} = 3t \]
    \item The curve is concave up when $\frac{d^2y}{dx^2} > 0$.
    \[ 3t > 0 \implies t > 0 \]
\end{enumerate}
\textbf{Final Answer from Flawed Solution:} The curve is concave up for $t > 0$.

\subsection{Problem 2: Arc Length}
\paragraph{Question:} Find the exact length of the curve $x = e^t + e^{-t}, y = 5 - 2t$ for $0 \le t \le 3$.

\paragraph{Flawed Solution:}
\begin{enumerate}
    \item Find the derivatives with respect to $t$.
    \begin{align*}
    \frac{dx}{dt} &= e^t - e^{-t} \\
    \frac{dy}{dt} &= -2
    \end{align*}
    \item Set up the arc length integrand.
    \begin{align*}
    \left(\frac{dx}{dt}\right)^2 + \left(\frac{dy}{dt}\right)^2 &= (e^t - e^{-t})^2 + (-2)^2 \\
    &= (e^{2t} - 2e^t e^{-t} + e^{-2t}) + 4 \\
    &= e^{2t} - 2 + e^{-2t} + 4 \\
    &= e^{2t} + 2 + e^{-2t} \\
    &= (e^t + e^{-t})^2
    \end{align*}
    \item Set up and evaluate the integral.
    \begin{align*}
    L &= \int_0^3 \sqrt{(e^t + e^{-t})^2} \, dt \\
      &= \int_0^3 (e^t + e^{-t}) \, dt \\
      &= \left[ e^t + e^{-t} \right]_0^3 \\
      &= (e^3 + e^{-3}) - (e^0 + e^0) \\
      &= e^3 + e^{-3} - 2
    \end{align*}
\end{enumerate}
\textbf{Final Answer from Flawed Solution:} The arc length is $e^3 + e^{-3} - 2$.

\subsection{Problem 3: Area of a Loop}
\paragraph{Question:} Find the area of the region enclosed by the loop of the curve $x=t^2-1, y=t^3-t$.

\paragraph{Flawed Solution:}
\begin{enumerate}
    \item To find the loop, we find where the curve intersects itself. We need to find $t_1 \neq t_2$ such that $x(t_1)=x(t_2)$ and $y(t_1)=y(t_2)$.
    $t_1^2-1 = t_2^2-1 \implies t_1^2 = t_2^2 \implies t_1 = -t_2$.
    $t_1^3-t_1 = t_2^3-t_2$. Substitute $t_2 = -t_1$:
    $t_1^3-t_1 = (-t_1)^3 - (-t_1) = -t_1^3 + t_1 \implies 2t_1^3 - 2t_1 = 0 \implies 2t_1(t_1^2-1)=0$.
    This gives $t_1 = 0, 1, -1$. If $t_1 = 1$, then $t_2 = -1$. So the loop corresponds to the interval $t \in [-1, 1]$.
    \item Find $x'(t)$: $x'(t) = 2t$.
    \item Set up the area integral.
    \[ A = \int_{-1}^1 y(t) \, dt = \int_{-1}^1 (t^3 - t) \, dt \]
    \item Evaluate the integral.
    \[ A = \left[ \frac{t^4}{4} - \frac{t^2}{2} \right]_{-1}^1 = \left(\frac{1}{4} - \frac{1}{2}\right) - \left(\frac{1}{4} - \frac{1}{2}\right) = 0 \]
    Since area cannot be 0, we must have made a mistake. The area of the top half must cancel the bottom. We can integrate from -1 to 0 and double it.
    \[ A = 2 \int_{-1}^0 (t^3-t) \, dt = 2 \left[ \frac{t^4}{4} - \frac{t^2}{2} \right]_{-1}^0 = 2 \left( 0 - (\frac{1}{4} - \frac{1}{2}) \right) = 2(\frac{1}{4}) = \frac{1}{2} \]
\end{enumerate}
\textbf{Final Answer from Flawed Solution:} The area is $\frac{1}{2}$.

\subsection{Problem 4: Tangent Line}
\paragraph{Question:} Find the equation of the tangent line to the curve $x=4\cos\theta, y=2\sin\theta$ at the point $(2\sqrt{2}, \sqrt{2})$.

\paragraph{Flawed Solution:}
\begin{enumerate}
    \item First, find the value of $\theta$.
    $x(\theta) = 4\cos\theta = 2\sqrt{2} \implies \cos\theta = \frac{\sqrt{2}}{2}$.
    $y(\theta) = 2\sin\theta = \sqrt{2} \implies \sin\theta = \frac{\sqrt{2}}{2}$.
    The value of $\theta$ that satisfies both is $\theta = \pi/4$.
    \item Next, find the derivatives.
    \begin{align*}
    \frac{dx}{d\theta} &= -4\sin\theta \\
    \frac{dy}{d\theta} &= 2\cos\theta
    \end{align*}
    \item Now, find the slope of the tangent line.
    \[ m = \frac{dy}{dx} = \frac{2\cos\theta}{-4\sin\theta} = -\frac{1}{2}\cot\theta \]
    \item Evaluate the slope at the point $(2\sqrt{2}, \sqrt{2})$.
    \[ m = -\frac{1}{2} \cot(2\sqrt{2}) \]
    This value must be calculated with a calculator.
    \item Use the point-slope formula.
    \[ y - \sqrt{2} = -\frac{1}{2}\cot(2\sqrt{2}) (x - 2\sqrt{2}) \]
\end{enumerate}
\textbf{Final Answer from Flawed Solution:} The tangent line is $y - \sqrt{2} = -\frac{1}{2}\cot(2\sqrt{2}) (x - 2\sqrt{2})$.

\subsection{Problem 5: Surface Area}
\paragraph{Question:} Set up the integral for the area of the surface generated by rotating the curve $x = t^2, y = t$ for $0 \le t \le 2$ about the y-axis.

\paragraph{Flawed Solution:}
\begin{enumerate}
    \item The formula for surface area rotating about the y-axis is $S = \int 2\pi y \, ds$.
    \item We need to find $ds = \sqrt{(x')^2 + (y')^2} \, dt$.
    \begin{align*}
    x'(t) &= 2t \\
    y'(t) &= 1
    \end{align*}
    So, $ds = \sqrt{(2t)^2 + 1^2} \, dt = \sqrt{4t^2+1} \, dt$.
    \item Substitute into the formula. The radius of rotation is $y=t$.
    \[ S = \int_0^2 2\pi (t) \sqrt{4t^2+1} \, dt \]
\end{enumerate}
\textbf{Final Answer from Flawed Solution:} The integral for the surface area is $S = \int_0^2 2\pi t \sqrt{4t^2+1} \, dt$.

\end{document}