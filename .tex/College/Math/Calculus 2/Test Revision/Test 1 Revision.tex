\documentclass{article}
\usepackage{graphicx} % Required for inserting images
\usepackage{amsmath}  % Required for advanced math environments like align*
\usepackage{amssymb}  % For more math symbols
\usepackage{amsfonts}

% --- GEOMETRY PACKAGE FOR PAGE LAYOUT ---
\usepackage[
    left=1in,
    textwidth=7in, 
    top=1in,
    bottom=1in
]{geometry}
% -----------------------------------------

\title{Test 1 Revision}
\author{Tashfeen Omran}
\date{September 2025}

\begin{document}

\maketitle

\section{Homework 5.5 The Substitution Rule}

\subsection*{Question 1}
Evaluate the integral by making the given substitution. $\int \cos(2x) \,dx$, $u = 2x$.
\subsubsection*{Answer}
$ \frac{1}{2}\sin(2x) + C $

\subsection*{Question 2}
Evaluate the integral by making the given substitution. $\int xe^{-x^2} \,dx$, $u = -x^2$.
\subsubsection*{Answer}
$ -\frac{1}{2}e^{-x^2} + C $

\subsection*{Question 3}
Evaluate the integral by making the given substitution. $\int x^2\sqrt{x^3 + 9} \,dx$, $u = x^3 + 9$.
\subsubsection*{Answer}
$ \frac{2}{9}(x^3 + 9)^{3/2} + C $

\subsection*{Question 4}
Evaluate the integral by making the given substitution. $\int \sin^4(\theta)\cos(\theta) \,d\theta$, $u = \sin(\theta)$.
\subsubsection*{Answer}
$ \frac{1}{5}\sin^5(\theta) + C $

\subsection*{Question 5}
Evaluate the integral by making the given substitution. $\int \frac{x^4}{x^5 - 3} \,dx$, $u = x^5 - 3$.
\subsubsection*{Answer}
$ \frac{1}{5}\ln|x^5 - 3| + C $

\subsection*{Question 6}
Evaluate the integral by making the given substitution. $\int \frac{\cos(\sqrt{t})}{\sqrt{t}} \,dt$, $u = \sqrt{t}$.
\subsubsection*{Answer}
$ 2\sin(\sqrt{t}) + C $

\subsection*{Question 7}
Evaluate the indefinite integral. $\int t^4 e^{-t^5} \,dt$.
\subsubsection*{Answer}
$ -\frac{1}{5}e^{-t^5} + C $

\subsection*{Question 8}
Evaluate the indefinite integral. $\int \sin(t)\sqrt{1 + \cos(t)} \,dt$.
\subsubsection*{Answer}
$ -\frac{2}{3}(1 + \cos(t))^{3/2} + C $

\subsection*{Question 9}
Evaluate the indefinite integral. $\int y^2(5-y^3)^{2/3} \,dy$.
\subsubsection*{Answer}
$ -\frac{1}{5}(5 - y^3)^{5/3} + C $

\subsection*{Question 10}
Evaluate the indefinite integral. $\int \frac{\sin(\frac{1}{x^3})}{x^4} \,dx$.
\subsubsection*{Answer}
$ \frac{1}{3}\cos(\frac{1}{x^3}) + C $

\subsection*{Question 11}
Evaluate the indefinite integral. $\int \frac{(\ln(x))^{28}}{x} \,dx$.
\subsubsection*{Answer}
$ \frac{1}{29}(\ln(x))^{29} + C $

\subsection*{Question 12}
Evaluate the indefinite integral. $\int \sin(22x)\sin(\cos(22x)) \,dx$.
\subsubsection*{Answer}
$ \frac{1}{22}\cos(\cos(22x)) + C $

\subsection*{Question 13}
Evaluate the indefinite integral. $\int \sec^2(\theta)\tan^9(\theta) \,d\theta$.
\subsubsection*{Answer}
$ \frac{1}{10}\tan^{10}(\theta) + C $

\subsection*{Question 14}
Evaluate the indefinite integral. $\int x\sqrt{x+5} \,dx$.
\subsubsection*{Answer}
$ \frac{2}{5}(x+5)^{5/2} - \frac{10}{3}(x+5)^{3/2} + C $

\subsection*{Question 15}
Evaluate the indefinite integral. $\int \sqrt[6]{\cot(x)}\csc^2(x) \,dx$.
\subsubsection*{Answer}
$ -\frac{6}{7}(\cot(x))^{7/6} + C $

\subsection*{Question 16}
Evaluate the indefinite integral. $\int \frac{\sin(2x)}{45 + \cos^2(x)} \,dx$.
\subsubsection*{Answer}
$ -\ln(45 + \cos^2(x)) + C $

\newpage
\section{Homework 6.1 Area Between Curves}

\subsection*{Question 1}
Set up an integral for the area of the shaded region. Evaluate the integral to find the area of the shaded region bounded by $y=5x-x^2$ and $y=x$.
\subsubsection*{Answer}
$ \frac{32}{3} $

\subsection*{Question 2}
Set up an integral for the area of the shaded region. Evaluate the integral to find the area of the shaded region bounded by $y=e^x$ and $y=x^6$ from $x=0$ to $x=1$.
\subsubsection*{Answer}
$ e - \frac{8}{7} $

\subsection*{Question 3}
Set up an integral for the area of the shaded region. Evaluate the integral to find the area of the shaded region bounded by $x=y^2-5$ and $x=e^y$ from $y=-1$ to $y=1$.
\subsubsection*{Answer}
$ e - \frac{1}{e} + \frac{28}{3} $

\subsection*{Question 4}
Set up an integral for the area of the shaded region. Evaluate the integral to find the area of the shaded region bounded by $x=y^2-4y$ and $x=2y-y^2$.
\subsubsection*{Answer}
$ 9 $

\subsection*{Question 5}
Find the area of the shaded region bounded by $y=x^3-15x$ and $y=x$.
\subsubsection*{Answer}
$ 128 $

\subsection*{Question 6}
Find the area of the shaded region bounded by $y=x^2$, $y=8-2x$, and $y = \frac{2}{3}x + \frac{16}{3}$.
\subsubsection*{Answer}
$ \frac{44}{3} $

\subsection*{Question 7}
Set up an integral representing the area A of the region enclosed by the given curves: $x=y^4$, $x=2-y^2$.
\subsubsection*{Answer}
$ \int_{-1}^{1} (2-y^2-y^4) \,dy $

\subsection*{Question 8}
Sketch the region enclosed by the given curves and find the area: $y=3+x^3$, $y=5-x$, $x=-1$, $x=0$.
\subsubsection*{Answer}
$ \frac{11}{4} $

\subsection*{Question 9}
Sketch the region enclosed by the given curves and find the area: $y=4\cos(x)$, $y=4e^x$, $x=\frac{\pi}{2}$.
\subsubsection*{Answer}
$ 4e^{\pi/2} - 8 $

\subsection*{Question 10}
Sketch the region enclosed by the given curves and find the area: $y=x^2-4x$, $y=4x$.
\subsubsection*{Answer}
$ \frac{256}{3} $

\subsection*{Question 11}
Sketch the region enclosed by the given curves and find the area: $x=4-y^2$, $x=y^2-4$.
\subsubsection*{Answer}
$ \frac{64}{3} $

\subsection*{Question 12}
Sketch the region enclosed by the given curves and find the area: $2x+y^2=8$, $x=y$.
\subsubsection*{Answer}
$ 18 $

\subsection*{Question 13}
Sketch the region enclosed by the given curves and find the area: $x=8y^2$, $x=28+y^2$.
\subsubsection*{Answer}
$ \frac{224}{3} $

\subsection*{Question 14}
Sketch the region enclosed by the given curves and find the area: $y=\sqrt{x}$, $y=\frac{1}{5}x$, $0 \le x \le 36$.
\subsubsection*{Answer}
$ \frac{409}{15} $

\subsection*{Question 15}
Sketch the region enclosed by the given curves and find the area: $y=\cos(x)$, $y=2-\cos(x)$, $0 \le x \le 2\pi$.
\subsubsection*{Answer}
$ 4\pi $

\subsection*{Question 16}
Sketch the region enclosed by the given curves and find the area: $y=\cos(x)$, $y=\sin(2x)$, $0 \le x \le \frac{\pi}{2}$.
\subsubsection*{Answer}
$ \frac{1}{2} $

\newpage
\section{Homework 6.2 Volumes}

\subsection*{Question 1}
Set up, but do not evaluate, an integral for the volume of the solid obtained by rotating the region bounded by the given curves about the specified line. $y=\ln(x)$, $y=0$, $x=2$; about the x-axis.
\subsubsection*{Answer}
$ V = \int_1^2 \pi(\ln(x))^2 \,dx $

\subsection*{Question 2}
Set up, but do not evaluate, an integral for the volume of the solid obtained by rotating the region bounded by the given curves about the specified line. $x=\sqrt{6-y}$, $y=0$, $x=0$; about the y-axis.
\subsubsection*{Answer}
$ V = \int_0^6 \pi(\sqrt{6-y})^2 \,dy $

\subsection*{Question 3}
Find the volume V of the solid obtained by rotating the region bounded by the given curves about the specified line. $y=x+1$, $y=0$, $x=0$, $x=2$; about the x-axis.
\subsubsection*{Answer}
$ V = \frac{26\pi}{3} $

\subsection*{Question 4}
Find the volume V of the solid obtained by rotating the region bounded by the given curves about the specified line. $y=\sqrt{x-1}$, $y=0$, $x=5$; about the x-axis.
\subsubsection*{Answer}
$ V = 8\pi $

\subsection*{Question 5}
Consider the solid obtained by rotating the region bounded by the given curves about the specified line. $y=\sqrt{x-1}$, $y=0$, $x=6$; about the x-axis.
\subsubsection*{Answer}
$ V = \frac{25\pi}{2} $

\subsection*{Question 6}
Find the volume V of the solid obtained by rotating the region bounded by the given curves about the specified line. $y=e^x$, $y=0$, $x=-2$, $x=2$; about the x-axis.
\subsubsection*{Answer}
$ V = \frac{\pi}{2}(e^4 - e^{-4}) $

\subsection*{Question 7}
Find the volume V of the solid obtained by rotating the region bounded by the given curves about the specified line. $x=6\sqrt[3]{y}$, $x=0$, $y=3$; about the y-axis.
\subsubsection*{Answer}
$ V = 486\pi $

\subsection*{Question 8}
Find the volume V of the solid obtained by rotating the region bounded by the given curves about the specified line. $y=x^2$, $y=4x$; about the y-axis.
\subsubsection*{Answer}
$ V = \frac{128\pi}{3} $

\subsection*{Question 9}
Find the volume of the solid obtained by rotating the region bounded by the given curves about the specified line. $y=x^3$, $y=\sqrt{x}$; about the x-axis.
\subsubsection*{Answer}
$ V = \frac{5\pi}{14} $

\subsection*{Question 10}
Find the volume V of the solid obtained by rotating the region bounded by the given curves about the specified line. $y=x^2$, $x=y^2$; about $y=1$.
\subsubsection*{Answer}
$ V = \frac{11\pi}{30} $

\subsection*{Question 11}
Find the volume V of the solid obtained by rotating the region bounded by the given curves about the specified line. $x=y^2$, $x=1-y^2$; about $x=5$.
\subsubsection*{Answer}
$ V = 6\sqrt{2}\pi $

\subsection*{Question 12}
Find the volume generated by rotating the region about the specified line. Region $R_1$ bounded by $y=2x$, $y=0$, $x=1$ about the line $y=2$ (BC).
\subsubsection*{Answer}
$ V = \frac{8\pi}{3} $

\subsection*{Question 13}
Find the volume generated by rotating the region about the specified line. Region $R_2$ bounded by $y=2\sqrt{x}$, $y=2$, $x=0$ about the line $x=1$ (AB).
\subsubsection*{Answer}
$ V = \frac{26\pi}{45} $

\subsection*{Question 14}
Find the volume V of the solid obtained by rotating the region bounded by the given curves about the specified line. $y=8x^3$, $y=0$, $x=1$; about $x=2$.
\subsubsection*{Answer}
$ V = \frac{24\pi}{5} $

\subsection*{Question 15}
Find the volume V of the solid obtained by rotating the region bounded by the given curves about the specified line. $y=5\sin(x)$, $y=5\cos(x)$, $0 \le x \le \frac{\pi}{4}$; about $y=-1$.
\subsubsection*{Answer}
$ (\frac{5}{2} + 10\sqrt{2})\pi $

\newpage
\section{Homework 6.3 Volumes by Cylindrical Shells}

\subsection*{Question 1}
Use the method of cylindrical shells to find the volume V of S, the solid obtained by rotating the region bounded by $y=4x(x-1)^2$ about the y-axis.
\subsubsection*{Answer}
$ V = \frac{4\pi}{15} $

\subsection*{Question 2}
Set up and evaluate an integral using the method of cylindrical shells for the volume of the solid obtained by rotating the region bounded by $y=\sqrt{x}$ and $y=6-x$ about the x-axis.
\subsubsection*{Answer}
$ V = \frac{32\pi}{3} $

\subsection*{Question 3}
Set up, but do not evaluate, an integral for the volume of the solid obtained by rotating the region bounded by $y=\ln(x)$, $y=0$, $x=9$ about the y-axis.
\subsubsection*{Answer}
$ V = \int_1^9 2\pi x \ln(x) \,dx $

\subsection*{Question 4}
Set up, but do not evaluate, an integral for the volume of the solid obtained by rotating the region bounded by $y=x^3$, $y=27$, $x=0$ about the x-axis.
\subsubsection*{Answer}
$ V = \int_0^{27} 2\pi y \sqrt[3]{y} \,dy $

\subsection*{Question 5}
Use the method of cylindrical shells to find the volume generated by rotating the region bounded by $y=\sqrt{x}$, $y=0$, $x=9$ about the y-axis.
\subsubsection*{Answer}
$ V = \frac{972\pi}{5} $

\subsection*{Question 6}
Use the method of cylindrical shells to find the volume generated by rotating the region bounded by $y=x^3$, $y=0$, $x=1$, $x=3$ about the y-axis.
\subsubsection*{Answer}
$ V = \frac{484\pi}{5} $

\subsection*{Question 7}
Use the method of cylindrical shells to find the volume generated by rotating the region bounded by $y=8x-x^2$ and $y=x$ about the y-axis.
\subsubsection*{Answer}
$ V = \frac{2401\pi}{6} $

\subsection*{Question 8}
Use the method of cylindrical shells to find the volume of the solid obtained by rotating the region bounded by $xy=2$, $x=0$, $y=2$, $y=4$ about the x-axis.
\subsubsection*{Answer}
$ V = 8\pi $

\subsection*{Question 9}
Use the method of cylindrical shells to find the volume of the solid obtained by rotating the region bounded by $y=\sqrt{x}$, $x=0$, $y=5$ about the x-axis.
\subsubsection*{Answer}
$ V = \frac{625\pi}{2} $

\subsection*{Question 10}
Use the method of cylindrical shells to find the volume of the solid obtained by rotating the region bounded by $y=x^{3/2}$, $y=8$, $x=0$ about the x-axis.
\subsubsection*{Answer}
$ V = 192\pi $

\subsection*{Question 11}
Use the method of cylindrical shells to find the volume of the solid obtained by rotating the region bounded by $x=2+(y-4)^2$ and $x=3$ about the x-axis.
\subsubsection*{Answer}
$ V = \frac{32\pi}{3} $

\subsection*{Question 12}
Use the method of cylindrical shells to find the volume generated by rotating the region bounded by $y=x^3$, $y=8$, $x=0$ about the line $x=7$.
\subsubsection*{Answer}
$ V = \frac{744\pi}{5} $

\subsection*{Question 13}
Use the method of cylindrical shells to find the volume generated by rotating the region bounded by $y=5x-x^2$ and $y=4$ about the line $x=1$.
\subsubsection*{Answer}
$ V = \frac{27\pi}{2} $

\subsection*{Question 14}
Use the method of cylindrical shells to find the volume generated by rotating the region bounded by $x=7y^2$, $y \ge 0$, $x=7$ about the line $y=2$.
\subsubsection*{Answer}
$ V = \frac{91\pi}{6} $

\subsection*{Question 15}
Use the method of cylindrical shells to find the volume generated by rotating the region bounded by $x=2y^2$ and $x=y^2+16$ about the line $y=-17$.
\subsubsection*{Answer}
$ V = \frac{8704\pi}{3} $

\subsection*{Question 16}
Find the volume of the solid obtained by rotating the shaded region bounded by $y=8-6x^2$ and $y=2x^2$ about the x-axis.
\subsubsection*{Answer}
$ V = \frac{192\pi}{5} $

\subsection*{Question 17}
A solid is obtained by rotating the shaded region bounded by $x=6y-y^2$ and $x=5$ about the line $y=6$.
\subsubsection*{Answer}
$ V = 64\pi $

\subsection*{Question 18}
The region bounded by $x^2 + (y-2)^2 = 4$ is rotated about the y-axis. Find the volume of the resulting solid by any method.
\subsubsection*{Answer}
$ V = \frac{32\pi}{3} $

\newpage
\section{Homework 7.1 Integration by Parts}

\subsection*{Question 1}
Evaluate the integral using integration by parts with the indicated choices of u and dv. $\int xe^{6x} \,dx$; $u=x$, $dv=e^{6x}\,dx$.
\subsubsection*{Answer}
$ \frac{1}{6}xe^{6x} - \frac{1}{36}e^{6x} + C $

\subsection*{Question 2}
Evaluate the integral using integration by parts with the indicated choices of u and dv. $\int x\cos(9x) \,dx$; $u=x$, $dv=\cos(9x)\,dx$.
\subsubsection*{Answer}
$ \frac{1}{9}x\sin(9x) + \frac{1}{81}\cos(9x) + C $

\subsection*{Question 3}
Evaluate the integral. $\int w\ln(w) \,dw$.
\subsubsection*{Answer}
$ \frac{1}{2}w^2\ln(w) - \frac{1}{4}w^2 + C $

\subsection*{Question 4}
Evaluate the integral. $\int \frac{\ln(x)}{x^2} \,dx$.
\subsubsection*{Answer}
$ -\frac{\ln(x)}{x} - \frac{1}{x} + C $

\subsection*{Question 5}
Evaluate the integral. $\int \ln(\sqrt{x}) \,dx$.
\subsubsection*{Answer}
$ x\ln(\sqrt{x}) - \frac{x}{2} + C $

\subsection*{Question 6}
Evaluate the integral. $\int e^{8\theta}\sin(9\theta) \,d\theta$.
\subsubsection*{Answer}
$ \frac{1}{145}e^{8\theta}(8\sin(9\theta) - 9\cos(9\theta)) + C $

\subsection*{Question 7}
Evaluate the integral. $\int e^{3\theta}\sin(4\theta) \,d\theta$.
\subsubsection*{Answer}
$ \frac{1}{25}e^{3\theta}(3\sin(4\theta) - 4\cos(4\theta)) + C $

\subsection*{Question 8}
Evaluate the integral. $\int_0^{2\pi} x\sin(x)\cos(x) \,dx$.
\subsubsection*{Answer}
$ -\frac{\pi}{2} $

\subsection*{Question 9}
Evaluate the integral. $\int_0^t 7e^s \sin(t-s) \,ds$.
\subsubsection*{Answer}
$ \frac{7}{2}(e^t - \sin(t) - \cos(t)) $

\subsection*{Question 10}
First make a substitution and then use integration by parts to evaluate the integral. $\int_0^\pi e^{\cos(t)}\sin(2t) \,dt$.
\subsubsection*{Answer}
$ \frac{4}{e} $

\subsection*{Question 11}
Use integration by parts to prove the reduction formula. $\int (\ln(x))^n \,dx = x(\ln x)^n - n\int (\ln x)^{n-1} \,dx$.
\subsubsection*{Proof}
Let $u = (\ln x)^n$ and $dv = dx$. Then $du = n(\ln x)^{n-1}\frac{1}{x}dx$ and $v=x$.
$ \int (\ln(x))^n \,dx = x(\ln x)^n - \int x \cdot n(\ln x)^{n-1}\frac{1}{x} \,dx = x(\ln x)^n - n\int (\ln x)^{n-1} \,dx $.

\subsection*{Question 12}
Use integration by parts to determine which of the reduction formulas is correct.
\subsubsection*{Answer}
$ \int 9x^n e^x \,dx = 9x^n e^x - 9n \int x^{n-1}e^x \,dx $

\subsection*{Question 13}
Use integration by parts to determine which of the reduction formulas is correct.
\subsubsection*{Answer}
$ \int 2\tan^n(x) \,dx = 2\frac{\tan^{n-1}(x)}{n-1} - \int 2\tan^{n-2}(x) \,dx, (n \neq 1) $

\subsection*{Question 14}
Evaluate the integral using integration by parts with the indicated choices of u and dv. $\int xe^{9x} \,dx$; $u=x$, $dv=e^{9x}\,dx$.
\subsubsection*{Answer}
$ \frac{1}{9}xe^{9x} - \frac{1}{81}e^{9x} + C $

\subsection*{Question 15}
Evaluate the integral. $\int te^{8t} \,dt$.
\subsubsection*{Answer}
$ \frac{1}{8}te^{8t} - \frac{1}{64}e^{8t} + C $

\subsection*{Question 16}
Evaluate the integral. $\int (\pi-x)\cos(\pi x) \,dx$.
\subsubsection*{Answer}
$ \frac{1}{\pi}(\pi-x)\sin(\pi x) - \frac{1}{\pi^2}\cos(\pi x) + C $

\newpage
\section{Homework 7.2 Trigonometric Integrals}

\subsection*{Question 1}
Evaluate the integral. $\int 2\sin^2(x)\cos^3(x) \,dx$.
\subsubsection*{Answer}
$ \frac{2}{3}\sin^3(x) - \frac{2}{5}\sin^5(x) + C $

\subsection*{Question 2}
Evaluate the integral. $\int \sin^3(y)\cos^4(y) \,dy$.
\subsubsection*{Answer}
$ -\frac{1}{5}\cos^5(y) + \frac{1}{7}\cos^7(y) + C $

\subsection*{Question 3}
Evaluate the integral. $\int_0^{\pi/2} \cos^{13}(x)\sin^5(x) \,dx$.
\subsubsection*{Answer}
$ \frac{1}{504} $

\subsection*{Question 4}
Evaluate the integral. $\int_0^{\pi/2} 9\sin^2(x)\cos^2(x) \,dx$.
\subsubsection*{Answer}
$ \frac{9\pi}{16} $

\subsection*{Question 5}
Evaluate the integral. $\int_0^{\pi/2} 5\cos^2(\theta) \,d\theta$.
\subsubsection*{Answer}
$ \frac{5\pi}{4} $

\subsection*{Question 6}
Evaluate the integral. $\int \sqrt{\cos(\theta)}\sin^3(\theta) \,d\theta$.
\subsubsection*{Answer}
$ \frac{2}{7}(\cos\theta)^{7/2} - \frac{2}{3}(\cos\theta)^{3/2} + C $

\subsection*{Question 7}
Evaluate the integral. $\int \sin(3x)\sec^5(3x) \,dx$.
\subsubsection*{Answer}
$ \frac{1}{12}\sec^4(3x) + C $

\subsection*{Question 8}
Evaluate the integral. $\int 4\tan(x)\sec^3(x) \,dx$.
\subsubsection*{Answer}
$ \frac{4}{3}\sec^3(x) + C $

\subsection*{Question 9}
Evaluate the integral. $\int 5\tan^2(x) \,dx$.
\subsubsection*{Answer}
$ 5\tan(x) - 5x + C $

\subsection*{Question 10}
Evaluate the integral. $\int 11\tan^4(x)\sec^6(x) \,dx$.
\subsubsection*{Answer}
$ \frac{11}{9}\tan^9(x) + \frac{22}{7}\tan^7(x) + \frac{11}{5}\tan^5(x) + C $

\subsection*{Question 11}
Evaluate the integral. $\int \tan^3(x)\sec(x) \,dx$.
\subsubsection*{Answer}
$ \frac{1}{3}\sec^3(x) - \sec(x) + C $

\subsection*{Question 12}
Evaluate the integral. $\int \tan^5(x)\sec^6(x) \,dx$.
\subsubsection*{Answer}
$ \frac{1}{8}\tan^8(x) + \frac{1}{3}\tan^6(x) + \frac{1}{4}\tan^4(x) + C $

\subsection*{Question 13}
Evaluate the integral. $\int_0^{\pi/6} \tan^4(t) \,dt$.
\subsubsection*{Answer}
$ \frac{\pi}{6} - \frac{8}{9\sqrt{3}} $

\subsection*{Question 14}
Evaluate the integral. $\int \tan^5(x) \,dx$.
\subsubsection*{Answer}
$ \frac{1}{4}\sec^4(x) - \tan^2(x) + \ln|\sec(x)| + C $ (Note: Answer in image is equivalent: $\frac{1}{4}\tan^4(x) - \frac{1}{2}\tan^2(x) + \ln|\sec x| + C$ is also common)

\subsection*{Question 15}
Evaluate the integral. $\int \frac{\tan(x)\sec^2(x)}{\cos(x)} \,dx$.
\subsubsection*{Answer}
$ \frac{1}{3}\sec^3(x) + C $

\subsection*{Question 16}
Evaluate the integral. $\int_{\pi/6}^{\pi/2} 5\cot^2(x) \,dx$.
\subsubsection*{Answer}
$ 5\sqrt{3} - \frac{5\pi}{3} $

\subsection*{Question 17}
Evaluate the integral. $\int \sin(8x)\cos(5x) \,dx$.
\subsubsection*{Answer}
$ -\frac{1}{26}\cos(13x) - \frac{1}{6}\cos(3x) + C $

\subsection*{Question 18}
Evaluate the integral. $\int 5\tan^2(x)\sec(x) \,dx$.
\subsubsection*{Answer}
$ \frac{5}{2}(\sec(x)\tan(x) - \ln|\sec(x)+\tan(x)|) + C $

\newpage
\section{Homework 7.3 Trigonometric Substitution}

\subsection*{Question 1}
Evaluate the integral. $\int \frac{x}{\sqrt{81+x^2}} \,dx$.
\subsubsection*{Answer}
$ \sqrt{81+x^2} + C $

\subsection*{Question 2}
Evaluate the integral. $\int \frac{x}{\sqrt{x^2-5}} \,dx$.
\subsubsection*{Answer}
$ \sqrt{x^2-5} + C $

\subsection*{Question 3}
Evaluate the integral. $\int_0^3 \sqrt{x^2+9} \,dx$.
\subsubsection*{Answer}
$ \frac{9}{2}(\sqrt{2} + \ln(1+\sqrt{2})) $

\subsection*{Question 4}
Evaluate the integral using the indicated trigonometric substitution. $\int \frac{x^3}{\sqrt{16+x^2}} \,dx$, $x=4\tan(\theta)$.
\subsubsection*{Answer}
$ \frac{1}{3}(x^2+16)^{3/2} - 16\sqrt{x^2+16} + C $

\subsection*{Question 5}
Evaluate the integral using the indicated trigonometric substitution. $\int \frac{\sqrt{4x^2-25}}{x} \,dx$, $x=\frac{5}{2}\sec(\theta)$.
\subsubsection*{Answer}
$ \sqrt{4x^2-25} - 5\sec^{-1}(\frac{2x}{5}) + C $

\subsection*{Question 6}
Determine an appropriate trigonometric substitution for $\int \frac{x^4}{\sqrt{1+x^2}} \,dx$ and transform the integral.
\subsubsection*{Answer}
Substitution: $x=\tan(\theta)$. Transformed Integral: $\int \tan^4(\theta)\sec(\theta) \,d\theta$.

\subsection*{Question 7}
Evaluate the integral. $\int_2^5 \frac{dx}{(x^2-1)^{3/2}}$.
\subsubsection*{Answer}
$ \frac{5\sqrt{6}}{12} + \frac{2\sqrt{3}}{3} $ (Note: The image answer seems simplified incorrectly. It should be $\frac{5}{\sqrt{24}} - \frac{2}{\sqrt{3}} = \frac{5\sqrt{6}}{12} - \frac{2\sqrt{3}}{3}$)

\subsection*{Question 8}
Evaluate the integral. $\int_0^4 \frac{dt}{\sqrt{16+t^2}}$.
\subsubsection*{Answer}
$ \ln(1+\sqrt{2}) $

\subsection*{Question 9}
Evaluate the integral. $\int_0^7 \frac{dt}{\sqrt{49+t^2}}$.
\subsubsection*{Answer}
$ \ln(1+\sqrt{2}) $

\subsection*{Question 10}
Evaluate the integral. $\int \frac{\sqrt{x^2-25}}{x^3} \,dx$.
\subsubsection*{Answer}
$ \frac{1}{10}\sec^{-1}(\frac{x}{5}) - \frac{\sqrt{x^2-25}}{2x^2} + C $

\subsection*{Question 11}
Evaluate the integral. $\int \frac{\sqrt{4+x^2}}{x} \,dx$.
\subsubsection*{Answer}
$ \sqrt{4+x^2} + 2\ln|\frac{\sqrt{x^2+4}-2}{x}| + C $

\subsection*{Question 12}
Evaluate the integral. $\int 3x\sqrt{1-x^4} \,dx$.
\subsubsection*{Answer}
$ \frac{3}{4}\sin^{-1}(x^2) + \frac{3}{4}x^2\sqrt{1-x^4} + C $

\subsection*{Question 13}
Evaluate the integral. $\int x^3\sqrt{64+x^2} \,dx$.
\subsubsection*{Answer}
$ \frac{1}{5}(x^2+64)^{5/2} - \frac{64}{3}(x^2+64)^{3/2} + C $

\subsection*{Question 14}
Evaluate the integral. $\int \frac{x^2}{\sqrt{49-x^2}} \,dx$.
\subsubsection*{Answer}
$ \frac{49}{2}\sin^{-1}(\frac{x}{7}) - \frac{1}{2}x\sqrt{49-x^2} + C $

\newpage
\section{Homework 7.4 integration by fraction decomposition}

\subsection*{Question 1}
Write out the form of the partial fraction decomposition.
(a) $\frac{x-42}{x^2+x-42}$
(b) $\frac{1}{x^2+x^4}$
\subsubsection*{Answer}
(a) $\frac{A}{x+7} + \frac{B}{x-6}$
(b) $\frac{A}{x} + \frac{B}{x^2} + \frac{Cx+D}{x^2+1}$

\subsection*{Question 2}
Write out the form of the partial fraction decomposition.
(a) $\frac{x^5+36}{(x^2-x)(x^4+12x^2+36)}$
(b) $\frac{x^2}{x^2+x-20}$
\subsubsection*{Answer}
(a) $\frac{A}{x} + \frac{B}{x-1} + \frac{Cx+D}{x^2+6} + \frac{Ex+F}{(x^2+6)^2}$
(b) $1 + \frac{A}{x+5} + \frac{B}{x-4}$

\subsection*{Question 3}
Write out the form of the partial fraction decomposition.
(a) $\frac{x^6}{x^2-4}$
(b) $\frac{x^4}{(x^2-x+1)(x^2+4)^2}$
\subsubsection*{Answer}
(a) $x^4+4x^2+16 + \frac{A}{x-2} + \frac{B}{x+2}$
(b) $\frac{Ax+B}{x^2-x+1} + \frac{Cx+D}{x^2+4} + \frac{Ex+F}{(x^2+4)^2}$

\subsection*{Question 4}
Evaluate the integral. $\int \frac{3}{(x-1)(x+2)} \,dx$.
\subsubsection*{Answer}
$ \ln|x-1| - \ln|x+2| + C $

\subsection*{Question 5}
Evaluate the integral. $\int \frac{82x+8}{(9x+1)(x-1)} \,dx$.
\subsubsection*{Answer}
$ \frac{1}{9}\ln|9x+1| + 9\ln|x-1| + C $

\subsection*{Question 6}
Evaluate the integral. $\int \frac{5t-2}{t+1} \,dt$.
\subsubsection*{Answer}
$ 5t - 7\ln|t+1| + C $

\subsection*{Question 7}
Evaluate the integral. $\int \frac{2}{(t^2-4)^2} \,dt$.
\subsubsection*{Answer}
$ \frac{1}{16}\ln|\frac{t-2}{t+2}| - \frac{t}{4(t^2-4)} + C $

\subsection*{Question 8}
Evaluate the integral. $\int \frac{17}{(x-1)(x^2+16)} \,dx$.
\subsubsection*{Answer}
$ \ln|x-1| - \frac{1}{2}\ln(x^2+16) - \frac{1}{4}\arctan(\frac{x}{4}) + C $

\subsection*{Question 9}
Evaluate the integral. $\int \frac{x^2-x+12}{x^3+2x} \,dx$.
\subsubsection*{Answer}
$ 6\ln|x| - \frac{5}{2}\ln(x^2+2) - \frac{\sqrt{2}}{2}\arctan(\frac{x}{\sqrt{2}}) + C $

\subsection*{Question 10}
Evaluate the integral. $\int \frac{5x^2+x+5}{(x^2+1)^2} \,dx$.
\subsubsection*{Answer}
$ 5\arctan(x) - \frac{1}{2(x^2+1)} + C $

\subsection*{Question 11}
Evaluate the integral. $\int \frac{x+12}{x^2+14x+53} \,dx$.
\subsubsection*{Answer}
$ \frac{1}{2}\ln(x^2+14x+53) + \frac{5}{2}\arctan(\frac{x+7}{2}) + C $

\subsection*{Question 12}
Evaluate the integral. $\int_0^1 \frac{x^3+3x}{x^4+6x^2+3} \,dx$.
\subsubsection*{Answer}
$ \frac{1}{4}\ln(\frac{10}{3}) $

\subsection*{Question 13}
Evaluate the integral. $\int_0^1 \frac{2}{2x^2+3x+1} \,dx$.
\subsubsection*{Answer}
$ 2\ln(\frac{4}{3}) $ or $ \ln(\frac{16}{9}) $

\subsection*{Question 14}
Evaluate the integral. $\int \frac{x^2}{x-5} \,dx$.
\subsubsection*{Answer}
$ \frac{1}{2}x^2+5x+25\ln|x-5| + C $





\newpage
\begin{titlepage}
    \centering
    \Huge\bfseries Comprehensive Guide to Integration Problem Types and Tricks \par
    \vspace{1em}
    \vfill
\end{titlepage}

\section{Section 5.5: The Substitution Rule (U-Substitution)}
The core idea of u-substitution is to simplify an integral by replacing a part of the integrand with a single variable, u, effectively reversing the chain rule. The key is to choose u such that its derivative, du, also appears in the integral (perhaps off by a constant).

\subsection{Problem Types and Tricks in Your Homework}
\subsubsection{Problem Type 1: Direct Substitution with Constant Adjustment}
This is the most common type, where the derivative of your chosen u is present in the integral, but might be off by a constant multiplier.
\paragraph{Example (Q2):} $ \int xe^{-x^2} \,dx $
\paragraph{Solution Strategy:}
\begin{itemize}
    \item Identify the "inner function." Here, it's the exponent, $-x^2$.
    \item Let $u = -x^2$.
    \item Find the differential: $du = -2x \,dx$.
    \item Notice the integral has $x \,dx$, not $-2x \,dx$. Algebraically solve for what you have: $x \,dx = -\frac{1}{2}du$.
    \item Substitute both $u$ and the expression for $x \,dx$ into the integral: $ \int e^u (-\frac{1}{2}du) = -\frac{1}{2} \int e^u \,du $.
    \item Integrate with respect to $u$: $-\frac{1}{2}e^u + C$.
    \item Substitute back for $x$: $-\frac{1}{2}e^{-x^2} + C$.
\end{itemize}
\paragraph{Tricks Used:}
\begin{itemize}
    \item \textbf{Identifying the Inner Function:} Recognizing that the derivative of the exponent ($x^2$) is related to the other factor ($x$).
    \item \textbf{Constant Adjustment:} Manipulating the differential ($du = -2x \,dx$) to solve for the parts available in the integrand ($x \,dx$).
\end{itemize}

\subsubsection{Problem Type 2: "Change of Variables" or Back-Substitution}
This occurs when, after substituting for u and du, you still have a variable x left in the integrand. You must then use your original substitution equation to solve for x in terms of u.
\paragraph{Example (Q14):} $ \int x\sqrt{x+5} \,dx $
\paragraph{Solution Strategy:}
\begin{itemize}
    \item The complicated part is the radical, so let $u = x+5$.
    \item The differential is simple: $du = dx$.
    \item Substituting gives: $\int x\sqrt{u} \,du$. We still have an x.
    \item Go back to the substitution equation, $u = x+5$, and solve for $x$: $x = u-5$.
    \item Substitute this back into the integral: $\int (u-5)\sqrt{u} \,du = \int (u-5)u^{1/2} \,du$.
    \item Distribute and integrate using the power rule: $\int (u^{3/2} - 5u^{1/2}) \,du = \frac{2}{5}u^{5/2} - 5(\frac{2}{3})u^{3/2} + C$.
    \item Substitute back for $x$: $\frac{2}{5}(x+5)^{5/2} - \frac{10}{3}(x+5)^{3/2} + C$.
\end{itemize}
\paragraph{Tricks Used:}
\begin{itemize}
    \item \textbf{Solving for x:} Recognizing that the leftover variable must be eliminated by using the original substitution equation.
    \item \textbf{Algebraic Simplification:} Distributing the radical term to create a simple polynomial in u that can be integrated with the power rule.
\end{itemize}

\subsection{Other Common Problem Types and Tricks}
\subsubsection{Problem Type 3: Splitting the Integrand}
Sometimes, an integrand can be split into two parts: one that can be solved with u-substitution and another that has a different form (often leading to an arctangent).
\paragraph{Example:} $ \int \frac{2x+5}{x^2+9} \,dx $
\paragraph{Solution Strategy:}
\begin{itemize}
    \item Split the fraction: $\int \frac{2x}{x^2+9} \,dx + \int \frac{5}{x^2+9} \,dx$.
    \item For the first integral, use u-sub: $u = x^2+9$, $du = 2x \,dx$. This becomes $\int \frac{1}{u} \,du = \ln|u| = \ln(x^2+9)$.
    \item For the second integral, recognize the arctan form: $5 \int \frac{1}{x^2+3^2} \,dx = 5(\frac{1}{3}\arctan(\frac{x}{3}))$.
    \item Combine the results: $\ln(x^2+9) + \frac{5}{3}\arctan(\frac{x}{3}) + C$.
\end{itemize}
\paragraph{Tricks Used:}
\begin{itemize}
    \item \textbf{Fraction Splitting:} A key algebraic step to separate the integral into manageable parts.
    \item \textbf{Recognizing Standard Forms:} Identifying both a ln pattern ($du/u$) and an arctan pattern ($1/(x^2+a^2)$) in the same problem.
\end{itemize}

\subsubsection{Problem Type 4: U-Substitution with Completing the Square (leading to arctan)}
When the denominator is an irreducible quadratic that isn't a simple sum of squares, you must complete the square first.
\paragraph{Example:} $ \int \frac{1}{x^2+6x+13} \,dx $
\paragraph{Solution Strategy:}
\begin{itemize}
    \item Complete the square in the denominator: $x^2+6x+9-9+13 = (x+3)^2 + 4$.
    \item The integral becomes: $\int \frac{1}{(x+3)^2 + 2^2} \,dx$.
    \item Now use u-substitution: Let $u=x+3$, so $du=dx$.
    \item The integral is now in the standard arctan form: $\int \frac{1}{u^2+2^2} \,du = \frac{1}{2}\arctan(\frac{u}{2}) + C$.
    \item Substitute back: $\frac{1}{2}\arctan(\frac{x+3}{2}) + C$.
\end{itemize}
\paragraph{Tricks Used:}
\begin{itemize}
    \item \textbf{Completing the Square:} A fundamental algebraic technique to transform a quadratic into a more useful form.
    \item \textbf{Shifting the Variable:} The substitution $u=x+3$ handles the horizontal shift.
\end{itemize}

\subsubsection{Problem Type 5: U-Substitution with Definite Integrals}
When evaluating a definite integral with u-substitution, you have two options: (1) change the limits of integration to be in terms of u, or (2) integrate, substitute back to x, and then use the original x limits. Changing the limits is usually faster and less error-prone.
\paragraph{Example:} $ \int_0^2 x(x^2+1)^3 \,dx $
\paragraph{Solution Strategy (Changing Limits):}
\begin{itemize}
    \item Let $u = x^2+1$, so $du = 2x \,dx \Rightarrow x \,dx = \frac{1}{2}du$.
    \item Change the limits: When $x=0$, $u = 0^2+1=1$. When $x=2$, $u = 2^2+1=5$.
    \item The new integral is: $\int_1^5 u^3 (\frac{1}{2}du) = \frac{1}{2}\left[ \frac{u^4}{4} \right]_1^5$.
    \item Evaluate with the new limits: $\frac{1}{8}(5^4 - 1^4) = \frac{1}{8}(625-1) = \frac{624}{8} = 78$.
\end{itemize}
\paragraph{Tricks Used:}
\begin{itemize}
    \item \textbf{Changing Integration Limits:} This is the most important trick for definite integrals with substitution. It avoids the need to substitute back to x.
\end{itemize}

\section{Section 7.1: Integration by Parts (IBP)}
The formula is $ \int u \,dv = uv - \int v \,du $. The goal is to choose u and dv such that the new integral, $\int v \,du$, is simpler than the original. The acronym LIATE (Logarithmic, Inverse Trig, Algebraic, Trigonometric, Exponential) is a useful guideline for choosing u.

\subsection{Problem Types and Tricks in Your Homework}
\subsubsection{Problem Type 1: Standard Application (Polynomial x Transcendental)}
This is the classic IBP problem where you differentiate the polynomial part down to a constant.
\paragraph{Example (Q1):} $ \int xe^{6x} \,dx $
\paragraph{Solution Strategy:}
\begin{itemize}
    \item Using LIATE, choose the algebraic part for u: $u=x$.
    \item The rest is dv: $dv = e^{6x}dx$.
    \item Differentiate u and integrate dv: $du=dx$ and $v=\frac{1}{6}e^{6x}$.
    \item Apply the formula: $uv - \int v \,du = x(\frac{1}{6}e^{6x}) - \int \frac{1}{6}e^{6x} \,dx$.
    \item The new integral is simple: $\frac{1}{6}xe^{6x} - \frac{1}{36}e^{6x} + C$.
\end{itemize}
\paragraph{Tricks Used:}
\begin{itemize}
    \item \textbf{LIATE Rule:} A heuristic for choosing u to ensure the new integral is simpler.
\end{itemize}

\subsubsection{Problem Type 2: Integrating Logarithmic or Inverse Trig Functions}
When the integrand is just a log or inverse trig function, there appears to be only one part. The trick is to choose dv = dx.
\paragraph{Example (Q3):} $ \int w \ln(w) \,dw $
\paragraph{Solution Strategy:}
\begin{itemize}
    \item Using LIATE, choose u as the log: $u = \ln(w)$.
    \item The rest is dv: $dv = w \,dw$.
    \item Differentiate and integrate: $du = \frac{1}{w}dw$ and $v = \frac{1}{2}w^2$.
    \item Apply the formula: $uv - \int v \,du = (\ln w)(\frac{1}{2}w^2) - \int \frac{1}{2}w^2 (\frac{1}{w}dw)$.
    \item Simplify and solve the new integral: $\frac{1}{2}w^2\ln(w) - \frac{1}{2}\int w \,dw = \frac{1}{2}w^2\ln(w) - \frac{1}{4}w^2 + C$.
\end{itemize}
\paragraph{Tricks Used:}
\begin{itemize}
    \item \textbf{Strategic dv choice:} Even though the log function is the "L" in LIATE, here it's paired with an algebraic term, making the choice straightforward. For just $\int \ln(x)dx$, the trick is to choose $dv=dx$.
\end{itemize}

\subsubsection{Problem Type 3: Looping Integration by Parts}
This advanced technique is used for integrals of the form $\int e^{ax}\sin(bx) \,dx$ or $\int e^{ax}\cos(bx) \,dx$. Applying IBP twice will result in the original integral appearing on the right side of the equation, allowing you to solve for it algebraically.
\paragraph{Example (Q6):} $ \int e^{8\theta}\sin(9\theta) \,d\theta $
\paragraph{Solution Strategy:}
\begin{itemize}
    \item Let $u = \sin(9\theta)$ and $dv = e^{8\theta}d\theta$. Then $du = 9\cos(9\theta)d\theta$ and $v=\frac{1}{8}e^{8\theta}$.
    \item First IBP: $ I = \frac{1}{8}e^{8\theta}\sin(9\theta) - \frac{9}{8}\int e^{8\theta}\cos(9\theta) \,d\theta $.
    \item Apply IBP to the new integral: Let $U=\cos(9\theta)$ and $dV=e^{8\theta}d\theta$. Then $dU = -9\sin(9\theta)d\theta$ and $V=\frac{1}{8}e^{8\theta}$.
    \item Substitute this back in: $I = \frac{1}{8}e^{8\theta}\sin(9\theta) - \frac{9}{8}\left[\frac{1}{8}e^{8\theta}\cos(9\theta) - \int \frac{1}{8}e^{8\theta}(-9\sin(9\theta)d\theta)\right]$.
    \item Simplify: $I = \frac{1}{8}e^{8\theta}\sin(9\theta) - \frac{9}{64}e^{8\theta}\cos(9\theta) - \frac{81}{64}\int e^{8\theta}\sin(9\theta) \,d\theta$.
    \item Notice the original integral $I$ has reappeared. Substitute $I$: $I = \dots - \frac{81}{64}I$.
    \item Solve for $I$ algebraically: $I + \frac{81}{64}I = \dots \Rightarrow \frac{145}{64}I = \dots \Rightarrow I = \frac{64}{145}[\dots]$.
\end{itemize}
\paragraph{Tricks Used:}
\begin{itemize}
    \item \textbf{Repeating IBP:} Recognizing that a single application isn't enough.
    \item \textbf{The "Boomerang" or "Loop":} Identifying that the original integral has returned.
    \item \textbf{Algebraic Solution:} Treating the integral I as a variable and solving the equation for it.
\end{itemize}

\subsection{Other Common Problem Types and Tricks}
\subsubsection{Problem Type 4: The Tabular Method}
This is a fantastic shortcut for repeated IBP when one function is a polynomial (differentiates to zero) and the other can be repeatedly integrated.
\paragraph{Example:} $ \int x^3 e^{2x} \,dx $
\paragraph{Solution Strategy:}
\begin{itemize}
    \item Create two columns: D (for derivatives) and I (for integrals).
    \item Place the polynomial ($x^3$) in the D column and the other function ($e^{2x}$) in the I column.
    \item Differentiate down the D column until you reach zero. Integrate down the I column the same number of times.
    \item Add alternating signs to the D column (+, -, +, -).
    \item The answer is the sum of the products of the diagonal terms.
    \begin{center}
    \begin{tabular}{| c | c | c |}
    \hline
    \textbf{Signs} & \textbf{D} & \textbf{I} \\ \hline
    + & $x^3$ & $e^{2x}$ \\
    - & $3x^2$ & $\frac{1}{2}e^{2x}$ \\
    + & $6x$ & $\frac{1}{4}e^{2x}$ \\
    - & $6$ & $\frac{1}{8}e^{2x}$ \\
    + & $0$ & $\frac{1}{16}e^{2x}$ \\ \hline
    \end{tabular}
    \end{center}
    \item \textbf{Answer:} $x^3(\frac{1}{2}e^{2x}) - 3x^2(\frac{1}{4}e^{2x}) + 6x(\frac{1}{8}e^{2x}) - 6(\frac{1}{16}e^{2x}) + C$.
\end{itemize}
\paragraph{Tricks Used:}
\begin{itemize}
    \item \textbf{Organizational Shortcut:} The tabular method organizes the repeated applications of IBP, reducing the chance of algebraic errors.
\end{itemize}

\section{Section 7.2: Trigonometric Integrals}
This section focuses on integrals containing powers of trigonometric functions. The strategy depends entirely on the powers (even or odd) and which functions are present. It relies heavily on trigonometric identities.

\subsection{Problem Types and Tricks in Your Homework}
\subsubsection{Problem Type 1: Powers of Sine and Cosine (At least one is ODD)}
If the power of cosine is odd, save one cosine factor and convert the rest to sines. If the power of sine is odd, save one sine factor and convert the rest to cosines.
\paragraph{Example (Q1):} $ \int 2\sin^2(x)\cos^3(x) \,dx $
\paragraph{Solution Strategy:}
\begin{itemize}
    \item The power of cosine (3) is odd.
    \item Split off one cosine factor: $\int 2\sin^2(x)\cos^2(x)\cos(x) \,dx$.
    \item Use the Pythagorean Identity $\cos^2(x) = 1-\sin^2(x)$ to convert the remaining even-powered cosines: $\int 2\sin^2(x)(1-\sin^2(x))\cos(x) \,dx$.
    \item Now use u-substitution. Let $u=\sin(x)$, so $du=\cos(x)dx$.
    \item The integral becomes a simple polynomial: $\int 2u^2(1-u^2) \,du = \int (2u^2-2u^4) \,du$.
    \item Integrate and substitute back: $\frac{2}{3}u^3 - \frac{2}{5}u^5 + C = \frac{2}{3}\sin^3(x) - \frac{2}{5}\sin^5(x) + C$.
\end{itemize}
\paragraph{Tricks Used:}
\begin{itemize}
    \item \textbf{Identity Conversion:} Using $\sin^2\theta + \cos^2\theta = 1$ is the core trick.
    \item \textbf{"Save One" Strategy:} Saving a single factor to serve as the du in the subsequent u-substitution.
\end{itemize}

\subsubsection{Problem Type 2: Powers of Sine and Cosine (Both are EVEN)}
If both powers are even, you must use the half-angle (or power-reducing) identities repeatedly.
\paragraph{Example (Q4):} $ \int_0^{\pi/2} 9\sin^2(x)\cos^2(x) \,dx $
\paragraph{Solution Strategy:}
\begin{itemize}
    \item Both powers (2) are even.
    \item Use the identities: $\sin^2(x) = \frac{1-\cos(2x)}{2}$ and $\cos^2(x) = \frac{1+\cos(2x)}{2}$.
    \item Substitute: $9 \int_0^{\pi/2} (\frac{1-\cos(2x)}{2})(\frac{1+\cos(2x)}{2}) \,dx = \frac{9}{4} \int_0^{\pi/2} (1-\cos^2(2x)) \,dx$.
    \item Notice we have another even power, $\cos^2(2x)$. Apply the half-angle identity again: $\cos^2(2x) = \frac{1+\cos(4x)}{2}$.
    \item Substitute and simplify: $\frac{9}{4} \int_0^{\pi/2} (1 - \frac{1+\cos(4x)}{2}) \,dx = \frac{9}{8} \int_0^{\pi/2} (1-\cos(4x)) \,dx$.
    \item Integrate and evaluate.
\end{itemize}
\paragraph{Tricks Used:}
\begin{itemize}
    \item \textbf{Half-Angle/Power-Reducing Identities:} This is the essential tool for this case.
    \item \textbf{Repeated Application:} Recognizing that you may need to apply the identities more than once.
    \item An alternative trick for this specific example is to use $\sin(x)\cos(x) = \frac{1}{2}\sin(2x)$, which simplifies the integrand to $\frac{9}{4}\int \sin^2(2x) \,dx$ before using the half-angle identity once.
\end{itemize}

\subsubsection{Problem Type 3: Powers of Tangent and Secant}
There are two main sub-cases here.
\paragraph{Case A: Power of Secant is EVEN (and $\ge 2$)}
\subparagraph{Example (Q10):} $ \int 11\tan^4(x)\sec^6(x) \,dx $
\subparagraph{Strategy:} Save a $\sec^2(x)$ factor. Convert the remaining secants to tangents using $\sec^2(x) = 1+\tan^2(x)$. Use $u=\tan(x)$.
\[ \int 11\tan^4(x)\sec^4(x)\sec^2(x) \,dx = \int 11\tan^4(x)(1+\tan^2x)^2\sec^2(x) \,dx. \text{ Let } u=\tan x. \]
\paragraph{Case B: Power of Tangent is ODD (and secant exists)}
\subparagraph{Example (Q8):} $ \int 4\tan(x)\sec^3(x) \,dx $
\subparagraph{Strategy:} Save a $\sec(x)\tan(x)$ factor. Convert the remaining tangents to secants using $\tan^2(x) = \sec^2(x)-1$. Use $u=\sec(x)$.
\[ \int 4\sec^2(x)(\sec(x)\tan(x)) \,dx. \text{ Let } u=\sec x. \]
\paragraph{Tricks Used:}
\begin{itemize}
    \item \textbf{Secant/Tangent Pythagorean Identity:} The core tool is $\tan^2(x)+1 = \sec^2(x)$.
    \item \textbf{Strategic Factoring:} Saving the correct factor ($\sec^2x$ or $\sec x \tan x$) to serve as du.
\end{itemize}

\subsubsection{Problem Type 4: Product-to-Sum Integrals}
For integrals of $\sin(mx)\cos(nx)$, $\sin(mx)\sin(nx)$, or $\cos(mx)\cos(nx)$.
\paragraph{Example (Q17):} $ \int \sin(8x)\cos(5x) \,dx $
\paragraph{Strategy:} Use the identity $\sin A \cos B = \frac{1}{2}[\sin(A-B) + \sin(A+B)]$.
\[ \int \frac{1}{2}[\sin(3x) + \sin(13x)] \,dx. \text{ This is now a simple integral.} \]
\paragraph{Tricks Used:}
\begin{itemize}
    \item \textbf{Product-to-Sum Identities:} These must be known or looked up. They convert a difficult product into a simple sum.
\end{itemize}

\subsection{Other Common Problem Types and Tricks}
\subsubsection{Problem Type 5: Powers of Tangent and Secant (ODD secant, EVEN tangent)}
This is the hardest case and often requires integration by parts.
\paragraph{Example (Q18):} $ \int 5\tan^2(x)\sec(x) \,dx $
\paragraph{Solution Strategy:}
\begin{itemize}
    \item Convert $\tan^2(x)$ to $\sec^2(x)-1$.
    \item $\int 5(\sec^2(x)-1)\sec(x) \,dx = 5\int (\sec^3(x) - \sec(x)) \,dx$.
    \item The integral of $\sec(x)$ is a standard result: $\ln|\sec(x)+\tan(x)|$.
    \item The integral of $\sec^3(x)$ is a famous and tricky one that requires integration by parts (let $u=\sec x, dv=\sec^2x dx$). This will lead to a looping result similar to the $e^x\sin x$ type.
\end{itemize}
\paragraph{Tricks Used:}
\begin{itemize}
    \item \textbf{Identity Conversion:} Starting with the Pythagorean identity.
    \item \textbf{Decomposition into Known Integrals:} Breaking the problem down.
    \item \textbf{IBP for Trig Functions:} Applying integration by parts to solve a purely trigonometric integral.
\end{itemize}

\section{Section 7.3: Trigonometric Substitution}
This technique is used to evaluate integrals containing radical expressions like $\sqrt{a^2-x^2}$, $\sqrt{a^2+x^2}$, and $\sqrt{x^2-a^2}$. The substitution turns the radical into a simple trig function using Pythagorean identities.

\subsection{Problem Types and Tricks in Your Homework}
The problem types are defined by the form of the radical.
\subsubsection{Problem Type 1: Form $ \sqrt{a^2+x^2} $ (or $a^2+x^2$)}
\paragraph{Example (Q4):} $ \int \frac{x^3}{\sqrt{16+x^2}} \,dx $
\paragraph{Solution Strategy:}
\begin{itemize}
    \item Identify $a^2=16 \Rightarrow a=4$. This is the form $a^2+x^2$.
    \item \textbf{Substitution:} Let $x = 4\tan(\theta)$. Then $dx=4\sec^2(\theta)d\theta$.
    \item Simplify the radical: $\sqrt{16+16\tan^2\theta} = \sqrt{16\sec^2\theta} = 4\sec\theta$.
    \item Substitute everything into the integral: $\int \frac{(4\tan\theta)^3}{4\sec\theta} (4\sec^2\theta) \,d\theta$.
    \item Simplify and solve the resulting trig integral (Section 7.2 skills).
    \item \textbf{Draw the Reference Triangle:} From $x=4\tan\theta \Rightarrow \tan\theta = x/4$. This triangle has opposite side `x`, adjacent side `4`, and hypotenuse $\sqrt{x^2+16}$.
    \item Use the triangle to substitute back from $\theta$ to $x$.
\end{itemize}
\paragraph{Tricks Used:}
\begin{itemize}
    \item \textbf{Choosing the Correct Substitution:} `tan` for sum of squares.
    \item \textbf{Reference Triangle:} This is the crucial, non-negotiable step for converting the answer back to the original variable `x`.
\end{itemize}

\subsubsection{Problem Type 2: Form $ \sqrt{x^2-a^2} $ (or $x^2-a^2$)}
\paragraph{Example (Q5):} $ \int \frac{\sqrt{4x^2-25}}{x} \,dx $
\paragraph{Solution Strategy:}
\begin{itemize}
    \item Rewrite as $\sqrt{(2x)^2-5^2}$. Let $u=2x$ or substitute directly. Let's use $2x$.
    \item Identify $a^2=25 \Rightarrow a=5$.
    \item \textbf{Substitution:} Let $2x = 5\sec(\theta)$. Then $2dx = 5\sec\theta\tan\theta d\theta$. Also $x=\frac{5}{2}\sec\theta$.
    \item Simplify the radical: $\sqrt{25\sec^2\theta - 25} = \sqrt{25\tan^2\theta} = 5\tan\theta$.
    \item Substitute, solve the trig integral, draw the triangle (from $\sec\theta = 2x/5$), and convert back.
\end{itemize}
\paragraph{Tricks Used:}
\begin{itemize}
    \item \textbf{Choosing the Correct Substitution:} `sec` for (variable squared) - (constant squared).
    \item \textbf{Algebraic pre-processing:} Recognizing $4x^2$ as $(2x)^2$.
\end{itemize}

\subsubsection{Problem Type 3: Form $ \sqrt{a^2-x^2} $ (or $a^2-x^2$)}
\paragraph{Example (from Q14):} $ \int \frac{x^2}{\sqrt{49-x^2}} \,dx $
\paragraph{Solution Strategy:}
\begin{itemize}
    \item Identify $a^2=49 \Rightarrow a=7$.
    \item \textbf{Substitution:} Let $x = 7\sin(\theta)$. Then $dx=7\cos(\theta)d\theta$.
    \item Simplify the radical: $\sqrt{49-49\sin^2\theta} = \sqrt{49\cos^2\theta} = 7\cos\theta$.
    \item Substitute, solve the trig integral, draw the triangle (from $\sin\theta = x/7$), and convert back.
\end{itemize}
\paragraph{Tricks Used:}
\begin{itemize}
    \item \textbf{Choosing the Correct Substitution:} `sin` for (constant squared) - (variable squared).
\end{itemize}

\subsection{Other Common Problem Types and Tricks}
\subsubsection{Problem Type 4: Trig Substitution after Completing the Square}
This combines the trick from U-Sub with the methods of this section. It's used for integrals with a full quadratic under the radical.
\paragraph{Example:} $ \int \frac{1}{\sqrt{x^2-4x+13}} \,dx $
\paragraph{Solution Strategy:}
\begin{itemize}
    \item Complete the square: $x^2-4x+4-4+13 = (x-2)^2 + 9$.
    \item The integral becomes: $\int \frac{1}{\sqrt{(x-2)^2 + 3^2}} \,dx$.
    \item First, do a simple u-sub: let $u=x-2, du=dx$. Integral becomes $\int \frac{1}{\sqrt{u^2+3^2}}du$.
    \item Now, perform a trig substitution on `u`: let $u = 3\tan\theta$.
    \item Solve the resulting simple trig integral, convert back to `u` using a triangle, then convert back to `x`.
\end{itemize}
\paragraph{Tricks Used:}
\begin{itemize}
    \item \textbf{Completing the Square:} The essential first step to reveal the trig sub form.
    \item \textbf{Multi-step Substitution:} A u-substitution followed by a trig substitution.
\end{itemize}

\section{Section 7.4: Partial Fraction Decomposition (PFD)}
This is an algebraic technique to break down complex rational functions (polynomials in the numerator and denominator) into simpler fractions that are easy to integrate.

\subsection{Problem Types and Tricks in Your Homework}
\subsubsection{Problem Type 1: Improper Fractions (requiring Long Division)}
If the degree of the numerator is greater than or equal to the degree of the denominator, you MUST perform polynomial long division first.
\paragraph{Example (Q14):} $ \int \frac{x^2}{x-5} \,dx $
\paragraph{Solution Strategy:}
\begin{itemize}
    \item Degree of top (2) $\ge$ Degree of bottom (1). Perform long division.
    \item $x^2 \div (x-5)$ gives a quotient of $x+5$ and a remainder of $25$.
    \item So, $\frac{x^2}{x-5} = x+5 + \frac{25}{x-5}$.
    \item Integrate the new expression: $\int (x+5 + \frac{25}{x-5}) \,dx = \frac{1}{2}x^2 + 5x + 25\ln|x-5| + C$.
\end{itemize}
\paragraph{Tricks Used:}
\begin{itemize}
    \item \textbf{Polynomial Long Division:} The critical first step for any improper rational function.
\end{itemize}

\subsubsection{Problem Type 2: Distinct Linear Factors in Denominator}
This is the simplest PFD case. For each factor $(x-a)$, the decomposition gets a term $\frac{A}{x-a}$.
\paragraph{Example (Q4):} $ \int \frac{3}{(x-1)(x+2)} \,dx $
\paragraph{Solution Strategy:}
\begin{itemize}
    \item Decomposition: $\frac{3}{(x-1)(x+2)} = \frac{A}{x-1} + \frac{B}{x+2}$.
    \item Multiply by the common denominator: $3 = A(x+2) + B(x-1)$.
    \item Solve for A and B. A quick way is the \textbf{Heaviside Cover-up Method}:
    \begin{itemize}
        \item To find A, let $x=1$: $3 = A(1+2) + B(0) \Rightarrow 3=3A \Rightarrow A=1$.
        \item To find B, let $x=-2$: $3 = A(0) + B(-2-1) \Rightarrow 3=-3B \Rightarrow B=-1$.
    \end{itemize}
    \item Integrate the simple fractions: $\int (\frac{1}{x-1} - \frac{1}{x+2}) \,dx = \ln|x-1| - \ln|x+2| + C$.
\end{itemize}
\paragraph{Tricks Used:}
\begin{itemize}
    \item \textbf{Heaviside Cover-up Method:} A very fast way to find coefficients for distinct linear factors.
\end{itemize}

\subsubsection{Problem Type 3: Irreducible Quadratic Factors in Denominator}
If the denominator has a factor like $x^2+c^2$ that cannot be factored further, its term in the decomposition is $\frac{Ax+B}{x^2+c^2}$.
\paragraph{Example (Q8):} $ \int \frac{17}{(x-1)(x^2+16)} \,dx $
\paragraph{Solution Strategy:}
\begin{itemize}
    \item Decomposition: $\frac{17}{(x-1)(x^2+16)} = \frac{A}{x-1} + \frac{Bx+C}{x^2+16}$.
    \item Multiply by the common denominator and group terms: $17 = A(x^2+16) + (Bx+C)(x-1)$.
    \item You can't use the cover-up method for B and C. You must expand and equate coefficients of powers of `x`:
    \begin{itemize}
        \item $x^2$ terms: $0 = A+B$.
        \item $x$ terms: $0 = -B+C$.
        \item Constant terms: $17 = 16A-C$.
    \end{itemize}
    \item Solve this system of 3 equations.
    \item Integrate the resulting terms. The term with the irreducible quadratic will often split into a `ln` (from a u-sub) and an `arctan` (see PFD Trick 1 below).
\end{itemize}
\paragraph{Tricks Used:}
\begin{itemize}
    \item \textbf{Equating Coefficients:} The general method for finding unknown constants when the cover-up method isn't fully applicable.
\end{itemize}

\subsubsection{Problem Type 4: Repeated Irreducible Quadratic Factors}
If you have a factor like $(x^2+c^2)^n$, you need a term for each power up to `n`.
\paragraph{Example (Q10):} $ \int \frac{5x^2+x+5}{(x^2+1)^2} \,dx $
\paragraph{Solution Strategy:}
\begin{itemize}
    \item Decomposition: $\frac{Ax+B}{x^2+1} + \frac{Cx+D}{(x^2+1)^2}$.
    \item Solve the system of equations for A, B, C, D.
    \item Integrate each term. The integral of the $(x^2+1)^2$ term may require trig substitution.
\end{itemize}
\paragraph{Tricks Used:}
\begin{itemize}
    \item \textbf{Systematic Decomposition:} Remembering to include a term for each power of the repeated factor.
\end{itemize}

\subsection{Other Common Problem Types and Tricks}
\subsubsection{PFD Trick 1: Splitting the Irreducible Quadratic Integral}
When you integrate a term like $\int \frac{Bx+C}{x^2+a^2} \,dx$, you almost always split it.
\paragraph{Strategy:} $\int \frac{Bx}{x^2+a^2} \,dx + \int \frac{C}{x^2+a^2} \,dx$.
\begin{itemize}
    \item The first part is solved with a u-sub, $u=x^2+a^2$, and becomes a natural log.
    \item The second part is a direct arctangent integral.
\end{itemize}

\subsubsection{PFD Trick 2: Repeated Linear Factors}
Similar to repeated quadratics, a factor of $(x-a)^n$ requires terms for each power.
\paragraph{Example:} $\frac{x+5}{(x-2)^3}$
\paragraph{Decomposition:} $\frac{A}{x-2} + \frac{B}{(x-2)^2} + \frac{C}{(x-2)^3}$.

\section{Sections 6.1, 6.2, 6.3: Area and Volume Applications}
These sections are about setting up the correct integral based on geometry. The integration itself often uses simpler techniques, but the setup is the key challenge.

\subsection{Section 6.1: Area Between Curves}
\subsubsection{Problem Type 1: Integrating with Respect to x (Top - Bottom)}
Used when the region is clearly bounded by a function on the top and a function on the bottom.
\paragraph{Formula:} $ A = \int_a^b [y_{\text{top}}(x) - y_{\text{bottom}}(x)] \,dx $
\paragraph{Example (Q1):} Bounded by $y=5x-x^2$ and $y=x$.
\paragraph{Strategy:}
\begin{itemize}
    \item Find intersection points: $5x-x^2 = x \Rightarrow 4x-x^2=0 \Rightarrow x(4-x)=0 \Rightarrow x=0, 4$. These are your limits `a` and `b`.
    \item Determine which function is on top in the interval. Pick a test point like $x=1$: $y=5(1)-1^2=4$ and $y=1$. The parabola is on top.
    \item Set up the integral: $A = \int_0^4 [(5x-x^2) - (x)] \,dx = \int_0^4 (4x-x^2) \,dx$.
\end{itemize}
\paragraph{Tricks Used:}
\begin{itemize}
    \item \textbf{Finding Intersections:} Setting the functions equal to each other to find the limits of integration.
    \item \textbf{Test Points:} Plugging in a value within the interval to determine the top vs. bottom function.
\end{itemize}

\subsubsection{Problem Type 2: Integrating with Respect to y (Right - Left)}
Used when integrating with respect to `x` would require multiple integrals, but the region has a consistent right and left boundary.
\paragraph{Formula:} $ A = \int_c^d [x_{\text{right}}(y) - x_{\text{left}}(y)] \,dy $
\paragraph{Example (Q7):} Bounded by $x=y^4$ and $x=2-y^2$.
\paragraph{Strategy:}
\begin{itemize}
    \item This is much harder to do with respect to `x`. It's set up for `y`.
    \item Find intersections: $y^4 = 2-y^2 \Rightarrow y^4+y^2-2=0 \Rightarrow (y^2+2)(y^2-1)=0 \Rightarrow y=\pm 1$. These are your limits `c` and `d`.
    \item Determine right vs. left. At $y=0$, $x=0$ and $x=2$. So $x=2-y^2$ is the right function.
    \item Set up integral: $A = \int_{-1}^1 [(2-y^2) - (y^4)] \,dy$.
\end{itemize}
\paragraph{Tricks Used:}
\begin{itemize}
    \item \textbf{Changing Perspective:} Deciding to integrate with respect to `y` when it's simpler. This requires solving equations for `x` in terms of `y`.
\end{itemize}

\subsection{Sections 6.2 \& 6.3: Volumes of Revolution}
The most important decision is choosing the method: Disk/Washer or Cylindrical Shells.
\begin{itemize}
    \item \textbf{Disk/Washer:} Use when your representative rectangle is \textbf{PERPENDICULAR} to the axis of rotation.
    \item \textbf{Shells:} Use when your representative rectangle is \textbf{PARALLEL} to the axis of rotation.
\end{itemize}

\subsubsection*{Key Trick: Finding the Radius R(x) or r(x)}
This is the most common point of confusion.
\begin{itemize}
    \item \textbf{If rotating around the x-axis ($y=0$):} The radius is simply the function value, $R(x)=f(x)$.
    \item \textbf{If rotating around a horizontal line $y=k$:}
    \begin{itemize}
        \item If the region is \textit{above} the line, the radius is `function - k`.
        \item If the region is \textit{below} the line, the radius is `k - function`.
        \item \textbf{General Rule:} Radius is always the `(Farther Boundary) - (Closer Boundary)`. For the outer radius $R(x)$, it's $|y_{\text{top}} - k|$. For the inner radius $r(x)$, it's $|y_{\text{bottom}} - k|$.
    \end{itemize}
    \item \textbf{If rotating around the y-axis ($x=0$):} The radius is $R(y)=f(y)$.
    \item \textbf{If rotating around a vertical line $x=k$:}
    \begin{itemize}
        \item The same logic applies. The radius is $|x_{\text{right}} - k|$ or $|x_{\text{left}} - k|$.
    \end{itemize}
\end{itemize}

\subsubsection{Problem Type 1: Disk Method (Rotation about x- or y-axis)}
This is a washer with an inner radius of 0.
\paragraph{Example (from Q3):} Region under $y=x+1$ from $x=0$ to $x=2$, rotated about the x-axis.
\paragraph{Strategy (Washer/Disk):}
\begin{itemize}
    \item Axis is horizontal (x-axis), so integrate w.r.t `x`. Rectangle is vertical (perpendicular).
    \item Radius $R(x) = (x+1) - 0 = x+1$.
    \item Volume: $V = \int_0^2 \pi [R(x)]^2 \,dx = \int_0^2 \pi (x+1)^2 \,dx$.
\end{itemize}

\subsubsection{Problem Type 2: Washer Method (Rotation about off-axis line)}
\paragraph{Example (Q10):} Region between $y=x^2$ and $x=y^2$ (or $y=\sqrt{x}$), rotated about $y=1$.
\paragraph{Strategy (Washer/Disk):}
\begin{itemize}
    \item Axis is horizontal ($y=1$), so integrate w.r.t `x`. Rectangle is vertical.
    \item Outer Radius $R(x)$: The axis is $y=1$. The bottom curve is $y=x^2$. This is farther from the axis. Radius is `(axis) - (curve)` = $1-x^2$.
    \item Inner Radius $r(x)$: The top curve is $y=\sqrt{x}$. This is closer to the axis. Radius is `(axis) - (curve)` = $1-\sqrt{x}$.
    \item Volume: $V = \int_0^1 \pi [(1-x^2)^2 - (1-\sqrt{x})^2] \,dx$.
\end{itemize}

\subsubsection{Problem Type 3: Cylindrical Shells Method}
Sometimes this method avoids solving for y in terms of x (or vice-versa).
\paragraph{Example (Q6):} Region under $y=x^3$ from $x=1$ to $x=3$, rotated about the y-axis.
\paragraph{Strategy (Shells):}
\begin{itemize}
    \item Axis is vertical (y-axis). To use shells, we need a rectangle parallel to the axis, so a vertical rectangle, meaning we integrate w.r.t `x`.
    \item Shell radius: The distance from the y-axis to the rectangle is simply `x`.
    \item Shell height: The height of the rectangle is the function value, `h(x) = x^3`.
    \item Volume: $V = \int_1^3 2\pi (\text{radius})(\text{height}) \,dx = \int_1^3 2\pi (x)(x^3) \,dx = \int_1^3 2\pi x^4 \,dx$.
\end{itemize}

\subsubsection{Problem Type 4: Shells with Off-Axis Rotation}
\paragraph{Example (Q12):} Region bounded by $y=x^3, y=8, x=0$, rotated about $x=7$.
\paragraph{Strategy (Shells):}
\begin{itemize}
    \item Axis is vertical ($x=7$). For shells, we need a parallel (vertical) rectangle. Integrate w.r.t `x`.
    \item Shell radius: The distance from the axis of rotation ($x=7$) to the rectangle at $x$ is $7-x$.
    \item Shell height: The height of the rectangle is `top curve - bottom curve` = $8-x^3$.
    \item Volume: $V = \int_0^2 2\pi (7-x)(8-x^3) \,dx$.
\end{itemize}
\paragraph{Tricks Used:}
\begin{itemize}
    \item \textbf{Choosing the right method:} Ask yourself, "Is it easier to express functions in terms of x or y?" and "Does one method avoid multiple integrals?"
    \item \textbf{Correctly defining the radius:} This is the most critical trick. Always think of it as a distance: `|curve - axis of rotation|`.
    \item \textbf{Correctly defining the height/width:} For washers, it's `dx` or `dy`. For shells, the height `h(x)` is `top-bottom` and the width is `dx`.
\end{itemize}


\end{document}

































\end{document}
